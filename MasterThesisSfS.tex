\newif\ifdraft %XXX
\drafttrue % \draftfalse %XXX

%%%--- Template for master thesis at SfS
%%%--- Modified template with more comments and examples -- SG, 11/06/09
%%%------
\ifdraft \documentclass[11pt,a4paper]{report} \else%XXX
\documentclass[11pt,a4paper,twoside,openright]{report}
\fi %XXX
\usepackage[english]{sty/ETHDAsfs}%--> ETHDASA + fancyhdr + ... "umlaute"
%  + sfs-hyper -> hyperref 

\usepackage{pdfpages}%%to include the confirmation of originality (plagiarism
\usepackage{amsbsy}%% for \boldsymbol and \pmb{.}
\usepackage{amssymb}%% calls  amsfonts...
\usepackage{graphicx}%-- für PostScript-Grafiken (besser als  psfig!)
%\usepackage[draft]{graphicx} % grafics shown as boxes --> faster compilation
%
\usepackage[longnamesfirst]{natbib}%was {sfsbib}%- Für  Literatur-Referenzen
%           ^^^^^^^^^^^^^^ 1) "Hampel, Ronchetti, ..,"  2) "Hampel et al"
% Engineers (and other funny people) want to see [1], [2] 
% ---> use 'numbers' : \usepackage[longnamesfirst,number]{natbib}
%
%
\usepackage{sty/texab}%- 'tex Abkürzungen' /u/sfs/tex/tex/latex/texab.sty
        %%- z.B.  \R, \Z, \Q, \Nat für reelle, ganze, rationale, natürl. Zahlen;
        %%-       \N   (Normalvert.)  \W == Wahrscheinlichkeit .....
        %%-  \med, \var, \Cov, \....
        %%-  \abs{x} == |x|   und   \norm{y} ==  || y ||   (aber anständig)
%% NOTE: texab contains many useful definitions and "shortcuts". It is
%% worth to open the file and have a look at them. HOWEVER, some
%% definitions are a bit can lead to conflicts with other packages. You
%% might for example want to comment out the line defininf \IF as an
%% operator when working with the algorithmic package, or to comment out
%% the line defining a command \Cite with working with the Biblatex package  
\usepackage{amsmath}
%\usepackage{mathrsfs}% Raph Smith's Formal Script font --> provides \mathscr
\usepackage[utf8]{inputenc}% <<------- Unicode, *NOT* iso-latin1 !
\usepackage{ae}% A[lmost] E[uropean] Fonts
\usepackage{enumerate}% Fuer selbstdefinierte Nummerierungen
%--------
\usepackage{relsize}%-> \smaller (etc) used here
\usepackage{color} %% to allow coloring in code listings
\usepackage{pgf}
\usepackage{listings}% Fuer R-code, C-code, ....  and settings for these:
% listings
  \definecolor{Mygrey}{gray}{0.75}% for linenumbers and only!
  \definecolor{Cgrey}{gray}{0.4}% for comments
  \lstloadlanguages{R}
  %%--- first version of "listings of R"-style : ---------------------------
  % %% using \smaller here: makes R code listings use a *small* font:
  % \lstset{language=R,basicstyle=\smaller[2],commentstyle=\rmfamily\smaller,
  %   showstringspaces=false,xleftmargin=4ex,
  %   literate={<-}{{$\leftarrow$}}1 {~}{{$\sim$}}1}
  % \lstset{escapeinside={(*}{*)}} % for (*\ref{ }*) inside lstlistings (Scode) 
  %\newcommand{\lil}[1]{\lstinline|#1|}
  %%--- newer version of "listings of R"-style : ---------------------------
  \lstset{%% Help, e.g. --> https://en.wikibooks.org/wiki/LaTeX/Source_Code_Listings
    language=R,
    basicstyle=\ttfamily\scriptsize,%%- \small > \footnotesize > \scriptsize > \tiny
    %commentstyle=\ttfamily\color{Cgrey},
    commentstyle=\itshape\color{Cgrey},
    numbers=left,
    numberstyle=\ttfamily\color{Mygrey}\tiny,
    stepnumber=1,
    numbersep=5pt,
    backgroundcolor=\color{white},
    showspaces=false,
    showstringspaces=false,
    showtabs=false,
    frame=single,
    tabsize=2,
    captionpos=b,
    breaklines=true,
    %breakatwhitespace=false,
    keywordstyle={},
    morekeywords={},
    xleftmargin=4ex, 
    literate={<-}{{$\leftarrow$}}1 {~}{{$\sim$}}1
  }
  \lstset{escapeinside={(*}{*)}} % for (*\ref{ }*) inside lstlistings (Scode) 
%%----------------------------------------------------------------------------

%%------- Theoreme ---
\newtheorem{definition}{Definition}[subsection]
\newtheorem{lemma}[definition]{Lemma}
\newtheorem{theorem}[definition]{Theorem}
\newtheorem{Coro}[definition]{Corollary}
\theoremstyle{definition} 
\newtheorem{example}[definition]{Example}
\newtheorem*{note}{Note}
\newtheorem*{remark}{Remark}

\DeclareMathOperator*{\plim}{plim}
% \def\MR#1{\href{http://www.ams.org/mathscinet-getitem?mr=#1}{MR#1}}

% \newcommand{\Lecture}[3]{\marginpar{#3.#2.#1}}
% \newcommand{\Fu}{\mathcal{F}}
\newcommand{\aatop}[2]{\genfrac{}{}{0pt}{}{#1}{#2}}

%\renewcommand{\theequation}{\arabic{equation}}
\numberwithin{equation}{subsection}

%%%%%%%%%%%%%%%%%%%%%%%%%%%%%%%%%%%%%%%%%%%%%%%%%
%%% Path for your figures                      %%%
%%%%%%%%%%%%%%%%%%%%%%%%%%%%%%%%%%%%%%%%%%%%%%%%%
% Set the paths where all figures are taken from:
\graphicspath{{./figures/}}

%%%%%%%%%%%%%%%%%%%%%%%%%%%%%%%%%%%%%%%%%%%%%%%%%
%%% Define your own commands here             %%%
%%%%%%%%%%%%%%%%%%%%%%%%%%%%%%%%%%%%%%%%%%%%%%%%%
\newcommand{\Bruch}[2]{{}^{#1}\!\!/\!_{#2}}
\renewcommand{\labelenumi}{\roman{enumi}.)}
\ifdraft \usepackage{lineno} \linenumbers  %XXX
\definecolor{mygray}{gray}{0.75}
\renewcommand{\linenumberfont}{\normalfont\scriptsize\color{mygray}}
\fi %XXX

\usepackage{my_modifications}




\begin{document}
\bibliographystyle{chicago}% ---> Hampel,F., E.Ronchetti,... W.Stahel(1986) ...
%was \bibliographystyle{sfsbib}\citationstyle{dcu} %OR DEFAULT : \citationstyle{agsm}

\pagenumbering{roman}%- roman numbering for first few pages

%%%%%%%%%%%%%%%%%%%%%%%%%%%%%%%%%%%%%%%%%%%%%%%%%
%%% Title page                                %%%
%%%%%%%%%%%%%%%%%%%%%%%%%%%%%%%%%%%%%%%%%%%%%%%%%
\period{Spring 2022}
\dasatype{Master Thesis}
\students{Lukas Graz}
\mainreaderprefix{Adviser:}
\mainreader{Prof.\ Dr.\ Nicolai Meinshausen}
\alternatereaderprefix{Co-Adviser:}
\alternatereader{Gregor Perich}
\submissiondate{September 18th 2022}
\title{Interpolation and Correction \\ of \\ Multispectral Satellite Image Time Series}

\maketitle%- Titelseite wird abgeschlossen
\ifdraft \else%XXX
  \cleardoublepage
\fi %XXX
%%~~~~~~~~~~~~~~~~~~~~~~~~~~~~~~~~~~~~~~~~

%%%%%%%%%%%%%%%%%%%%%%%%%%%%%%%%%%%%%%%%%%%%%%%%%
%%% Insert here acknowledgements and abstract %%%
%%%%%%%%%%%%%%%%%%%%%%%%%%%%%%%%%%%%%%%%%%%%%%%%%
%% Dedication (optional)

  % \markright{}
  % \vspace*{\stretch{1}}
  % \begin{center}
  %   To some special person
  % \end{center}
  % \vspace*{\stretch{2}}

  % Preface (optional)
  \newpage
  \markboth{Preface}{Preface}
  \chapter*{Preface}
\section*{Complementary Material}


Github: 


\subsection*{Acknowlegements}
Betreuung von Gregor

Ideen mit Meinshausen

Resourccen vom SFS

%%% Local Variables: 
%%% mode: latex
%%% TeX-master: "MasterThesisSfS"
%%% End: 


  % Abstract should not be longer than one page.
  \newpage
  \markboth{Abstract}{Abstract}
  \chapter*{Abstract}

%% Intro
Multispectral satellite imagery Time Series (TS) are utilized to estimate TS of spectral indices at the ground. As such, the TS of the Normalized Difference Vegetation Index (NDVI) is used to model vegetation development. 
%% Problem Illustration
Due to atmospheric effects (e.g., clouds or shadows) satellite measurements may not match the ground signal. Therefore, traditional approaches try to filter out contaminated observations before extracting and subsequently interpolating the NDVI. After filtering, remaining contaminated observations and resulting data gaps are the two challenges for interpolation that we address in this thesis.

%% Our Setting
For this purpose, we use crop yield maps from 2017-2021 of cereals from a farm in Switzerland and corresponding Sentinel 2 satellite image TS published by the European Space Agency. Contaminated observations can be filtered with the provided Scene Classification Layer (SCL). 

%% NDVI itpl
We give a benchmark-supported review of different interpolation methods and opt for Smoothing Splines as a flexible non-parametric method and Double Logistic approximation as a parametric method with implicit shape assumptions. In addition, we generalize an iterative technique which robustifies interpolation methods against outliers by reducing their weight. In most cases, this robustification successfully decreased the 50\% and 75\% quantiles of the absolute out-of-bag residuals. 

%% NDVI corr. 
Moreover, we present a general interpolation procedure that utilizes additional information to correct the target variable with an uncertainty estimate and then performs a weighted interpolation. In our setting, the target variable is the NDVI and as additional information we use the SCL, the observed NDVI and the spectral bands. Consequently, we do not filter using the SCL but weight observations according to their reliability. The combination of different interpolation methods and correction models yields 28 interpolation strategies. In order to choose the best one, we assume that the better the interpolated NDVI TS models crop growth, the more suitable it is to predict crop yield. 
% {The resulting interpolation strategy uses Smoothing Splines and corrects the NDVI with uncertainty estimation through a simple linear model considering only of the observed NDVI and the associated SCL class.} 
Applying this procedure, the variance in crop yield explained by the resulting NDVI TS decreases by more than 5\%. 

%% Reproducibility  +  R-package
Instructions and a codebase for reproducibility of the results, as well as an R package making the presented general interpolation procedure accessible to the user, are supplied. 



%%% Local Variables: 
%%% mode: latex
%%% TeX-master: "MasterThesisSfS"
%%% End: 


%%%%%%%%%%%%%%%%%%%%%%%%%%%%%%%%%%%%%%%%%%%%%%%%%
%%% Table of contents and list of figures and %%%   
%%% tables (no need to change this usually)   %%%
%%%%%%%%%%%%%%%%%%%%%%%%%%%%%%%%%%%%%%%%%%%%%%%%%
\newpage
\tableofcontents
\listoftodos
\ifdraft 
\else%XXX
  % \newpage
  % \listoffigures
  % \newpage
  % \listoftables

  %% Notations and glossary (optional)
  \cleardoublepage
\fi
\phantomsection
\addcontentsline{toc}{chapter}{\protect\numberline{}{Notation}}
\markboth{Notation}{Notation}
\chapter*{Notations}
\label{c:Notation}

\section*{Variables}

\renewcommand{\arraystretch}{1.3} % for non-dense tables
\begin{tabular}{l l}
$c$		& a (vector of) constant(s)\\
$\lambda \in \R$		& a scalar\\
$n\in \mathbb{N}$		& sample size\\
$i,j$		& are indices in $\{1,\dots,n\}$\\
$x\in \R^n$		& covariable in 1-dim interpolation setting\\
$w \in \R^n$		& a vector of weights for each location $x$\\
$y\in \R^n$		& response in 1-dim interpolation setting\\
$\hat y\in \R^n$		& estimate of $y$\\
$\bar y\in \R$		& sample mean of $y$\\
$r \in \R^n$		& residuals given by $y - \hat y$
\end{tabular}

%%%%%%%%%%%%%%%%%%%%%%%%%%%
\section*{Abbreviations and Objects}
\begin{longtable}{p{0.08\linewidth} p{0.87\linewidth}}
Pixel	
		& A pixel originates of an image pixel and describes a square of 10 x 10 meters in the field which coincides with the resolution (and location) of the Sentinel-2 pixels. Such pixels are illustrated in figure~\ref{fig:satelite/witzwil_2021_P112_yield_cropped.png}. Additional information like yield is also attached.\\

$P_t$	
		& describes the observed data (weather and spectral bands) at time $t$ and the location of one pixel. \\

$P$	
		& is a pixel. We see it as a collection of all the observations at the specified location within one season. More formally, $P := \left\{P_t | t\text{ is a valid sample time within a defined season}\right\}$\\

SCL	
		& Scene Classification Layer provided by the European Space Agency (ESA) that gives an estimation of the land cover class of each pixel. It indicates what one can expect at a pixel at a sampled time. For an overview, c.f. table~\ref{tab:satelite/scl_classes}\\

$P^{SCL45}$	
		& is similar to $P$ but we only consider observations which belong to the classes 4 and 5. This is used done to get a subset of observations which are less contaminated by clouds and shadows.\\

NDVI	
		& Normalized Difference Vegetation Index \citep{rouseMonitoringVernalAdvancement1974}\\

DAS	
		& Days After Sowing\\

GDD	
		& Growing Degree Days -- cumulative sum of ``$\max(0, \text{temperature}-\text{threshold})$''\\

RYEA 	
		& Relative Yield-Estimation-Accuracy. Definition \ref{def:ryea}\\

OOB 	
		& Out Of the Box. Describes the procedure of  estimating the value for a point but not consider the point itself (c.f. section \ref{sec:OOB_LOOCV})\\
\end{longtable} \renewcommand{\arraystretch}{1}


XXX ML models and their shortnames

European Space Agency (ESA)

\subsection*{MATLAB Matrix Notation}{ \label{sec:MATLAB}
		We will use the MATLAB `\texttt{:}' notation to indicate rows and columns of a matrix. That is if $X\in\R^{n\times p}$ is a matrix, then $X_{[:,3]}$ is the $3$rd column of $X$ and $X_{[2,:]}$ is the second row of X. 
}


%%% Local Variables: 
%%% mode: latex
%%% TeX-master: "MasterThesisSfS"
%%% End: 


\ifdraft \else%XXX
  \cleardoublepage
\fi %XXX
\pagenumbering{arabic}%--- switch back to standard numbering 


%%%%%%%%%%%%%%%%%%%%%%%%%%%%%%%%%%%%%%%%%%%%%%%%%
%%% Your text... Either write here directly,  %%%
%%% or even better: write in separate files   %%%
%%% that you just have to include here.       %%% 
%%%%%%%%%%%%%%%%%%%%%%%%%%%%%%%%%%%%%%%%%%%%%%%%%
\chapter{Introduction}

Remote sensing aims to measure target variables efficiently from a distance. 
% stakeholders + applications  
Large scale monitoring of forest and agricultural vegetation dynamics is of great interest to authorities, insurance companies and research. Examples include crop classification for subsidizing farmers \citep{henitsSentinel2EnablesNationwide2022} and the creation of crop models for estimating crop yields or nitrogen concentrations \citep{couraultSTICSCropModel2021,perichCropNitrogenRetrieval2021}. 
% Season Start (start of spring) (community name: land surface plant phenology)
For this, freely distributed multi-spectral satellite imagery from the  Sentinel-2 (S2) satellites are examined \citep{esaSentinel22022}.
% NDVI
In order to transform the high dimensional satellite images into easily interpretable metrics, spectral indices such as the Normalized Difference Vegetation Index (NDVI) are used \citep{rouseMonitoringVernalAdvancement1974}. The NDVI serves as a proxy for photosynthetic activity \citep{gamonRelationshipsNDVICanopy1995a}, and thus the corresponding {NDVI Time Series ({TS})} reflects the vegetation development. 
% S2 issues (clouds ...)
The quality of a satellite image, however, depends on atmospheric conditions. Thus, in case of a dense cloud cover, the information content derived from the NDVI is impaired. Therefore, \cite{esaEuropeanSpaceAgency2022} also provides a Scene Classification Layer (SCL), which provides additional metadata about what is observed (e.g., shadows, clouds, vegetation, etc.) . So when extracting the NDVI {TS} from the Sentinel 2 satellite imagery {TS}, we can filter out the contaminated observations using the SCL classification. However, due to this filtration it may occur that we have no observations for several weeks, especially in winter. It is also possible that some observations are wrongly classified by the SCL (e.g., as vegetation) and thus result in an outlier in the NDVI TS. Consequently, the main challenge is to interpolate an NDVI {TS}, which can contain large data gaps and outliers. 

% state-of-the-art
Currently, there are several approaches to address these issues. One is to look at the observed evolution of the canopy coverage and assume its bell shape for the NDVI {TS} given the strong correlation between NDVI and photosynthetic activity. Approaches to model this include a \nth{2} order Fourier approximation \citep{stockliEuropeanPlantPhenology2004} or a Double Logistic function \citep{beckImprovedMonitoringVegetation2006}.
On the other hand, assumptions are made about more abstract properties of the curve, such as smoothness. We divide these into local and global approaches. Nadaraya-Watson \citep{strbacEstimationEvapotrasnpirationUrban2017}, Savitzky-Golay Filter \citep{chenSimpleMethodReconstructing2004a} and Locally Reweighed Regression \citep{omoriAssessmentPaddyFields2021} use a sliding window to interpolate the {TS} stepwise. Global methods like B-Splines \citep{gurungPredictingEnhancedVegetation2009} and Smoothing Splines \citep{caiPerformanceSmoothingMethods2017} reduce the squares of all residuals simultaneously, and Universal Kriging fits a Gaussian process to the data \citep{chandolaScalableTimeSeries2010}.
% SS defined in \cite{cravenSmoothingNoisyData1978}

\pagebreak
The research questions pursued in this thesis are:
\begin{Nenumerate}
    \item Which {{IM}}s are used in the context of NDVI, and what are their advantages and disadvantages?
    \item How may contaminated data be dealt with?
    \item How do data gaps affect interpolation?
    \item How to deal with data gaps?
    \item How can we recognize a good interpolation of the NDVI?
\end{Nenumerate}
\bigskip



% our contribution + roadmap:
In this thesis, we will discuss the strengths and weaknesses of Interpolation Methods ({{IM}}s) and evaluate them with respect to NDVI interpolation. For this purpose, we use the Sentinel 2 satellite image {TS} and crop yield maps of different fields of different cereal species on a farm in Witzwil, Switzerland over the years 2017-2021. After presenting the available data, illustrating challenges and defining different concepts in chapter~\ref{sec:data_methods} (\nameref{sec:data_methods}), we turn to the two main blocks of this thesis. One covers the study of IMs and the other presents a general procedure of correcting (NDVI) TS with uncertainty estimation by utilizing additional information.
On the first block, in chapter~\ref{sec:itpl} (\nameref{sec:itpl}) we examine parametric and non-parametric {{IM}}s and discuss their strengths and weaknesses (question i.). We generalize and test an iterative technique that makes IMs more robust to outliers by weighting them less (question ii.). To evaluate IMs, we present an approach that uses out-of-bag residuals (question v.). In section~\ref{sec:discussion_itpl_data_gaps} (\nameref{sec:discussion_itpl_data_gaps}), we discuss how different {{IM}}s respond to data gaps (question iii.), and in section~\ref{sec:itpl_preselection} (\nameref{sec:itpl_preselection}) we preselect {{IM}}s. This preselection, we evaluate in the results section~\ref{sec:results_itpl} (\nameref{sec:results_itpl}) and select two candidates from the {{IM}}s in section~\ref{sec:itpl_candiate_selection} (\nameref{sec:itpl_candiate_selection}).
For the second block, we correct possibly contaminated data with statistical models in chapter~\ref{sec:corr} (\nameref{sec:corr}) (question ii.) and utilize previously ignored observations, which we hope will further reduce data gaps (question iv.). Thus, we no longer filter the observations a priori via the SCL, but instead correct the observed NDVI and weight the observations via estimated uncertainties. By combining different statistical models and IMs, we obtain 28 Interpolation Strategies ({{ISs}}). We compare those with a vegetation-oriented quality measure (question v.) and describe the results in section~\ref{sec:results_ndvi_corr} (\nameref{sec:results_ndvi_corr}). Based on these results, in section~\ref{sec:discussion_corr} (\nameref{sec:discussion_corr}) we argue what the best {{IS}} is. In addition, we justify why our NDVI correction can be understood as unsupervised learning and why we relied only on satellite imagery and not on meteorological data for the NDVI correction.
Our conclusions of this thesis, recommendations, as well as an outlook on future work is given in chapter~\ref{sec:Conclusion} (\nameref{sec:Conclusion}). 


% und Definieren verschiedene Konzepte. Darunter eine transformation der Zeitachse, welche datenlücken im Winter zusammenschrumpfen lässt (Frage iv.). 




%smoothing:   ``Similarly, smoothing the {TS} of satellite data is helpful to address inconsistency in observation frequency and timing due to clouds and other sensor artefacts \cite{skakunWinterWheatYield2019}''


%%% Local Variables: 
%%% mode: latex
%%% TeX-master: "MasterThesisSfS"
%%% End: 

\chapter{Data and Methods}\label{sec:data_methods}
{
	We will start by describing the available data and the challenges associated with it.
	Our study region is a farm of over 800ha, which is located in western Switzerland. From \cite{perichPixelbasedYieldMapping2022a} we acquire satellite image data (section \ref{sec:s2_img_data}), yield maps of several cereals from 2017 to 2021 (section \ref{sec:yieldmapping_data}), and meteorological data (section \ref{sec:gather_data_to_pixel}).
	Afterwards, we will introduce general methods in section \ref{sec:general_methods}, which will be used in the remaining chapters.
}


\section{Sentinel 2 Data}{
	\label{sec:s2_img_data}
	%\subsection*{General Information}
	{
		
		The European Space Agency (ESA)\footnote{REF: https://sentinel.esa.int/web/sentinel/missions/sentinel-2} freely distributes the high-quality images of the two Sentinel satellites (S2). Together, both satellites have a revisit time of 5 days at the Equator and 2-3 days at mid-latitudes. However, in our study region, we only receive an image every 5 days.
		
		\begin{table}[h]
    \centering
    \small
    \caption{List of spectral bands of the S2-satellites. Each band has its center at the wavelength $\lambda$ in $nm$ with the spectral width $\Delta\lambda$ in $nm$ with a spatial resolution $SR$ in $m$ \citep{jaramazESASentinel2Mission2013}.}
    \begin{tabular}{p{0.03\linewidth} p{0.04\linewidth} p{0.03\linewidth} p{0.03\linewidth} p{0.73\linewidth}}
    \toprule
        \hspace*{-5pt} Band & $\;\lambda$ & $\Delta\lambda$ & $SR$ & Purpose \\ \hline
        1 & 443 & 20 & 60 & Atmospheric correction (aerosol scattering) \\ %\hline
        2 & 490 & 65 & 10 & Sensitive to vegetation senescing, carotenoid, browning and soil background; atmospheric correction (aerosol scattering) \\ %\hline
        3 & 560 & 35 & 10 & Green peak, sensitive to total chlorophyll in vegetation \\ %\hline
        4 & 665 & 30 & 10 & Maximum chlorophyll absorption \\ %\hline
        5 & 705 & 15 & 20 & Position of red edge; consolidation of atmospheric corrections / fluorescence baseline. \\ %\hline
        6 & 740 & 15 & 20 & Position of red edge, atmospheric correction, retrieval of aerosol load. \\ %\hline
        7 & 783 & 20 & 20 & Leaf Area Index (LAI), edge of the Near-Infrared (NIR) plateau. \\ %\hline
        8 & 842 & 115 & 10 & LAI \\ %\hline
        8a & 865 & 20 & 20 & NIR plateau, sensitive to total chlorophyll, biomass, LAI and protein; water vapor absorption reference; retrieval of aerosol load and type. \\ %\hline
        9 & 945 & 20 & 60 & Water vapor absorption, atmospheric correction. \\ %\hline
        10 & 1375 & 30 & 60 & Detection of thin cirrus for atmospheric correction. \\ %\hline
        11 & 1610 & 90 & 20 & Sensitive to lignin, starch and forest above ground biomass. Snow/ice/cloud separation. \\ %\hline
        12 & 2190 & 180 & 20 & Assessment of Mediterranean vegetation conditions. Distinction of clay soils for the monitoring of soil erosion. Distinction between live biomass, dead biomass and soil, e.g. for burn scars mapping. \\
        \bottomrule
    \end{tabular}
    \label{table:S2-bands}
\end{table}

		
		The S2 images contain 12 spectral bands with spatial resolutions up to 10 meters (see \ref{table:S2-bands}). Bands with a lower resolution (20 and 60 meters) were upscaled to 10 meter resolution using cubic interpolation (\cite{perichPixelbasedYieldMapping2022a}). In order to decrease the effect of atmospheric conditions like reflections and scattering, bottom-of-atmosphere, radiometric corrected Level-2A data was used\footnote{According to \cite{perichPixelbasedYieldMapping2022a}: ``Data prior to March 2018 was only available in the top-of-atmosphere L1C format and was downloaded as such [...] L1C data was processed to L2A product level using the `Sen2Cor' processor provided by ESA''}. 
		The ESA also supplies an algorithm\footnote{REF https://sentinels.copernicus.eu/web/sentinel/technical-guides/sentinel-2-msi/level-2a/algorithm} produces Scene Classification Layer (\textit{SCL}) where for each location the observed subject is assigned to one of 11 SCL-classes (c.f. table~\ref{tab:satelite/scl_classes}). 
		In this thesis,  we will use this classification to filter out data points, which we belive to be less informative. That are all observations which SCL-class does not correspond to vegetation or bare soils (classes 4 and 5). For convenience, we define the set SCL45 as the observations which belong to SCL-class 4 or 5.
		
		% \begin{figure}[h]
		% 	\label{fig:satelite/sentinel-2-bands}
		% 	\center
		% 	\includegraphics[width=0.4\textwidth]{satelite/sentinel-2-bands.jpg}
		% 	\caption{XXX Sentinel 2 bands}
		% \end{figure}
				% \begin{table}[!h]
		% 	\caption{Overview: Scene Classification Layers (SCL)}
		% 	\label{tab:satelite/scl_classes}
		% 	\center
		% 	\includegraphics[width=0.8\textwidth]{satelite/scl_classes.png}
		% \end{table}
		
		\begin{table}[h]
			\caption{Overview: Scene Classification Layers (SCL)}
			\label{tab:satelite/scl_classes}
			\centering
			\small
			\begin{tblr}{
			  colspec = {p{0.05\linewidth} p{0.03\linewidth} p{0.3\linewidth} p{0.05\linewidth} p{0.03\linewidth} p{0.3\linewidth} },
			%   row{2} = {SCL3color},
			%   column{3} = {teal7},
			  cell{2}{1} = {SCL0color},
			  cell{3}{1} = {SCL1color},
			  cell{4}{1} = {SCL2color},
			  cell{5}{1} = {SCL3color},
			  cell{6}{1} = {SCL4color},
			  cell{7}{1} = {SCL5color},
			  cell{2}{4} = {SCL6color},
			  cell{3}{4} = {SCL7color},
			  cell{4}{4} = {SCL8color},
			  cell{5}{4} = {SCL9color},
			  cell{6}{4} = {SCL10color},
			  cell{7}{4} = {SCL11color},
			}
			% \toprule % not working :8
			\hline
			Color & No. & Class & Color & No. & Class \\
			\hline
			& 0: & Missing Data 	& & 6: &  Water\\	 
			& 1: & Saturated or defective pixel 	& & 7: &  Cloud low probability\\
			& 2: & Dark features / Shadows 	& & 8: &  Cloud medium probability\\
			& 3: & Cloud shadows 	& & 9: &  Cloud high probability\\
			& 4: & Vegetation 	& & 10: &  Thin cirrus cloud\\
			& 5: & Bare soils 	& & 11: &  Snow or ice\\
			  \hline
			%   \bottomrule
			\end{tblr}
		  \end{table}


	% 0: "#000000",  Missing Data
	% 1: "#ff0000",	 Saturated or defective pixel
	% 2: "#404040",  Dark features / Shadows
	% 3: "#bf8144",  Cloud shadows
	% 4: "#00ff3c",  Vegetation
	% 5: "#ffed50",  Bare soils
	% 6: "#0d00fa",  Water
	% 7: "#808080",  Cloud low probability
	% 8: "#bfbfbf",  Cloud medium probability
	% 9: "#eeeeee",  Cloud high probability
	% 10: "#0bb8f0", Thin cirrus cloud
	% 11: "#ffbfbf", Snow or ice

		

	}
}

\section{Crop Yield Data}{
	\label{sec:yieldmapping_data}
	The crop yield data were collected using a combine harvester. Equipped with GPS, the harvester drives over the fields and continuously estimates the dry crop yield density in $t/ha$ (see fig. \ref{fig:satelite/witzwil_2021_P112_yield_harvester_cropped}). 
	We take the data set derived in \cite{perichPixelbasedYieldMapping2022a}, where error-prone measurement points (such as during a tight curve of the combine harvester) were removed and then the yield map was rasterized using linear interpolation (c.f. fig. \ref{fig:satelite/witzwil_2021_P112_yield_cropped.png}). We summarize the rasterized dry-yield values by the following statistics:

	% tabelle anstadt histogramm, da schon zu viele figuren ``herumschwirren'' und es so kompakter ist.
	\begin{tabular}{l l l l l l l} 
		Minimum & 1st Quartile & Median & Mean  & 3rd Quartile & Maximum & Variance \\
		0.107   & 6.186        & 7.560  & 7.359 & 8.756        & 13.35   & 4.035
	\end{tabular}    

	Comparing the average per-field crop yield reported by the farmer with the yield estimated by the combine harvester shows that the latter overestimates crop yield by ca. $10\%$ (c.f. \cite{perichPixelbasedYieldMapping2022a}). Since the relative estimation error is approximately constant and we do not aim for an accurate yield prediction, we will not consider this deviation. 



	\begin{figure}
		\centering
		\begin{subfigure}{.5\textwidth}
			\centering
			\includegraphics[height=.75\linewidth]{satelite/witzwil_2021_P112_yield_harvester_cropped.png}
			\caption{Raw combine harvester data (cleaned)}
			\label{fig:satelite/witzwil_2021_P112_yield_harvester_cropped}
		\end{subfigure}%
		\begin{subfigure}{.5\textwidth}
			\centering
			\includegraphics[height=.75\linewidth]{satelite/witzwil_2021_P112_yield_cropped.png}
			\caption{rasterized to Sentinel 2 resolution.}
			\label{fig:satelite/witzwil_2021_P112_yield_cropped.png}
		\end{subfigure}
		\caption{Crop yield density map of a field. Ranges from 0.1 t/ha (black) to 5.35 t/ha (white) }
		\label{fig:satelite_witzwil_yield}
	\end{figure}

}



\section{The Concept of a `Pixel'}{
		\label{sec:gather_data_to_pixel}
		Before we join all the data, we define a few concepts.

	\subsection{Normalized Difference Vegetation Index (NDVI)}{% NDVI
		The well-known  (\textit{NDVI}) introduced in \cite{rouseMonitoringVernalAdvancement1974} can be calculated using the bands $B4$ and $B8$ (table \ref{table:S2-bands}) by:
		\begin{equation}
			NDVI = \frac{B8 - B4}{B8 + B4}
			\label{eq:ndvi}
		\end{equation}
		Note that we call the calculated values merely the \textit{observed NDVI}, as we must be aware of imprecisions due to clouds and shadows\todo{Please clarify this in more detail. We used pixels flagged with SCL 4 \& 5, but as can be seen in Fig. 2.1 d), this can yield erroneous NDVI values, etc.}. 
	}

		{% GDD & DAS
			To define a timescale, we consider Days After Sowing (\textit{DAS}) and a transformed timescale, Growing Degree Days (\textit{GDD}) (\cite{mcmasterGrowingDegreedaysOne1997}REF). The latter are defined as the cumulative sum (since sowing) of temperature above a given base temperature $T_{base}$. For cereals, we use $T_{base}=0$ (\cite{perichPixelbasedYieldMapping2022a}). Thus, the GGD for $n$ days after sowing will be equal to:\todo{Für den Leser wäre es interessant, wenn Du noch kurz die wichtigsten GDD Werte aus der Literatur beschreiben würdest (D.h. z.B. Sowing, Emergence of Plants, Anthesis, Senescence, Harvest)}
			\begin{equation}
				\label{eq:gdd}
				GDD_n := \sum_{i=0}^n \max(T_i - T_{base}, 0).
			\end{equation}
		} 

		Now we create a data set, which will contain all the necessary information\todo{necessary info for what? To answer the research questions asked in section XXX}. Given that we have the spectral data at a $10m \times 10m$ resolution, we introduce the concept of a Pixel. A \textit{Pixel} $P$ is associated with a $10m \times 10m$ square defined by the S2 satellites and contains all relevant information\todo{which?} for a season and this location. More precisely, $P$ is a collection of general information (like yield and coordinates) and all associated $P_t$ of a given season. Where $P_t$ represents a tuple of the spectral data for time $t$, the NDVI calculated from it, and the associated GDD. 
		We will call the resulting data set \textit{PIXELS}, as it is the collection of all Pixels (over all seasons). 
		
		% Finally, we split PIXELS randomly into a train ($80\%$) and test  ($20\%$) set. 
	\subsection{Challenges in S2 Data}{
				%satelite/time_series_2021_P112/15_scl5_2021-02-23.png
	%satelite/time_series_2021_P112/30_scl4_2021-05-09.png
	%satelite/time_series_2021_P112/33_scl9_2021-05-24.png
	%satelite/time_series_2021_P112/35_scl4_2021-06-03.png
	%satelite/time_series_2021_P112/40_scl10_2021-06-28.png
	%satelite/time_series_2021_P112/45_scl2_2021-07-23.png

\begin{figure*}
	\centering
	\begin{subfigure}[b]{0.31\textwidth}
		\centering
		\includegraphics[width=\textwidth]{satelite/time_series_2021_P112/15_scl5_2021-02-23.png}
		\caption[2021-02-23 scl5]%
		{{\small 2021-02-23 scl5}}    
		\label{fig:satelite/time_series_2021_P112/15_scl5_2021-02-23.png}
	\end{subfigure}
	\hfill
	\begin{subfigure}[b]{0.31\textwidth}  
		\centering 
		\includegraphics[width=\textwidth]{satelite/time_series_2021_P112/30_scl4_2021-05-09.png}
		\caption[2021-05-09 scl4]%
		{{\small 2021-05-09 scl4}}    
		\label{fig:satelite/time_series_2021_P112/30_scl4_2021-05-09.png}
	\end{subfigure}
	\hfill
	\begin{subfigure}[b]{0.31\textwidth}  
		\centering 
		\includegraphics[width=\textwidth]{satelite/time_series_2021_P112/33_scl9_2021-05-24.png}
		\caption[2021-05-24 scl9]%
		{{\small 2021-05-24 scl9}}    
		\label{fig:satelite/time_series_2021_P112/33_scl9_2021-05-24.png}
	\end{subfigure}

	\vskip\baselineskip
	\begin{subfigure}[b]{0.31\textwidth}   
		\centering 
		\includegraphics[width=\textwidth]{satelite/time_series_2021_P112/35_scl4_2021-06-03.png}
		\caption[2021-06-03 scl4]%
		{{\small 2021-06-03 scl4}}    
		\label{fig:satelite/time_series_2021_P112/35_scl4_2021-06-03.png}
	\end{subfigure}
	\hfill
	\begin{subfigure}[b]{0.31\textwidth}   
		\centering 
		\includegraphics[width=\textwidth]{satelite/time_series_2021_P112/40_scl10_2021-06-28.png}
		\caption[2021-06-28 scl10]%
		{\small 2021-06-28 scl10}    
		\label{fig:satelite/time_series_2021_P112/40_scl10_2021-06-28.png}
	\end{subfigure}
	\hfill
	\begin{subfigure}[b]{0.31\textwidth}  
		\centering 
		\includegraphics[width=\textwidth]{satelite/time_series_2021_P112/45_scl2_2021-07-23.png}
		\caption[2021-07-23 scl2]%
		{{\small 2021-07-23 scl2}}    
		\label{fig:satelite/time_series_2021_P112/45_scl2_2021-07-23.png}
	\end{subfigure}

	\vskip\baselineskip
	\begin{subfigure}[b]{0.7\textwidth}   
		\centering 
		\includegraphics[width=\textwidth]{interpol/ndvi_ts_scl.pdf}
		\caption[Corresponding NDVI time series]%
		{{\small Corresponding NDVI time series}}    
		\label{fig:interpol/ndvi_ts_scl45_grey.pdf}
	\end{subfigure}

	\caption{Satellite images of a field at selected times with a static background for orientation. Moreover, the NDVI time series of the red-highlighted pixel is shown in (g) colored by the SCL labels.} 
	\label{fig:witzwil_selected_satellite_images}
\end{figure*}


\todo{Hier noch eine NDVI Zeitreihe parallel dazu zeigen. Ansonsten wird nicht klar, warum wir die Interpolation überhaupt machen.}
			% Description of plot
			The figure~\ref{fig:witzwil_selected_satellite_images} shows a selection of 6 satellite images of a field, which display our challenges\todo{which challenges? were they introduced earlier? E.g. in the introduction?}. In February (image a), we see no vegetation but bare soil. At the beginning of May, we observe a cloudless dark green field. In (c) heavy cloud cover (SCL class 9) leads to a complete loss of plant information in this S2 observation. Figure (d) shows that the SCL classification is not reliable, since we evidently observe clouds. In (e) we see a pale green. This likely shimmers through cirrus clouds. 
			
			%% subfigures references:
			% (see. \ref{fig:satelite/time_series_2021_P112/15_scl5_2021-02-23.png})
			% (see. \ref{fig:satelite/time_series_2021_P112/30_scl4_2021-05-09.png})
			% (see. \ref{fig:satelite/time_series_2021_P112/33_scl9_2021-05-24.png})
			% (see. \ref{fig:satelite/time_series_2021_P112/35_scl4_2021-06-03.png})
			% (see. \ref{fig:satelite/time_series_2021_P112/40_scl10_2021-06-28.png})
			% (see. \ref{fig:satelite/time_series_2021_P112/45_scl2_2021-07-23.png})
	}
}

\section{General Methods}{\label{sec:general_methods}
	Here we will only introduce Methods which will accure in several places. For interpolation methods we refer to sections \ref{sec:itpl_parametric} and \ref{sec:itpl_nonparametric}, for a robustification strategy to section \ref{sec:loess_robustify}. In section \ref{sec:itpl_param_est} we describe a method to objectively determine the quality of an interpolation, and in section \ref{sec:corr_correction} we present the NDVI correction together with an adapted interpolation strategy.


	\subsection{Root Mean Square Error (RMSE)}
		In this section we describe different criteria to evaluate models. Hence, given a vector $y\in \R^n$ and its estimator $\hat y$ (estimated using the model), we define the RMSE as:
		\begin{equation}
			\label{eq:rmse}
			 \sqrt{\frac{1}{n}\sum_{i=1}^n (y_i - \hat y_i)^2}
		\end{equation}
		% keine definition für R2 und relative RMSE da diese nur im appendix verwendet werden. Die definition nehmen wir dort vor.
		
		\subsection{Out-Of-Bag (\textit{OOB}) and Leave-One-Out-Cross-Validation (\textit{LOOCV})}{ \label{sec:OOB_LOOCV}
		The rationale for OOB and LOOCV is that we intend to evaluate a model $M$ with unseen data. That is, if $D$ describes the entire dataset and we train a model on a subset of $D$, we can use the remaining data to evaluate the model. 
		
		To formally introduce this, let:
		$$
			D=\{(X_{[j,:]},y_j)|\; X\in\R^{n\times p}, y\in \R^n, j=1,\dots,n\}
		$$
		be a dataset, $i\in \{1,\dots,n\}$ and $M^{(-i)}$ a model fitted on a subset of $D\setminus\{(X_{[i,:]},y_i)\}$. Then we call $\hat y_i:= M^{(-i)}(X_{[i,:]})$ an \textit{OOB} estimator of $y_i$. If we do this for all $i\in\{1,\dots,n\}$, we obtain $\hat y := \left(\hat y_1,\dots,\hat y_n\right)$ the OOB estimator for $y\in \R^n$.

		In the bootstrap (e.g., random forest) framework, we define $\hat y_i$ to be the average of all computed and admissible $M^{(-i)}$. 
		
		In the case that $M^{(-i)}$ was fitted on the set $D\setminus\{(X_i,y_i)\}$ (i.e., not a true subset), we call the corresponding $\hat y_i$ also the LOOCV estimator.	

		If we optimize some parameter via OOB (or LOOCV) this means that we search for the parameter that minimizes some loss function which takes the OOB (or LOOCV) residuals. Usually we approximate this parameter by searching on a grid. 
	}

}

% \begin{my_pros_cons_table}{
%         \item 1
%         \item 2
%     }{
%         \item 1
%         \item 2
%     }
% \end{my_pros_cons_table}
\newcommand{\RobItPlot}{fitted to different (SCL45) NDVI time series. Iterations of a robustifing refit (as indicated in section~\ref{sec:loess_robustify}) are also displayed}


\chapter{Interpolation Methods} \label{sec:itpl}\todo{verdeutliche dem leser, dass ein auftrag das findne von interpolationmethoden war}
	{% Roadmap
		In section \ref{sec:s2_challangges} we have established the need for interpolating the NDVI time series. In this chapter we first specify a setting for the interpolation and divide the interpolation methods into those that make fundamental shape assumptions (parametric) and those that are more flexible (non-parametric). We give an introduction for each method with an compact definition, highlight adjustments or give remarks where appropriate, and then point out strengths and weaknesses of each method. Additionally, a brief overview of the considered interpolation methods is provided in table~\ref{table:pros_cons_overview}.
		% In this section, we take a closer look at several interpolation methods, which will be used to interpolate and smooth the NDVI time series, while considering only SCL45 in this chapter. First, we define the general setting and discuss a general approach to make the interpolation more robust (i.e. reduce the impact of outliers). Afterwards, we introduce and discuss each method.
		Afterwards, we extract an robustification strategy from the one interpolation method and generalize it so we can use it for all methods that allow for a priori weighted observations. Finally, using LOOCV, we tune the parameters (where necessary) and get a first idea of the performance of each method.


	}
	{% pros & cons table
		\footnotesize
		\begin{table}[!ht]
	\centering
	\caption[Overview of the studied interpolation methods.]{Overview of the studied interpolation methods containing important assumptions, strengths, and weaknesses and whether the method supports weighted observations (w) and if the resulting interpolation is bounded w.r.t. a fixed interval (b).}
	\small
	\begin{tabular}{p{1.6cm}p{3.3cm}p{3.3cm}p{3.4cm}p{0.4cm}p{0.4cm}p{3cm}p{3cm}p{3cm}p{3cm}p{2.7cm}p{3cm}|}
		\toprule
		% \hline
		~                                                                                                                                                            &
		\textbf{Assumptions}                                                                                                                                         &
		\textbf{Strengths}                                                                                                                                                &
		\textbf{Weaknesses}                                                                                                                                                &
		\textbf{w}                                                                                                                                      &
		\textbf{b}                                                                                                                                        \\ \hline

		Double-Logistic                                                                                                                                              &
		\begin{cptitemize} \item[--]  Bell shape of curve \item[--]  NDVI has a minimal value                       \end{cptitemize}        &
		\begin{cptitemize} \item[--]  Good for evergreen plants if snow masks NDVI \item[--] Handles data gaps well                            \end{cptitemize}        &
		\begin{cptitemize} \item[--]  Parameter estimation can be challenging, to solve this the parameter space can be bounded             \end{cptitemize}        &
		Yes                                                                                                                                                          &
		(Yes)                                                                                                                                                         \\ \hline%comment out?

		Fourier Series                                                                                                                                              &
		\begin{cptitemize} \item[--]  NDVI can be approximated by a \nth{2} order Fourier series.      \item[--] Prominent approach                      \end{cptitemize}        &
		\begin{cptitemize} \item[--]  Incorporates periodical growth-cycles                                 \end{cptitemize}        &
		\begin{cptitemize}  \item[--]  Curve easily exceeds the bounds of the NDVI within data gaps    \item[--]  Parameter estimation can be challenging, to solve this the parameter space can be bounded        \end{cptitemize}        &
		Yes                                                                                                                                                          &
		No                                                                                                                                                         \\ \hline%comment out?

		Nadaraya-Watson                                                                                                                            &
		\begin{cptitemize} \item[--]  Close points are related to each other via a kernel function \end{cptitemize}                                                                                                                                                            &
		\begin{cptitemize} \item[--]  Simple  \item[--]  Computationally very fast                                                             \end{cptitemize}        &
		\begin{cptitemize} \item[--]  Biased, especially at `peaks' and `valleys'   \item[--]  Bandwidth: fails if there are big data-gaps                                                     \end{cptitemize}               &
		Yes                                                                                                                                                          &
		Yes                                                                                                                                                            \\ \hline%comment out?

		Universal Kriging                                                                                                                                            &
		\begin{cptitemize} \item[--]  Function is a realization of a stationary Gaussian process                                      \end{cptitemize}               &
		\begin{cptitemize} \item[--]  Informative parameters \item[--]  Flexible                                                             \end{cptitemize}        &
		\begin{cptitemize} \item[--]  Assumption not met for NDVI TS \item[--] Regression to the mean, especially within data gaps          \end{cptitemize}        &
		Yes                                                                                                                                                          &
		(Yes)                \\ \hline%comment out?                                                                                                                                         \\ %\hline%comment out?

		Savitzky-Golay                                                                                                                                         &
		\begin{cptitemize}
			\item[--]  High frequencies are noise (Low-Pass-Filter) \item[--]  Equidistant points \item[--]  Local polynomials\end{cptitemize}                                              &
		\begin{cptitemize} \item[--]  Computationally very fast                                                                   \end{cptitemize}                   &
		\begin{cptitemize} \item[--]  Cannot deal natively with non-equidistant data                             \end{cptitemize}                 &
		No                                                                                                                                                           &
		(Yes)                                                                                                                                                         \\ \hline%comment out?

		Savitzky-Golay  +~NDVI                                                                                                                                                    &
		\begin{cptitemize} \item[--]  Upper envelope \item[--]  Vegetation cannot grow faster than some slope                                \end{cptitemize}        &
		\begin{cptitemize} \item[--]  Biological knowledge                                                                            \end{cptitemize}               &
		\begin{cptitemize} \item[--]  Bad `upper envelope' since weights are not used for the estimation itself                    \end{cptitemize}               &
		(No)                                                                                                                                                         &
		(Yes)                                                                                                                                                         \\ \hline%comment out?

		LOESS                                                                                                                                                        &
		\begin{cptitemize} \item[--]  Local  polynomial with points closer to the estimated point are more important                  \end{cptitemize}               &
		\begin{cptitemize} \item[--]  Flexible \item[--]  Generalization of the Savitzky-Golay \item[--] Intuitive weighting function                   \end{cptitemize} &
		\begin{cptitemize} \item[--]  Computationally expensive                                                                       \end{cptitemize}               &
		Yes                                                                                                                                                          &
		(Yes)                                                                                                                                                         \\ \hline%comment out?

		B-Splines                                                                                                                                          &
		\begin{cptitemize} \item[--]  Function can be approximated by a linear combination of B-splines basis functions               \end{cptitemize}               &
		\begin{cptitemize} \item[--]  General assumption \item[--]  Flexible shape                                                            \end{cptitemize}        &
		\begin{cptitemize} \item[--]  Unbounded \item[--]  Non-intuitive smoothing process                                                \end{cptitemize}        &
		Yes                                                                                                                                                            &
		No                                            \\ \hline%comment out?

		Smoothing Splines                                                                                                                                            &
		\begin{cptitemize} \item[--]  \nth{2} derivative of function is integrable                                                        \end{cptitemize}               &
		\begin{cptitemize} \item[--]  Intuitive meaning of penalty \item[--]  General assumptions \item[--]  Flexible shape                         \end{cptitemize} &
		\begin{cptitemize} \item[--]  Choice of smoothing parameter                                                                                       \end{cptitemize}               &
		Yes                                                                                                                                                          &
        (Yes) \\                                                                         
% Penalized Regression Splines &
		% \begin{itemize}
		%     \item[--]  High
		%     \item[--]  Bs
		% \end{itemize} &
		% ~ &
		% ~ &
		% ~ &
		% ~ \\ \hline%comment out?

		% Whittaker                                                                                                                                                    &
		% % \parskip=0pt
		% % \begin{minipage}[t]{\linewidth}
		% % 	\begin{itemize}[nosep,after=\strut]
		% % 		\item[--] First
		% % 		\item[--] Second
		% % 	\end{itemize}
		% % \end{minipage}                                                               &
		% ~                                                                                                                                                            &
		% ~                                                                                                                                                            &
		% ~                                                                                                                                                            &
		% ~                                                                                                                                                              \\ \hline%comment out?

		% Fourier                                                                                 &
		% ~                                                                                       &
		% ~                                                                                       &
		% ~                                                                                       &
		% ~                                                                                       &
		% ~                                                                                             \\ \hline%comment out?

		% methodname &
		% ~ &
		% ~ &
		% ~ &
		% ~ &
		% ~ \\ \hline%comment out?
		\bottomrule
	\end{tabular}
	\label{table:pros_cons_overview}
\end{table}

		\normalsize
	}



\section{Interpolation Setup}{\label{sec:itpl_setup}
	In this chapter, we will only consider SCL45 observations, since they are more reliably. Hence, data in the form of $\left(t_{i}, y_{i}\right)$ for $i=1, \ldots, n$ is given, where $t_i$ is the time in GDD and $y_i$ denotes the NDVI at time $t_i$. Assume that it can be represented by
	$$
		y_{i}=m\left(t_{i}\right)+\varepsilon_{i},
	$$
	where $\varepsilon_i$ is some noise and $m: \R \rightarrow \R$ is some (parametric or non-parametric) function. If we assume that $\varepsilon_{1}, \ldots, \varepsilon_{n}$ i.i.d. with $\mathbb{E}\left[\varepsilon_{i}\right]=0$ then 
	$$
		m(t)=\mathbb{E}[y \mid t]
	$$
	We will introduce parametric and non-parametric approaches to estimate $m$ in section \ref{sec:itpl_parametric} and \ref{sec:itpl_nonparametric}
	Furthermore, in the subsequent, we denote $w\in \R^n$ as the vector of weights such that $w_i$ corresponds to the weight that $(t_i, y_i)$ should have in the interpolation. 
}


\todo[inline]{gewichte einfügen}

\todo[inline]{Paper zitieren wo eingeführt oder wo benutzt (falls einführung fast schon trivial)}

\section{Parametric Regression} 
	\label{sec:itpl_parametric}
	Parametric Curve estimation tries to fit a parametric function, such as, for example, a Gaussian function with parameters $\mu$ and $\sigma$, to a dataset. In the following, we introduce two parametric approaches.

	\subsection{Double Logistic (DL)}
		\label{sec:double_logistic}
		The Double Logistic smoothing as described in \cite{beckImprovedMonitoringVegetation2006} heavily relies on shape assumptions of the fitted curve (i.e., the NDVI {TS}). First, we assume that there  is a minimum NDVI level $y_{\min}$ in the winter (e.g., due to evergreen plants), which might be masked by snow. This can be estimated beforehand, taking several years into account. Second, we assume that the growth cycle can be divided into an increase and a decrease period, where the {TS} follows a logistic function. The maximum increase (or decrease) is observed at $t_0$ (or $t_1$) with a slope of $d_0$ (or $d_1$). The equation of the double-logistic fit is given by:
		\begin{equation*}
			y(t) = y_{\min} + \left(y_{\max}-y_{\min}\right)\left(\frac{1}{1+e^{-d_0(t-t_0)}}+\frac{1}{1+e^{-d_1(t-t_1)}}-1\right)
		\end{equation*}
		Where the five free parameters: $y_{\max}$, $d_0$, $d_1$, $t_0$, $t_1$ are initially estimated by least squares. Such fit can be seen in figure~\ref{fig:interpol/fourier_dl_comparison}.

		\subsubsection*{Robustification}
		Similar as for the SG (cf. section~\ref{sec:Savitzky-Golay}) one can reestimate (only once) the parameters by giving less weight to the overestimated observations and more weight to the underestimated observations. For the details on the choice of the weights we refer to \cite{beckImprovedMonitoringVegetation2006}. We will not apply this reestimation but rather the robustification introduced later in section \ref{sec:loess_robustify}.

		\begin{my_pros_cons_table}{
				\item Incorporates subject specific knowledge in the case of evergreen plants covered in snow.
				\item Optimized parameters have an intuitive meaning.
			}{
				\item Strong shape assumptions on the NDVI curve.
				\item Parameter optimization might go wrong. This can be mitigated to some extent to provide bounds for the parameters
				\item Strange behavior in regions with little observations. (cf. figure~\ref{fig:interpol/fourier_dl_comparison})
			}
		\end{my_pros_cons_table}


	\subsection{Fourier Series (FS)}
		\label{sec:fourier_approx}
		\cite{stockliEuropeanPlantPhenology2004} approximates the NDVI curve using a second order FS:
		$$
			\operatorname{NDVI}(t)=\sum_{j=0}^{2} a_{j} \times \cos \left(j \times \Phi_{t}\right)+b_{j} \times \sin \left(j \times \Phi_{t}  \right)
		$$
		where $\Phi=2 \pi \times(t-1) / n$. Thus, we periodical behavior. If we would set the period to match one year this would coinced with the notion that plans grow every year. 
		Analogous to section~\ref{sec:double_logistic} we fit it to the data by least squares.
		Example fits can be seen in figure \ref{fig:interpol/fourier_dl_comparison}


		% \cite{beckImprovedMonitoringVegetation2006} shows in their lag-plots a heavy autocorrelation of resiudals

		\begin{my_pros_cons_table}{
				\item Assumption of periodicity can be helpful if we are modelling multiyear grow cycles
				\item Flexible curve shape
			}{
				\item Bad behavior in regions with little data (cf. figure~\ref{fig:interpol/fourier_dl_comparison})
				\item Hard to interpret estimated parameters
				\item Parameter estimation can go wrong. Introducing bounds can help.
			}
		\end{my_pros_cons_table}

		\begin{my_figure}[h]{width=1\textwidth}{interpol/fourier_dl_comparison}
			\caption{Here we observe the possibilities of a precise fit for the two parametric methods but notice also some misbehavior}
			\label{fig:interpol/fourier_dl_comparison}
		\end{my_figure}

	\subsection{Optimization Issues}\label{sec:itpl_param_optimizationissues}
		We shall mention some optimization issues we countered during implementation. Since we aim to minimize the residual sum of squares over 5 (or 6) parameters, we try to solve a non-convex optimization problem. Thus, the algorithm\footnote{We used the python function \texttt{scipy.optimize.curve\_fit}.} either struggles to find the global minimum or fails to converge. This was fixed by providing for each parameter reasonable initial values and generous bounds (that match our experience).

\section{Non-Parametric Regression}
	\label{sec:itpl_nonparametric}
	In non-parametric curve estimation, the curve does no longer have to be fully determined by parameters, but we allow it to flexibly approximate the data. Note that we do not exclude the use of tuning-parameters.

	\subsection{Kernel Regression: Nadaraya-Watson (NW)}
		\label{sec:Kernel}
		As described in section \ref{sec:itpl_setup}, we aim to estimate
		\begin{equation}
			\label{eq:nadaraya}
			\mathbb{E}[Y \mid T=t]
			= \int_{\R} y f_{Y \mid T}(y \mid t) d y
			=\frac{\int_{\R} y f_{T, Y}(t, y) d y}{f_{T}(t)},
		\end{equation}
		where $f_{Y \mid T}, f_{T, Y}, f_{T}$ denote the conditional, joint and marginal densities.
		This can be done with a kernel $K$:
		\begin{equation}
			\hat{f}_{T}(t)=\frac{\sum_{i=1}^{n} K\left(\frac{t-t_{i}}{h}\right)}{n h}, \quad \hat{f}_{T, Y}(t, y)=\frac{\sum_{i=1}^{n} K\left(\frac{t-t_{i}}{h}\right) K\left(\frac{y-Y_{i}}{h}\right)}{n h^{2}},
			\label{eq:kernel_with_bandwidt}
		\end{equation}
		where $h$, the bandwidth, symbolizes the windowsize of to consider. By using the above function in equation \eqref{eq:nadaraya} we arrive at the NW kernel estimator:
		$$\hat{m}(t)=\frac{\sum_{i=1}^{n} K\left(\left(t-t_{i}\right) / h\right) Y_{i}}{\sum_{i=1}^{n} K\left(\left(t-t_{i}\right) / h\right)}$$

		Common choices for the kernel are the normal function or a uniform function (also called `bot' function). 
		\subsubsection*{Choose Bandwidth}
		Note that we still need to choose the bandwidth of the function. This can be done with the help of LOOCV while optimizing the RMSE. For non-equidistant data we refere to \cite{brockmannLocallyAdaptiveBandwidth1993} where a local adaptive bandwidth selection is presented.

		\begin{my_pros_cons_table}{
				\item fletible due to different possible kernels
				\item can be assigned degrees of freedom (trace of the hat-matrit)
				\item estimation of the noise variance $\hat \sigma_\varepsilon^2$ (REF cf. CompStat 3.2.2)
			}{
				\item if the $t \mapsto K(t)$ is not continuous, $\hat m $ isn't either
				\item choice of bandwidth, especially if $t_i$ are not equidistant.
			}
		\end{my_pros_cons_table}


	\subsection{Universal Kriging (UK)}
		\label{sec:Kriging}

		UK as described in \cite{diggleGaussianModelsGeostatistical2007} was developed in geostatistics to deal with autocorrelation of the response variable at locations that are spatially close. By applying the notion that two spectral indices that are timewise close should also take similar values, we justify the application of UK. In the end, we would like to fit a smooth Gaussian process to the data.

		% definition (gaussian process)
		A Gaussian Process $\{S(t) : t\in \mathbb R\} $ is a stochastic process if $(S(t_1),\dots,S(t_k))$ has a multivariate Gaussian distribution for every collection of times ${t_1, \dots , t_k}$. $S$ can be fully characterized by the mean $\mu(t):=E[S(t)]$ and its covariance function $\gamma\left(t, t^{\prime}\right):=\operatorname{Cov}\left(S(t), S\left(t^{\prime}\right)\right)$. 
		% stationarity assumtpion
		Furthermore, we will assume the Gaussian process to be stationary. That is for $\mu(t)$ to be constant in $t$ and $\gamma(t,t')$ to depend only on $h=t-t'$. Thus, we will write in the following only $\gamma(h)$.\footnote{Note that the process is also \textit{isotropic} (i.e., $\gamma(h)=\gamma(\|h\|$) since we are in a one-dimensional setting and the covariance is symmetric.}

		% variogram
		Now, we need to make some assumption on the covariance function. For this we introduce the variogram of a Gaussian process as
		$$V(h):=V\left(t, t+h\right):=\frac{1}{2} \operatorname{Var}\left(S(t)-S(t+h)\right)\\ %align XXX
			=\gamma(0) + \gamma(t)
		$$
		and define $\gamma$ via the above equation by choosing the Gaussian Variogram defined by
		$$V(h) = p \cdot\left(1-e^{-\frac{h^{2}}{\left(\frac{4}{7} r\right)^{2}}}\right)+n.$$
		Here $h$ denotes the distance, $n$ is the nugget, $r$ is the range and $p$ is the partial sill. The influence of the parameters is visualized in figure~\ref{fig:interpol/kriging_gauss_variogram}.\footnote{Strictly speaking we use a scaled version of the variogram. Thus, only the ratio of $p/n$ matters.}

			\begin{my_figure}[h]{width=0.7\textwidth}{interpol/kriging_gauss_variogram}
				\caption{Gaussian Variogram with nugget=1, partial sill=3, range=55}
				\label{fig:interpol/kriging_gauss_variogram}
			\end{my_figure}

			\begin{my_figure}{width=1\textwidth}{interpol/kriging_parameter}
				\caption{On the left, we see how the interpolation change if we increase the nugget and the range parameter. On the right, we compare two UK interpolations, where one takes parameters by numerically maximizing the (which results in a very small nugget) and the other takes the median of many such numerical optimizations.}
				\label{fig:kriging_parameters}
			\end{my_figure}

		Finally, we consider a one-dimensional Gaussian process $G_\gamma$ with variogram $\gamma$ and tune the variogram parameters using matimum likelihood\footnote{As illustrated in figure~\ref{fig:kriging_parameters} matimum likelihood estimation can lead to overfitting. Thus, we will in practice sample several such optimized parameters and use their median in the end.}. Let $z$ be a vector with the new values to ettrapolate, then we can determine the values $m(z) = \mathbb{E}\left[G_\gamma(z) | (t,y)\right]$ using Bayes rule\footnote{Bayes rule generally claims that for two random variables $A$ and $B$ we have that $P(A|B) = P(B|A) / P(B)$}. For an etample fit, we refer to figure~\ref{fig:kriging_parameters}. 

		\subsubsection*{Violated Assumption}
			Since we observe a clear pattern of a growth period in spring and harvest in the end of summer, we have to admit that our stationarity assumtpion with the constant mean is structurally violated. This is also the reason why we observe (for every variogram parameter) a tendency to the mean, as indicated in figure~\ref{fig:kriging_parameters}.

		\begin{my_pros_cons_table}{
				\item It is a well-studied method.
				\item Variogram parameters have an intuitive meaning.
				\item Flexible covariance structure.
			}{
				\item Regression to the mean.
				\item Violated assumption of constant mean and constant variance. Thus, the NDVI is not a stationary process.
				\item Pure maximum likelihood can result in overfitting.
			}
		\end{my_pros_cons_table}


	\subsection{Savitzky-Golay Filter (SG)}
		\label{sec:Savitzky-Golay}
		The SG, introduced in \cite{savitzkySmoothingDifferentiationData1964} is a technique in signal processing and can be used to filter out high frequencies (low-pass filter) \citep{schaferWhatSavitzkyGolayFilter2011}. Furthermore, it can also be used for smoothing by filtering high frequency noise while keeping the low frequency signal.

		First, we choose a window size $m$. Then, for each point, $j \in \{m, m+1, \dots, n-m\}$ we fit a polynomial of degree $k$ by:
		$$\hat y_j=\min_{p\in P_k}\sum_{i=-m}^{m}(p (t_{j+i})-y_{i+j})^{2},$$
		where $P_k$ denotes the Polynomials of degree $k$ over $\R$.
		For equidistant points this can efficiently be calculated by
		$$
			\hat y_{j}=\sum_{i=-m}^{m} c_{i} y_{j+i},
		$$
		where the $c_i$ are only dependent on the $m$ and $k$ and are tabulated in the original paper.

		\cite{chenSimpleMethodReconstructing2004a} developed a `robust' {{IM}} for the NDVI based on the SG. 
		The method is based on the assumption that due to atmospheric effects the observed NDVI tends to be underestimated and that it cannot increase too quickly. The latter is argued by the biological impossibility of such fast vegetation changes. Their proposed algorithm is:\todo{figure / tabelle / pseudocode anstatt aufzählung}
			\begin{enumerate}
				\item Remove non-SCL45 points.
				\item Remove points that would indicate an increase greater than 0.4 within 20 days.
				\item Linearly interpolate to obtain an equidistant {TS} $X^0$.
				\item Apply the SG to obtain a new {TS} $X^1$.
				\item Update $X^1$ by applying again a SG. Repeat this until $w^T |X^1-X^0|$ stops decreasing, where w is a weight vector with $w_i = \min\left(1, 1 - \frac{X^1_i-X^0_i}{\max_i\|X^1_i-X^0_i\|}\right)$. This reduces the penalty introduced by outliers\footnote{Here we call a point $i$ an outlier if $X^0_i<X^1_i$.} and by repeating this step we approach the ``upper NDVI envelope''.
			\end{enumerate}

		\subsubsection*{Extension: Spatial-Temporal SG}
			One notable adaptation of the SG is the presented by \cite{caoSimpleMethodImprove2018b}. The key difference is the additional assumption of the cloud cover being discontinuous and that we can improve by looking at adjacent pixels\footnote{Here, we say that a pixel is adjacent if it is the same pixel but from a different year (keeping the same day of the year) or (if not enough of such temporal-adjacent pixel are found) it is spatially adjacent}. Because we are working with rather high resolution satellite data, and we need the variance in the predictors, we will waive this extension.

		\begin{my_pros_cons_table}{
				\item Popular technique in signal processing.
				\item Efficient calculation for equidistant points.
				\item Upper envelope matches intuition for the NDVI. Therefore, it is robust against outliers with small values.
			}{
				\item No natural way of how to estimate points that are not in the data.
				\item Not generalizable to other spectral indices.
				\item Linear interpolation to account for missing data might be not appropriate.
				\item No smooth interpolation between two measurements.
			}
		\end{my_pros_cons_table}


	\subsection{Locally Weighted Regression (LOESS)}
		\label{sec:loess}
		% Introduced by : \cite{clevelandRobustLocallyWeighted1979}
		% implemented here \cite{cappellariATLAS3DProjectXX2013}

		The LOESS introduced by \cite{clevelandRobustLocallyWeighted1979} can be understood as a generalization of the SG (cf. sec.~\ref{sec:Savitzky-Golay}).

		Given a proportion $\alpha \in (0,1]$, we estimate each $y_i$ separately by fitting a polynomial of order $d$ by weighted least squares. The weights are (usually) defined by
		$$w_i(t_j)=\begin{cases}
				\left(1-\left(\frac{|t_j-t_i|}{h_i}\right)^{3}\right)^{3}, & \text{for } |t_j-t_i|<h_i           \\
				0,                                                   & \text{for } |t_j-t_i| \geqslant h_i
			\end{cases} ,$$
		where $h_i$ is the minimal distance such that $\lceil \alpha n\rceil$ observations are in the ball $B_{h_i}(t_i)$.\footnote{\label{footnote:LOESS}If too many weights are set to zero, we might end up considering not enough observations and thus get a singular design-matrit (for the least squares estimation). Therefore, we substitute $h_i$ with $1.01 h_i$, so that the observation on the boundary of $B_{h_i}(t_i)$ does not get completely ignored. But we also have to assure that $\alpha$ is big enough.} So for each $y_i$ we only consider a proportion $\alpha$ of the observations.

		\subsubsection{Differences between the Robust LOESS and the SG}
		The LOESS smoother takes a fraction of points instead of a fixed number and therefore automatically adapts to the size of the data we wish to interpolate. However, we run into the danger of considering too little observations, since the estimation breaks down if $\lceil \alpha n\rceil < d+1$.\footnoteref{footnote:LOESS}
		Furthermore, LOESS gives less weight to points further away. This yields a "smoother" estimate, since when we slide the window (e.g., for estimating the next value) an influential point at the border does not suddenly get zero weight from being weighted equally before.
		Finally, the LOESS also can be used for non-equidistant data and allows for arbitrary interpolation.

		\begin{my_pros_cons_table}{
				\item Flexible generalization of SG
				\item arbitrary interpolation possible
				\item Intuitive parameters
			}{
				\item The nature of local regression might lead to surprising estimates (no smoothness guarantees for the second derivative)
			}
		\end{my_pros_cons_table}


	\subsection{B-Splines (BS)}
		\label{sec:B}
		BS as discussed in \cite{lycheSplineMethods2005} are piecewise cubic polynomials defined by 
		$$
			S(t)=\sum_{j=0}^{n-1} c_{j} B_{j, k ; t}(t),
		$$
		where $B$ are basis functions and recursively defined by:
		
		\begin{equation}
				B_{i, 0}(z)=1, \text { if } t_{i} \leq z<t_{i+1}, \text { otherwise } 0 \\
		\end{equation}
		\begin{equation}
			B_{i, k}(z)=\frac{z-t_{i}}{t_{i+k}-t_{i}} B_{i, k-1}(z)+\frac{t_{i+k+1}-z}{t_{i+k+1}-t_{i+1}} B_{i+1, k-1}(z).
		\end{equation}
		
		Assuming that all $t_i$ are distinct, this yields an interpolation that fits the data perfectly. To reduce the amount of overfitting and increase the smoothness, we relax the constraint that we have to perfectly interpolate. Thus, we use the minimum number of basis functions\footnote{So we do not require one basis function for each neighboring pair of knots. SciPy uses FITPACK and DFITPACK, the documentation suggests that smoothness is achieved by reducing the number of knots used} such that:
		$$\sum_{i=1}^n(w_i (y_i - \hat y_i))^2 \leq s$$

		\begin{my_pros_cons_table}{
				\item can be assigned degrees of freedom
				\item extendable to "smooth" version
				\item performs also well if points are not equidistant
			}{
				\item smoothing process does not translate well to a interpretation (unlike SS)
				\item choice of smoothing parameter $s$
			}
		\end{my_pros_cons_table}


	\subsection{Smoothing Splines (SS)}
		\label{sec:Natural_SS}
		Let $\mathcal F$ be the Sobolev space (the space of functions of which the second derivative is integrable). Then the unique\footnote{Strictly speaking it is only unique for $\lambda > 0$} minimizer
		\begin{equation}
			\label{eq:ss}
			\hat m :=\argmin_{f \in \mathcal F} \sum_{i=1}^{n}w_i\left(y_{i}-{f}\left(t_{i}\right)\right)^{2}+\lambda \int {f}^{\prime \prime}(t)^{2} dt
		\end{equation}
			
		is a % natural\footnote{It is called natural since it is affine outside the data range ($\forall t\notin [t_1, t_n]:\hat m''(t) = 0$)} 
		cubic spline (i.e., a piecewise cubic polynomial function).
		The objective function ensures that we decrease the curvature while keeping the RMSE low.

		\subsubsection{XXX Whittaker}

		\begin{my_pros_cons_table}{
				\item Can be assigned degrees of freedom (trace of the hat-matrix).
				\item Efficient estimation (closed form solution).
				\item Intuitive penalty (we don't want the function to be too ``wobbly'' --- change slopes).
				\item Also performs well if points are not equidistant.
				\item Fixes the Runge's phenomenon (fluctuation of high degree polynomial interpolation).
			}{
				\item The tuning parameter $\lambda$ must be chosen. This can be done via cross validation and optimizing a score function (e.g., the RMSE). 
			}
		\end{my_pros_cons_table}


	% \subsection{XXX Whittaker Smoother}

	% \label{sec:whittaker}
	% XXX
	% from [HERE](https://eigenvector.com/wp-content/uploads/2020/01/WhittakerSmoother.pdf):  
	%     The Whittaker Smoother: Eiler's paper[1] introduces the following objective function
	%     $$
	%     O(\mathbf{z})=(\mathbf{y}-\mathbf{z})^{\mathrm{T}} \mathbf{W}_{0}(\mathbf{y}-\mathbf{z})+\lambda_{\mathrm{s}} \mathbf{z}^{\mathrm{T}} \mathbf{D}_{\mathrm{s}}^{\mathrm{T}} \mathbf{D}_{\mathrm{s}} \mathbf{z}
	%     $$
	%     where $\mathbf{y}$ is a $N \times 1$ vector of measured data, $\mathbf{z}$ is smooth curve to be fit to the data, $\mathbf{W}_{0}$ is a diagonal matrix of weights (typically $0 \leq w_{0, n} \leq 1$ for $n=1, \ldots, N, \mathbf{D}_{\mathrm{s}}$ is a second derivative operator (e.g., $\mathbf{D}_{\mathrm{s}} \mathbf{z}$ is the second derivative of $\mathbf{z}$ ) and $\lambda_{\mathrm{s}}$ is a scalar penalty on the smoothing term. When data are missing, the corresponding weight, $w_{0, n}$, can be set to zero. Once that $\mathbf{W}_{0}$ and $\lambda_{\mathrm{s}}$ are given (set by default or provided by the user) the corresponding estimate of $\mathbf{z}$ is given by
	%     $$
	%     \hat{\mathbf{z}}=\left(\mathbf{W}_{0}+\lambda_{\mathrm{s}} \mathbf{D}_{\mathrm{s}}^{\mathrm{T}} \mathbf{D}_{\mathrm{s}}\right)^{-1} \mathbf{W}_{0} \mathbf{y}
	%     $$
	%     For example, an optical emission (OES) spectrum is plotted Figure 1 along with two smoothed versions shown for $\mathbf{W}_{0}=\mathbf{I}$ and $\lambda_{\mathrm{s}}=0.1$ (low smoothing) and $\lambda_{\mathrm{s}}=10$ (stronger smoothing).   
	% **Original paper states use of the first derivative**  
	% --> second derivative is very similar to SS
	% \begin{my_pros_cons_table}{
	%     \item 1
	%     \item 2
	%   }{
	%     \item 1
	%     \item 2
	%   }
	% \end{my_pros_cons_table}



% \subsection{Other Methods to study:}
% From introduction of \cite{chenSimpleMethodReconstructing2004a}:\\
% (1) threshold-
% based methods, such as the best index slope extraction
% algorithm (BISE) (Viovy et al., 1992); (2) Fourier-based
% fitting methods (Cihlar, 1996; Roerink et al., 2000; Sellers
% et al., 1994); and (3) asymmetric function fitting methods
% such as the asymmetric Gaussian function fitting approach
% (Jonsson Eklundh, 2002) and the weighted least-squares
% linear regression approach (Swets et al., 1999).

\section{Tuning Parameter Estimation}{ \label{sec:itpl_param_est}
	Many of the interpolation methods introduced in section \ref{sec:itpl_parametric} and \ref{sec:itpl_nonparametric} include a free parameter. To determine this parameter for a specific interpolation method, we will estimate the absolute residuals using OOB estimation and then optimize the parameter using a score function. We clarify the procedure step by step:	
	\begin{Nenumerate}
		\item Construct a set $\Lambda$ of candidate parameters that generously covers the parameter space.
		\item Consider $\mathcal{P}$, a set of Pixels.
		\item For each parameter $\lambda \in \Lambda$ consider the individual pixels and compute the LOOCV\footnote{For a definition of the leave-one-out-cross-validation we refer to section~\ref{sec:OOB_LOOCV}} for the absolute residuals of the specific NDVI interpolation method for all Pixels in $\mathcal{P}$ and store them in the set $R_\lambda$. 
		\item Determine $\lambda_{optimal} = \argmin_{\lambda\in\lambda}q_{90}(R_\lambda)$, where we describe the 90\% quantile with $q_{90}$.
	\end{Nenumerate}

	\begin{my_figure}[h]{width=1\textwidth}{interpol/statistics_SS_param_optim}
		\caption{Smoothing splines fit with smoothing parameter optimized by minimizing the given quantile of the absolute leave-one-out residuals. Note that the larger the considered quantile is, the smoother the resulting curve becomes.}
		\label{fig:interpol/statistics_SS_param_optim}
	\end{my_figure}

	We choose $\operatorname{quantile}(90)$ as our optimization function because we want to allow 10\% of outliers (corrupt points) but also aim for an accurate fit in 90\% of the cases.  
	
	Figure \ref{fig:interpol/statistics_SS_param_optim} exemplifies the effect of the optimization function (different quantiles). To summarize, we may say that the higher the quantile, the stronger the smoothing. 
}


\section{Robustification}{
	\label{sec:loess_robustify}
	{ % Intro
		Now we discuss a general approach of how to make an interpolation more robust against outliers. The main idea is to give less weight to observations that have high residuals after the initial (or if we reiterate, the previous) fit.

		Even though the procedure is taken from the robust version of the LOESS smoother (c.f. section~\ref{sec:loess} and \cite{clevelandRobustLocallyWeighted1979}), we can apply it to every interpolation method that allows for prior weighting of observations.
	}
	
	{	
		After an initial fit we calculate the residuals $r_i := y_i - \hat y_i$ and obtain $\tilde r_i$ by scaling with the median of the absolute residuals: 
		\begin{equation}
			\tilde r_i := \frac{r_i}{6\operatorname{med}\left(|r_1|,\dots,|r_n|\right)}
		\end{equation}
		Next, we compute new weights by
		\begin{equation}
			w_i^\text{new}:=w_i^\text{old} \begin{cases}
				\left(1-\tilde r_i^{2}\right)^{2}, & \text{if } |\tilde r_i|<1 \\
				0,                        & \text{else }
			\end{cases};\quad
			\label{eq:bisquare}
		\end{equation}
		Using the new weights, we can re-interpolate. This reweighting can be iterated for several steps or till the change of the values is smaller than some tolerance.

	% Old more complicated version
		% Before we describe the procedure, we define a function that will determine the weight given to each observation, such that observations with large-scaled residuals will have less weight. That is the bisquare function B:
		% 	$$
		% 		B(x):=\begin{cases}
		% 			\left(1-x^{2}\right)^{2}, & \text{if } |x|<1 \\
		% 			0,                        & \text{else }
		% 		\end{cases}
		% 	$$ 
		% 	Now, we do something similar to what is done in iteratively reweighted least squares. After an initial interpolation, update the weights of each observation with
		% 	\begin{equation}
		% 		w_i^\text{new}:=w_i^\text{old} \operatorname{B}\left(     \right);\quad
		% 		r_i := y_i - \hat y_i
		% 		\label{eq:bisquare}
		% 	\end{equation}
		% 	and interpolate again using the new weights. We can iterate this reweighting and stop after several steps or when the change of the values is smaller than some tolerance.
	}

	Note that this procedure is indeed robust since we use the median for the normalization which has a breakdown point\footnote{Intuitively, the breakdown point denotes the fraction of observations a ``vicious'' player can replace without breaking the estimator. For example, the median has a breakdown point of $50 \%$.} of $50 \%$.\footnote{The breakdown point relates only to outliers in the $y$ values. Note that we do not require the interpolation methods to be robust, since the residual for an outlier will  still be larger than for non-outliers and thus will be down weighted more and more in each iteration (because for the next iteration the residual of the outlier will be even larger, since we gave less weight to it).}
	\subsection{Our Adjustment:}{
		During the iterations or when supplying prior weights, low-weighted observations can corrupt our estimation of scale (the median of absolute residuals). Thus, we introduce the weighted median as
		$$
			\med_\text{weighted}(r,w) := \argmin_{\lambda \in \R} \sum_{i=1}^n |r_iw_i -\lambda|
		$$
		for $r,w\in \R^n$. 
	}
	\subsection{Examples and Conclusions}{
		\begin{my_figure}[h]{width=1\textwidth}{interpol/2x3_SS_robust}
			\caption{Smoothing Splines \RobItPlot}
			\label{fig:interpol/2x3_SS_robust}
		\end{my_figure}

		Examples of the first four iterative fits using smoothing splines are shown in figure \ref{fig:interpol/2x3_SS_robust} for six pixels. For the analogous figures of the other interpolation methods c.f. figures \ref{fig:interpol/2x3_loess_robust}, \ref{fig:interpol/2x3_B-Splines_robust}, \ref{fig:interpol/2x3_DL_robust} and \ref{fig:interpol/2x3_loess_robust}.
		Indeed, we observe how the interpolated time series is less affected by outliers after each iteration. We notice the biggest difference in the first iteration. Furthermore, in the plot at the bottom left\todo{ consider naming the sub-plots} we see how the interpolation `escapes' from the right endpoint with each successive iteration, even though our intuition does not necessarily identify this point as an outlier. Therefore, in the following, we will always stop after one iteration.
	} 
	
	\subsection{Upper Envelope Approach - Penalty for Negative Residuals}
		If we artificially increase the negative residuals in \refeq{eq:bisquare} by multiplying (e.g. factor 2), the corresponding points will get less weight in the next iteration. This allows us to create an interpolation that resembles an upper envelope. Intuitively, this upper envelope can be thought of as a sheet that is laid on top of the points.
			
		This approach is based on the premise that we tend to underestimate the NDVI (as argued in \cite{caoSimpleMethodImprove2018b}). Since we want to develop a general method that is in principle not related to the NDVI, we will not pursue this approach further.	
}
\section{Performance Assessment}{
	Next, we will benchmark the different interpolation methods with and without robustification. For this, we will use the same technique as we did for the parameter determination in section \ref{sec:itpl_param_est}. On $B_\lambda$ we apply the RMSE and different quantiles.  

	The results are presented in section \ref{sec:results_itpl} and are discussed in section \ref{sec:discussion_itpl}. The double logistic turns out to be the best convincing parametric method and from the non-parametric methods we choose the smoothing splines.
}



\chapter{NDVI Correction}



\begin{my_figure}[ht]{width=0.75\textwidth}{ndvi_corr/residuals_scl_classes}
    \caption{XXX caption XXX}
    \label{fig:ndvi_corr/residuals_scl_classes}
\end{my_figure}


\begin{my_figure}[h]{width=1\textwidth}{ndvi_corr/scl_residuals_scatter}
    \caption{XXX caption XXX}
    \label{fig:ndvi_corr/scl_residuals_scatter}
\end{my_figure}

\chapter{Results}\label{sec:results}

\section{Goodness of Fit for Selected {{IM}}s}{
	\label{sec:results_itpl}
	The benchmarks of the in section~\ref{sec:itpl_preselection} selected {{IM}}s (considering only SCL45 observations) with respect to various score functions are displayed in table~\ref{tab:cv-statistics_itpl-methods}. The score functions summarize the absolute values of the LOOCV residuals (the smaller, the better). For each of the 5 selected {{IM}}s, we consider the basic and the robustified (see section~\ref{sec:loess_robustify}) version.

	\begin{table}[h]
		\begin{center}
			\caption[Goodness of fit for {{IM}}s  measured with score functions.]{Comparing the goodness of fit for selected {{IM}}s --- measured with score functions (see section~\ref{sec:scorefun}) that take the LOOCV residuals as input. Colored row-wise.}
			\scriptsize
			\begin{tabular}{lrrrrrrrrrr}
 & ss & loess & dl & bspl & fourier & ss rob & loess rob & dl rob & bspl rob & fourier rob \\
rmse & \background-color#e1e1e1 \color#000000 0.066 & \background-color#eeeeee \color#000000 0.062 & \background-color#f1f1f1 \color#000000 0.061 & \background-color#bfbfbf \color#000000 0.077 & \background-color#eeeeee \color#000000 0.062 & \background-color#cbcbcb \color#000000 0.073 & \background-color#dbdbdb \color#000000 0.068 & \background-color#e7e7e7 \color#000000 0.064 & \background-color#afafaf \color#000000 0.082 & \background-color#000000 \color#f1f1f1 0.137 \\
qtile50 & \background-color#646464 \color#f1f1f1 0.035 & \background-color#828282 \color#f1f1f1 0.032 & \background-color#bfbfbf \color#000000 0.026 & \background-color#1d1d1d \color#f1f1f1 0.042 & \background-color#a0a0a0 \color#f1f1f1 0.029 & \background-color#8d8d8d \color#f1f1f1 0.031 & \background-color#a0a0a0 \color#f1f1f1 0.029 & \background-color#f1f1f1 \color#000000 0.021 & \background-color#646464 \color#f1f1f1 0.035 & \background-color#000000 \color#f1f1f1 0.045 \\
qtile75 & \background-color#8e8e8e \color#f1f1f1 0.061 & \background-color#a2a2a2 \color#f1f1f1 0.057 & \background-color#d7d7d7 \color#000000 0.047 & \background-color#494949 \color#f1f1f1 0.074 & \background-color#c7c7c7 \color#000000 0.050 & \background-color#9d9d9d \color#f1f1f1 0.058 & \background-color#b7b7b7 \color#000000 0.053 & \background-color#f1f1f1 \color#000000 0.042 & \background-color#6d6d6d \color#f1f1f1 0.067 & \background-color#000000 \color#f1f1f1 0.088 \\
qtile85 & \background-color#b8b8b8 \color#000000 0.078 & \background-color#c4c4c4 \color#000000 0.074 & \background-color#e5e5e5 \color#000000 0.063 & \background-color#7c7c7c \color#f1f1f1 0.098 & \background-color#d9d9d9 \color#000000 0.067 & \background-color#b8b8b8 \color#000000 0.078 & \background-color#cacaca \color#000000 0.072 & \background-color#f1f1f1 \color#000000 0.059 & \background-color#8e8e8e \color#f1f1f1 0.092 & \background-color#000000 \color#f1f1f1 0.139 \\
qtile90 & \background-color#c8c8c8 \color#000000 0.090 & \background-color#d2d2d2 \color#000000 0.085 & \background-color#dedede \color#000000 0.080 & \background-color#9b9b9b \color#f1f1f1 0.111 & \background-color#d7d7d7 \color#000000 0.083 & \background-color#b4b4b4 \color#000000 0.099 & \background-color#cfcfcf \color#000000 0.087 & \background-color#f1f1f1 \color#000000 0.071 & \background-color#969696 \color#f1f1f1 0.113 & \background-color#000000 \color#f1f1f1 0.183 \\
qtile95 & \background-color#e6e6e6 \color#000000 0.120 & \background-color#f1f1f1 \color#000000 0.110 & \background-color#efefef \color#000000 0.112 & \background-color#d8d8d8 \color#000000 0.133 & \background-color#e7e7e7 \color#000000 0.119 & \background-color#dcdcdc \color#000000 0.130 & \background-color#eaeaea \color#000000 0.117 & \background-color#e8e8e8 \color#000000 0.118 & \background-color#bcbcbc \color#000000 0.160 & \background-color#000000 \color#f1f1f1 0.335 \\
\end{tabular}

			\normalsize
			\label{tab:cv-statistics_itpl-methods}
		\end{center}
	\end{table}
	
	DL performs the best among both robustified and non-robustified with respect to most of the score functions used (all except QAR\textsuperscript{95}) and is in particular superior to the other parametric approach, which is FS. Especially the robust FS performs poorly. The LOESS is superior than all other non-parametric methods with respect to every score function. However, it is closely followed by the SS. The BS exhibits the worst performance out of all non-parametric method tested here. 
}


\section{Yield Prediction Error for Tested ISs} \label{sec:results_ndvi_corr}
	\begin{table}
		\begin{center}
			\caption[Relative Yield Estimation Error for ISs]{Relative YPE for various ISs. For the non-relative YPE and the coefficient of determination (R\textsuperscript{2}) cf. table~\ref{tab:methods_vs_yieldprediction} and~\ref{tab:methods_vs_yieldprediction_r2}.}
			\small
			\begin{tabular}{lrrrrrrr}
\toprule
 & RF & OLS\textsuperscript{SCL} & OLS\textsuperscript{all} & MARS & GAM & LASSO & no corrections \\
\midrule
SS & {\cellcolor[HTML]{282828}} \color[HTML]{F1F1F1} 0.155 & {\cellcolor[HTML]{F1F1F1}} \color[HTML]{000000} 0.140 & {\cellcolor[HTML]{C9C9C9}} \color[HTML]{000000} 0.143 & {\cellcolor[HTML]{D6D6D6}} \color[HTML]{000000} 0.142 & {\cellcolor[HTML]{D6D6D6}} \color[HTML]{000000} 0.142 & {\cellcolor[HTML]{D6D6D6}} \color[HTML]{000000} 0.142 & {\cellcolor[HTML]{797979}} \color[HTML]{F1F1F1} 0.149 \\
SS\textsuperscript{rob} & {\cellcolor[HTML]{282828}} \color[HTML]{F1F1F1} 0.155 & {\cellcolor[HTML]{C9C9C9}} \color[HTML]{000000} 0.143 & {\cellcolor[HTML]{939393}} \color[HTML]{F1F1F1} 0.147 & {\cellcolor[HTML]{797979}} \color[HTML]{F1F1F1} 0.149 & {\cellcolor[HTML]{A0A0A0}} \color[HTML]{F1F1F1} 0.146 & {\cellcolor[HTML]{AEAEAE}} \color[HTML]{000000} 0.145 & {\cellcolor[HTML]{868686}} \color[HTML]{F1F1F1} 0.148 \\
DL & {\cellcolor[HTML]{1A1A1A}} \color[HTML]{F1F1F1} 0.156 & {\cellcolor[HTML]{5D5D5D}} \color[HTML]{F1F1F1} 0.151 & {\cellcolor[HTML]{505050}} \color[HTML]{F1F1F1} 0.152 & {\cellcolor[HTML]{505050}} \color[HTML]{F1F1F1} 0.152 & {\cellcolor[HTML]{797979}} \color[HTML]{F1F1F1} 0.149 & {\cellcolor[HTML]{797979}} \color[HTML]{F1F1F1} 0.149 & {\cellcolor[HTML]{000000}} \color[HTML]{F1F1F1} 0.158 \\
DL\textsuperscript{rob} & {\cellcolor[HTML]{0D0D0D}} \color[HTML]{F1F1F1} 0.157 & {\cellcolor[HTML]{434343}} \color[HTML]{F1F1F1} 0.153 & {\cellcolor[HTML]{505050}} \color[HTML]{F1F1F1} 0.152 & {\cellcolor[HTML]{AEAEAE}} \color[HTML]{000000} 0.145 & {\cellcolor[HTML]{868686}} \color[HTML]{F1F1F1} 0.148 & {\cellcolor[HTML]{6B6B6B}} \color[HTML]{F1F1F1} 0.150 & {\cellcolor[HTML]{0D0D0D}} \color[HTML]{F1F1F1} 0.157 \\
\bottomrule
\end{tabular}

			\label{tab:methods_vs_yieldprediction_relative}
			\normalsize
		\end{center}
	\end{table}
	The YPE for the in section~\ref{sec:corr_itpl_stat} chosen {{IS}}s is given in table~\ref{tab:methods_vs_yieldprediction_relative}. We note that robustification does not improve the quality of the fit (measured via the YPE) in most cases. 
	In terms of YPE, SS tend to be better than DL (with and without robustification), especially if no correction is made. The {{IS}} that leads to the lowest YPE is the OLS\textsuperscript{SCL} with SS. Note that it is the simplest model which performs best.Given that the OLS\textsuperscript{SCL} models have very good interpretability, we also present the regression equations below. The corrected NDVI is calculated using 
	\begin{equation}\label{eq:corr_lm}
		\begin{aligned}		
			{\widehat{\operatorname{NDVI}}}_{\text{true}}  = &
			0.711 \operatorname{NDVI}_\text{observed}  
			+ 0.215 \,\mathbbm 1_{{SCL = 2}} 
			+ 0.237 \,\mathbbm 1_{{SCL = 3}} 
			+ 0.210 \,\mathbbm 1_{{SCL = 4}} \\ &
			+ 0.116 \,\mathbbm 1_{{SCL = 5}} 
			+ 0.162 \,\mathbbm 1_{{SCL = 6}} 
			+ 0.327 \,\mathbbm 1_{{SCL = 7}} 
			+ 0.474 \,\mathbbm 1_{{SCL = 8}} \\ &
			+ 0.575 \,\mathbbm 1_{{SCL = 9}} 
			+ 0.306 \,\mathbbm 1_{{SCL = 10}} 
			+ 0.512 \,\mathbbm 1_{{SCL = 11}} 
		\end{aligned}
	\end{equation} 
	where $\mathbbm 1_{{SCL = 2}}$ is equal to one if the current observation corresponds to SCL class 2 and zero otherwise\footnote{$\mathbbm 1$ is also called an indicator function or characteristic function in mathematics.}. Whereas, we obtain the estimated absolute residuals (cf. equation~\refeq{eq:absndvires}) by: 
	\begin{equation}\label{eq:corr_lm_res}
		\begin{aligned}		
			\hat v  = &
			-0.133 \operatorname{NDVI}_\text{observed}  
			+ 0.186 \, \mathbbm 1_{{SCL = 2}} 
			+ 0.185 \, \mathbbm 1_{{SCL = 3}} 
			+ 0.146 \, \mathbbm 1_{{SCL = 4}} \\ &
			+ 0.089 \, \mathbbm 1_{{SCL = 5}} 
			+ 0.167 \, \mathbbm 1_{{SCL = 6}} 
			+ 0.203 \, \mathbbm 1_{{SCL = 7}} 
			+ 0.181 \, \mathbbm 1_{{SCL = 8}} \\ & 
			+ 0.173 \, \mathbbm 1_{{SCL = 9}} 
			+ 0.180 \, \mathbbm 1_{{SCL = 10}} 
			+ 0.172 \, \mathbbm 1_{{SCL = 11}} 
		\end{aligned}
	\end{equation} 
	Thus, if we observe a pixel with SCL class 4 (`vegetation') but a NDVI of only 0.4, the corrected NDVI would be $0.711 \cdot 0.4 + 0.21 = 0.494$ with an estimated absolute residual of $-0.133 \cdot 0.4 + 0.146 = 0.093$.
	In equation~\refeq{eq:corr_lm}, we notice the strongest upwards correction for SCL classes 8, 9 and 11 (`medium probability clouds', `high probability clouds' and `thin cirrus clouds'). The estimated absolute residuals, however, are the smallest for SCL classes 4 and 5 (`vegetation' and `bare soil'). Furthermore, the higher the observed NDVI the lower are the estimated absolute residuals.
	For the \texttt{R}-output of the \texttt{summary} function of the two models, we refer to the appendix~\ref{app:ols-scl-summary}. 


	








\chapter{Discussion}


\textbf{High RMSE in ...:} How much can we expect to get? We have multiple sources of uncertainty in the data:
1. Uncertainty in Yield data collected by the combine harverster
2. Uncertainty in Yield data through rasterization
3. Uncertainty in satellite images through ``measurement errors'' introduced via clouds and other atmospheric effects 
4. Uncertainty introduced by interpolating (especially when long data-gaps are present)


% \include{tex/example/Chapter1}
%%\include{tex/Chapter...}
\chapter{Conclusion}
\label{s:Conclusion}
\begin{verbatim}
    
- itpl methods, 
    parametric dl 
    non-param 
        discarded 
            kernel methods because of strong bias  
            kriging because assumptions and highdim parameters
            savitzky-golay filter since we will investigate the LOESS which can be thought as its generalization 
            loess slightly best performing itpl method but we notice non-smooth behaviour if data gaps are present
        loess > ss > bspl
        choose ss because of its meaningful definition (minimizing the integral of the second derivative squared (\refeq{eq:ss}))
- robustifying useful?

\end{verbatim}
XXX draw your conclusion to which you came during this thesis



\section{Future Work}{
    \label{ss:FutureWork}

    \subsection{Time Series Correction-Interpolation as a General Method}{
        Throughout this thesis, we developed a correction and interpolation method for the NDVI. However, we never used features of the NDVI. Only the parameter estimated via cross-validation in chapter \ref{sec:itpl_param_est} depends on the scale of the time series. For simplicity, we could thus determine the parameter using Generalized Cross Validation (as \citefullauthor{ripleyFitSmoothingSpline} suggest). Therefore, our approach of interpolation and correction of time series can be applied to arbitrary time series as long as additional information is available. However, further research is required, to demonstrate the usefulness of this approach in general.


        \subsubsection*{Example: Cloud Correction with Uncertainty Estimation and Interpolation}
            This generalization can be used in particular for cloud correction. In the same manner as we corrected the NDVI time series in chapter \ref{sec:corr}, we can correct each spectral band and reunite the corrected bands with the uncertainties. If desired, the time series can also be interpolated before merging as in chapter \ref{sec:corr_link}. The resulting question would be how well this approach performs.
    }



    \subsection{Minor Improvements}
        During this project, we also noticed some minor issues that we would have liked to investigate further if more resources were available. The most relevant of these are:
        \begin{Nitemize}
            \item \textbf{Data:}
            Method how data\todo{which data? I assume the combine harvester point data?} has been extrapolated to the grid could possibly be improved
            \item \textbf{Data:}
            For computational reasons, we mostly considered all years and split the data (on the pixel level) randomly into a train/test set. A leave one year out cross validation might yield more accurate results.
            \item \textbf{Data:}
            We have not included the spectral bands which have a resolution of 60 m. But precisely these seem to be promising for cloud correction, since they are a proxy of the water (content and form) in the atmosphere.
            % \item \textbf{Interpolation:}
            % Penalized Regressions as described in ... are similar to smoothing splines (c.f. ...) but different. Better?XXX
            \item \textbf{NDVI Correction:}
            Explore the effect of different link functions between the estimated absolute residuals and the weights in section \ref{sec:corr_link}.

            XXX weight/uncertainty function 
            (problem of weight function -> some outer points get really low weights (just because others in the middle have very little residuals and thus very high weight))
            \item \textbf{NDVI Correction:}
            Yield is not the only target variable of interest. Other variables like protein content could also be used in section \ref{sec:ndvi_corr_eval} for the method evaluation. 
        \end{Nitemize}
}


%%% Local Variables: 
%%% mode: latex
%%% TeX-master: "MasterThesisSfS"
%%% End: 

%%%%%%%%%%%%%%%%%%%%%%%%%%%%%%%%%%%%%%%%%%%%%%%%%
%%% Bibliography                              %%%
%%%%%%%%%%%%%%%%%%%%%%%%%%%%%%%%%%%%%%%%%%%%%%%%%
\addtocontents{toc}{\vspace{.5\baselineskip}}
\cleardoublepage
\phantomsection
\addcontentsline{toc}{chapter}{\protect\numberline{}{Bibliography}}
\bibliography{myReferences}
%% All books from our library (SfS) are already in a BiBTeX file
%% 'Assbib.bib' (included here as well), using
% \bibliography{myReferences,Assbib}
% ---------------------------------- instead of the above



%%%%%%%%%%%%%%%%%%%%%%%%%%%%%%%%%%%%%%%%%%%%%%%%% 
%%% Appendices (if needed, e.g. for R code)   %%%
%%%%%%%%%%%%%%%%%%%%%%%%%%%%%%%%%%%%%%%%%%%%%%%%%
\addtocontents{toc}{\vspace{.5\baselineskip}}
\appendix
\chapter{Reproducibility}\label{app:reproducibility}

\section{Reproduce Results}
For reproducibility of the whole computations, we refer to our codebase at:\\ \url{https://github.com/LGraz/MasterThesis-Code}\\ In order to reproduce our computations and results, set up the directory as described in the README. The `Yield Mapping' Data used, is published alongside \cite{perichPixelbasedCropYield2022}. Execute the computations via the script \texttt{./shell\_scripts/reproduce.sh} and do not execute the python and R files by hand (unless you follow the order in \texttt{./shell\_scripts/reproduce.sh}). 

\section{R--Package}
We also provide an \texttt{R} package for a general time series correction and interpolation if additional data is available at: \\
\url{https://github.com/LGraz/CorrectTimeSeries} \\
In our case, we consider the NDVI time series and the additional data consists of the unused spectral bands.

We recommend installing it via the \texttt{devtools} package by:\\
\texttt{devtools::install\_github("LGraz/CorrectTimeSeries")}

In the following, we shall give a stand-alone example of how the \texttt{R} package can be used:

\lstinputlisting[title= Example of how to use the \texttt{CorrectTimeSeries} package]{tex/misc/CorrectTimeSeries.R}


\chapter{Further Material}

\section{Reproducible Codebase}
\todo{refer to github and readme instructions}

\section{Interpolation}
\begin{my_figure}[h]{width=1\textwidth}{interpol/2x3_loess_robust}
	\caption{The LOESS smoother \RobItPlot}
	\label{fig:interpol/2x3_loess_robust}
\end{my_figure}

\begin{my_figure}[h]{width=1\textwidth}{interpol/2x3_B-Splines_robust}
	\caption{B-Splines \RobItPlot}
	\label{fig:interpol/2x3_B-Splines_robust}
\end{my_figure}

\begin{my_figure}[h]{width=1\textwidth}{interpol/2x3_DL_robust}
	\caption{A Double Logistic curve \RobItPlot}
	\label{fig:interpol/2x3_DL_robust}
\end{my_figure}




\section{NDVI correction}


% step_plot/2017-201_ndvi.pdf 
% step_plot/2017-202_itpl.pdf 
% step_plot/2017-203_itpl_rew.pdf 
% step_plot/2017-204_ndvi_scl.pdf 
% step_plot/2017-205_show_res.pdf 
% step_plot/2017-206_corr.pdf 
% step_plot/2017-207_uncert.pdf 
% step_plot/2017-208_corr_itpl_rew.pdf

\begin{figure*}
            \vspace{-15pt}
			\centering
			\begin{subfigure}[b]{0.42\textwidth}
				\centering
				\includegraphics[width=\textwidth]{step_plot/2017-201_ndvi.pdf}
				\vspace{-20pt}
                \caption[NDVI time series with scl45]%
				{{\footnotesize NDVI time series with scl45}}    
				\label{fig:step_plot/2017-201_ndvi.pdf}
			\end{subfigure}
			\hfill
			\begin{subfigure}[b]{0.42\textwidth}  
				\centering 
				\includegraphics[width=\textwidth]{step_plot/2017-202_itpl.pdf}
				\vspace{-20pt}
                \caption[Interpolation via Smoothing Splines]%
				{{\footnotesize Interpolation via Smoothing Splines}}    
				\label{fig:step_plot/2017-202_itpl.pdf}
			\end{subfigure}

			\vskip\baselineskip
			\begin{subfigure}[b]{0.42\textwidth}   
				\centering 
				\includegraphics[width=\textwidth]{step_plot/2017-203_itpl_rew.pdf}
				\vspace{-20pt}
                \caption[Robustly reweighted fit]%
				{{\footnotesize Robustly reweighted fit}}    
				\label{fig:step_plot/2017-203_itpl_rew.pdf}
			\end{subfigure}
			\hfill
			\begin{subfigure}[b]{0.42\textwidth}   
				\centering 
				\includegraphics[width=\textwidth]{step_plot/2017-204_ndvi_scl.pdf}
				\vspace{-20pt}
                \caption[Now also consider other SCL-classes]%
				{\footnotesize Now also consider other SCL-classes}    
				\label{fig:step_plot/2017-204_ndvi_scl.pdf}
			\end{subfigure}

            \vskip\baselineskip
			\begin{subfigure}[b]{0.42\textwidth}   
				\centering 
				\includegraphics[width=\textwidth]{step_plot/2017-205_show_res.pdf}
				\vspace{-20pt}
                \caption[OOB estim. for each point using SCL45]%
				{{\footnotesize OOB estim. for each point using SCL45}}    
				\label{fig:step_plot/2017-205_show_res.pdf}
			\end{subfigure}
			\hfill
			\begin{subfigure}[b]{0.42\textwidth}   
				\centering 
				\includegraphics[width=\textwidth]{step_plot/2017-206_corr.pdf}
				\vspace{-20pt}
                \caption[Correct NDVI]%
				{\footnotesize Correct NDVI}    
				\label{fig:step_plot/2017-206_corr.pdf}
			\end{subfigure}

            \vskip\baselineskip
			\begin{subfigure}[b]{0.42\textwidth}   
				\centering 
				\includegraphics[width=\textwidth]{step_plot/2017-207_uncert.pdf}
				\vspace{-20pt}
                \caption[Estimate absolute errors]%
				{{\footnotesize Estimate absolute errors}}    
				\label{fig:step_plot/2017-207_uncert.pdf}
			\end{subfigure}
			\hfill
			\begin{subfigure}[b]{0.42\textwidth}   
				\centering 
				\includegraphics[width=\textwidth]{step_plot/2017-208_corr_itpl_rew.pdf}
				\vspace{-20pt}
                \caption[Robust interpolation on weights derived from uncertainties.]%
				{\footnotesize Robust interpolation on weights derived from uncertainties.}    
				\label{fig:step_plot/2017-208_corr_itpl_rew.pdf}
			\end{subfigure}
            \caption{Stepwise illustration of robust NDVI-Correction}
		\end{figure*}


% \include{tex/example/Appendix1}
% \include{tex/example/Appendix2}
% \include{tex/example/Appendix_more_R}

\ifdraft \else %XXX
  % %% Epilogue (optional)
  % \addtocontents{toc}{\vspace{.5\baselineskip}}
  % \cleardoublepage
  % \phantomsection
  % \addcontentsline{toc}{chapter}{\protect\numberline{}{Epilogue}}
  % \markboth{Epilogue}{Epilogue}
  % \include{tex/misc/Epilogue}


  %%%%%%%%%%%%%%%%%%%%%%%%%%%%%%%%%%%%%%%%%%%%%%%%%% 
  %%% Declaration of originality (Do not remove!)%%%
  %%%%%%%%%%%%%%%%%%%%%%%%%%%%%%%%%%%%%%%%%%%%%%%%%%
  %% Instructions:
  %% -------------
  %% fill in the empty document confirmation-originality.pdf electronically
  %% print it out and sign it
  %% scan it in again and save the scan in this directory with name
  %% confirmation-originality-scan.pdf 
  %%
  %% General info on plagiarism:
  %% https://www.ethz.ch/students/en/studies/performance-assessments/plagiarism.html 
  \cleardoublepage
  \includepdf[pages={-}, frame=true,scale=1]{misc/confirmation-originality.pdf}
\fi
\end{document}

%%% Local Variables:
%%% mode: latex
%%% TeX-master: "MasterThesisSfS"
%%% End:
