\chapter{Data and Methods}\label{sec:data_methods}
{
	We will start by describing the available data and the challenges associated with it.
	Our study region is a farm of over 800ha, which is located in western Switzerland. From \cite{perichPixelbasedYieldMapping2022a} we acquire satellite image data (section \ref{sec:s2_img_data}), yield maps of several cereals from 2017 to 2021 (section \ref{sec:yieldmapping_data}), and meteorological data (section \ref{sec:gather_data_to_pixel}).
	Afterwards, we will introduce general methods in section \ref{sec:general_methods}, which will be used in the remaining chapters.
}


\section{Sentinel 2 Data}{
	\label{sec:s2_img_data}
	%\subsection*{General Information}
	{
		
		The European Space Agency (ESA)\footnote{REF: https://sentinel.esa.int/web/sentinel/missions/sentinel-2} freely distributes the high-quality images of the two Sentinel satellites (S2). Together, both satellites have a revisit time of 5 days at the Equator and 2-3 days at mid-latitudes. However, in our study region, we only receive an image every 5 days.
		
		\begin{table}[h]
    \centering
    \small
    \caption{List of spectral bands of the S2-satellites. Each band has its center at the wavelength $\lambda$ in $nm$ with the spectral width $\Delta\lambda$ in $nm$ with a spatial resolution $SR$ in $m$ \citep{jaramazESASentinel2Mission2013}.}
    \begin{tabular}{p{0.03\linewidth} p{0.04\linewidth} p{0.03\linewidth} p{0.03\linewidth} p{0.73\linewidth}}
    \toprule
        \hspace*{-5pt} Band & $\;\lambda$ & $\Delta\lambda$ & $SR$ & Purpose \\ \hline
        1 & 443 & 20 & 60 & Atmospheric correction (aerosol scattering) \\ %\hline
        2 & 490 & 65 & 10 & Sensitive to vegetation senescing, carotenoid, browning and soil background; atmospheric correction (aerosol scattering) \\ %\hline
        3 & 560 & 35 & 10 & Green peak, sensitive to total chlorophyll in vegetation \\ %\hline
        4 & 665 & 30 & 10 & Maximum chlorophyll absorption \\ %\hline
        5 & 705 & 15 & 20 & Position of red edge; consolidation of atmospheric corrections / fluorescence baseline. \\ %\hline
        6 & 740 & 15 & 20 & Position of red edge, atmospheric correction, retrieval of aerosol load. \\ %\hline
        7 & 783 & 20 & 20 & Leaf Area Index (LAI), edge of the Near-Infrared (NIR) plateau. \\ %\hline
        8 & 842 & 115 & 10 & LAI \\ %\hline
        8a & 865 & 20 & 20 & NIR plateau, sensitive to total chlorophyll, biomass, LAI and protein; water vapor absorption reference; retrieval of aerosol load and type. \\ %\hline
        9 & 945 & 20 & 60 & Water vapor absorption, atmospheric correction. \\ %\hline
        10 & 1375 & 30 & 60 & Detection of thin cirrus for atmospheric correction. \\ %\hline
        11 & 1610 & 90 & 20 & Sensitive to lignin, starch and forest above ground biomass. Snow/ice/cloud separation. \\ %\hline
        12 & 2190 & 180 & 20 & Assessment of Mediterranean vegetation conditions. Distinction of clay soils for the monitoring of soil erosion. Distinction between live biomass, dead biomass and soil, e.g. for burn scars mapping. \\
        \bottomrule
    \end{tabular}
    \label{table:S2-bands}
\end{table}

		
		The S2 images contain 12 spectral bands with spatial resolutions up to 10 meters (see \ref{table:S2-bands}). Bands with a lower resolution (20 and 60 meters) were upscaled to 10 meter resolution using cubic interpolation (\cite{perichPixelbasedYieldMapping2022a}). In order to decrease the effect of atmospheric conditions like reflections and scattering, bottom-of-atmosphere, radiometric corrected Level-2A data was used\footnote{According to \cite{perichPixelbasedYieldMapping2022a}: ``Data prior to March 2018 was only available in the top-of-atmosphere L1C format and was downloaded as such [...] L1C data was processed to L2A product level using the `Sen2Cor' processor provided by ESA''}. 
		The ESA also supplies an algorithm\footnote{REF https://sentinels.copernicus.eu/web/sentinel/technical-guides/sentinel-2-msi/level-2a/algorithm} produces Scene Classification Layer (\textit{SCL}) where for each location the observed subject is assigned to one of 11 SCL-classes (c.f. table~\ref{tab:satelite/scl_classes}). 
		In this thesis,  we will use this classification to filter out data points, which we belive to be less informative. That are all observations which SCL-class does not correspond to vegetation or bare soils (classes 4 and 5). For convenience, we define the set SCL45 as the observations which belong to SCL-class 4 or 5.
		
		% \begin{figure}[h]
		% 	\label{fig:satelite/sentinel-2-bands}
		% 	\center
		% 	\includegraphics[width=0.4\textwidth]{satelite/sentinel-2-bands.jpg}
		% 	\caption{XXX Sentinel 2 bands}
		% \end{figure}
				% \begin{table}[!h]
		% 	\caption{Overview: Scene Classification Layers (SCL)}
		% 	\label{tab:satelite/scl_classes}
		% 	\center
		% 	\includegraphics[width=0.8\textwidth]{satelite/scl_classes.png}
		% \end{table}
		
		\begin{table}[h]
			\caption{Overview: Scene Classification Layers (SCL)}
			\label{tab:satelite/scl_classes}
			\centering
			\small
			\begin{tblr}{
			  colspec = {p{0.05\linewidth} p{0.03\linewidth} p{0.3\linewidth} p{0.05\linewidth} p{0.03\linewidth} p{0.3\linewidth} },
			%   row{2} = {SCL3color},
			%   column{3} = {teal7},
			  cell{2}{1} = {SCL0color},
			  cell{3}{1} = {SCL1color},
			  cell{4}{1} = {SCL2color},
			  cell{5}{1} = {SCL3color},
			  cell{6}{1} = {SCL4color},
			  cell{7}{1} = {SCL5color},
			  cell{2}{4} = {SCL6color},
			  cell{3}{4} = {SCL7color},
			  cell{4}{4} = {SCL8color},
			  cell{5}{4} = {SCL9color},
			  cell{6}{4} = {SCL10color},
			  cell{7}{4} = {SCL11color},
			}
			% \toprule % not working :8
			\hline
			Color & No. & Class & Color & No. & Class \\
			\hline
			& 0: & Missing Data 	& & 6: &  Water\\	 
			& 1: & Saturated or defective pixel 	& & 7: &  Cloud low probability\\
			& 2: & Dark features / Shadows 	& & 8: &  Cloud medium probability\\
			& 3: & Cloud shadows 	& & 9: &  Cloud high probability\\
			& 4: & Vegetation 	& & 10: &  Thin cirrus cloud\\
			& 5: & Bare soils 	& & 11: &  Snow or ice\\
			  \hline
			%   \bottomrule
			\end{tblr}
		  \end{table}


	% 0: "#000000",  Missing Data
	% 1: "#ff0000",	 Saturated or defective pixel
	% 2: "#404040",  Dark features / Shadows
	% 3: "#bf8144",  Cloud shadows
	% 4: "#00ff3c",  Vegetation
	% 5: "#ffed50",  Bare soils
	% 6: "#0d00fa",  Water
	% 7: "#808080",  Cloud low probability
	% 8: "#bfbfbf",  Cloud medium probability
	% 9: "#eeeeee",  Cloud high probability
	% 10: "#0bb8f0", Thin cirrus cloud
	% 11: "#ffbfbf", Snow or ice

		

	}
}

\section{Crop Yield Data}{
	\label{sec:yieldmapping_data}
	The crop yield data were collected using a combine harvester. Equipped with GPS, the harvester drives over the fields and continuously estimates the dry crop yield density in $t/ha$ (see fig. \ref{fig:satelite/witzwil_2021_P112_yield_harvester_cropped}). 
	We take the data set derived in \cite{perichPixelbasedYieldMapping2022a}, where error-prone measurement points (such as during a tight curve of the combine harvester) were removed and then the yield map was rasterized using linear interpolation (c.f. fig. \ref{fig:satelite/witzwil_2021_P112_yield_cropped.png}). We summarize the rasterized dry-yield values by the following statistics:

	% tabelle anstadt histogramm, da schon zu viele figuren ``herumschwirren'' und es so kompakter ist.
	\begin{tabular}{l l l l l l l} 
		Minimum & 1st Quartile & Median & Mean  & 3rd Quartile & Maximum & Variance \\
		0.107   & 6.186        & 7.560  & 7.359 & 8.756        & 13.35   & 4.035
	\end{tabular}    

	Comparing the average per-field crop yield reported by the farmer with the yield estimated by the combine harvester shows that the latter overestimates crop yield by ca. $10\%$ (c.f. \cite{perichPixelbasedYieldMapping2022a}). Since the relative estimation error is approximately constant and we do not aim for an accurate yield prediction, we will not consider this deviation. 



	\begin{figure}
		\centering
		\begin{subfigure}{.5\textwidth}
			\centering
			\includegraphics[height=.75\linewidth]{satelite/witzwil_2021_P112_yield_harvester_cropped.png}
			\caption{Raw combine harvester data (cleaned)}
			\label{fig:satelite/witzwil_2021_P112_yield_harvester_cropped}
		\end{subfigure}%
		\begin{subfigure}{.5\textwidth}
			\centering
			\includegraphics[height=.75\linewidth]{satelite/witzwil_2021_P112_yield_cropped.png}
			\caption{rasterized to Sentinel 2 resolution.}
			\label{fig:satelite/witzwil_2021_P112_yield_cropped.png}
		\end{subfigure}
		\caption{Crop yield density map of a field. Ranges from 0.1 t/ha (black) to 5.35 t/ha (white) }
		\label{fig:satelite_witzwil_yield}
	\end{figure}

}



\section{The Concept of a `Pixel'}{
		\label{sec:gather_data_to_pixel}
		Before we join all the data, we define a few concepts.

	\subsection{Normalized Difference Vegetation Index (NDVI)}{% NDVI
		The well-known  (\textit{NDVI}) introduced in \cite{rouseMonitoringVernalAdvancement1974} can be calculated using the bands $B4$ and $B8$ (table \ref{table:S2-bands}) by:
		\begin{equation}
			NDVI = \frac{B8 - B4}{B8 + B4}
			\label{eq:ndvi}
		\end{equation}
		Note that we call the calculated values merely the \textit{observed NDVI}, as we must be aware of imprecisions due to clouds and shadows\todo{Please clarify this in more detail. We used pixels flagged with SCL 4 \& 5, but as can be seen in Fig. 2.1 d), this can yield erroneous NDVI values, etc.}. 
	}

		{% GDD & DAS
			To define a timescale, we consider Days After Sowing (\textit{DAS}) and a transformed timescale, Growing Degree Days (\textit{GDD}) (\cite{mcmasterGrowingDegreedaysOne1997}REF). The latter are defined as the cumulative sum (since sowing) of temperature above a given base temperature $T_{base}$. For cereals, we use $T_{base}=0$ (\cite{perichPixelbasedYieldMapping2022a}). Thus, the GGD for $n$ days after sowing will be equal to:\todo{Für den Leser wäre es interessant, wenn Du noch kurz die wichtigsten GDD Werte aus der Literatur beschreiben würdest (D.h. z.B. Sowing, Emergence of Plants, Anthesis, Senescence, Harvest)}
			\begin{equation}
				\label{eq:gdd}
				GDD_n := \sum_{i=0}^n \max(T_i - T_{base}, 0).
			\end{equation}
		} 

		Now we create a data set, which will contain all the necessary information\todo{necessary info for what? To answer the research questions asked in section XXX}. Given that we have the spectral data at a $10m \times 10m$ resolution, we introduce the concept of a Pixel. A \textit{Pixel} $P$ is associated with a $10m \times 10m$ square defined by the S2 satellites and contains all relevant information\todo{which?} for a season and this location. More precisely, $P$ is a collection of general information (like yield and coordinates) and all associated $P_t$ of a given season. Where $P_t$ represents a tuple of the spectral data for time $t$, the NDVI calculated from it, and the associated GDD. 
		We will call the resulting data set \textit{PIXELS}, as it is the collection of all Pixels (over all seasons). 
		
		% Finally, we split PIXELS randomly into a train ($80\%$) and test  ($20\%$) set. 
	\subsection{Challenges in S2 Data}{
				%satelite/time_series_2021_P112/15_scl5_2021-02-23.png
	%satelite/time_series_2021_P112/30_scl4_2021-05-09.png
	%satelite/time_series_2021_P112/33_scl9_2021-05-24.png
	%satelite/time_series_2021_P112/35_scl4_2021-06-03.png
	%satelite/time_series_2021_P112/40_scl10_2021-06-28.png
	%satelite/time_series_2021_P112/45_scl2_2021-07-23.png

\begin{figure*}
	\centering
	\begin{subfigure}[b]{0.31\textwidth}
		\centering
		\includegraphics[width=\textwidth]{satelite/time_series_2021_P112/15_scl5_2021-02-23.png}
		\caption[2021-02-23 scl5]%
		{{\small 2021-02-23 scl5}}    
		\label{fig:satelite/time_series_2021_P112/15_scl5_2021-02-23.png}
	\end{subfigure}
	\hfill
	\begin{subfigure}[b]{0.31\textwidth}  
		\centering 
		\includegraphics[width=\textwidth]{satelite/time_series_2021_P112/30_scl4_2021-05-09.png}
		\caption[2021-05-09 scl4]%
		{{\small 2021-05-09 scl4}}    
		\label{fig:satelite/time_series_2021_P112/30_scl4_2021-05-09.png}
	\end{subfigure}
	\hfill
	\begin{subfigure}[b]{0.31\textwidth}  
		\centering 
		\includegraphics[width=\textwidth]{satelite/time_series_2021_P112/33_scl9_2021-05-24.png}
		\caption[2021-05-24 scl9]%
		{{\small 2021-05-24 scl9}}    
		\label{fig:satelite/time_series_2021_P112/33_scl9_2021-05-24.png}
	\end{subfigure}

	\vskip\baselineskip
	\begin{subfigure}[b]{0.31\textwidth}   
		\centering 
		\includegraphics[width=\textwidth]{satelite/time_series_2021_P112/35_scl4_2021-06-03.png}
		\caption[2021-06-03 scl4]%
		{{\small 2021-06-03 scl4}}    
		\label{fig:satelite/time_series_2021_P112/35_scl4_2021-06-03.png}
	\end{subfigure}
	\hfill
	\begin{subfigure}[b]{0.31\textwidth}   
		\centering 
		\includegraphics[width=\textwidth]{satelite/time_series_2021_P112/40_scl10_2021-06-28.png}
		\caption[2021-06-28 scl10]%
		{\small 2021-06-28 scl10}    
		\label{fig:satelite/time_series_2021_P112/40_scl10_2021-06-28.png}
	\end{subfigure}
	\hfill
	\begin{subfigure}[b]{0.31\textwidth}  
		\centering 
		\includegraphics[width=\textwidth]{satelite/time_series_2021_P112/45_scl2_2021-07-23.png}
		\caption[2021-07-23 scl2]%
		{{\small 2021-07-23 scl2}}    
		\label{fig:satelite/time_series_2021_P112/45_scl2_2021-07-23.png}
	\end{subfigure}

	\vskip\baselineskip
	\begin{subfigure}[b]{0.7\textwidth}   
		\centering 
		\includegraphics[width=\textwidth]{interpol/ndvi_ts_scl.pdf}
		\caption[Corresponding NDVI time series]%
		{{\small Corresponding NDVI time series}}    
		\label{fig:interpol/ndvi_ts_scl45_grey.pdf}
	\end{subfigure}

	\caption{Satellite images of a field at selected times with a static background for orientation. Moreover, the NDVI time series of the red-highlighted pixel is shown in (g) colored by the SCL labels.} 
	\label{fig:witzwil_selected_satellite_images}
\end{figure*}


\todo{Hier noch eine NDVI Zeitreihe parallel dazu zeigen. Ansonsten wird nicht klar, warum wir die Interpolation überhaupt machen.}
			% Description of plot
			The figure~\ref{fig:witzwil_selected_satellite_images} shows a selection of 6 satellite images of a field, which display our challenges\todo{which challenges? were they introduced earlier? E.g. in the introduction?}. In February (image a), we see no vegetation but bare soil. At the beginning of May, we observe a cloudless dark green field. In (c) heavy cloud cover (SCL class 9) leads to a complete loss of plant information in this S2 observation. Figure (d) shows that the SCL classification is not reliable, since we evidently observe clouds. In (e) we see a pale green. This likely shimmers through cirrus clouds. 
			
			%% subfigures references:
			% (see. \ref{fig:satelite/time_series_2021_P112/15_scl5_2021-02-23.png})
			% (see. \ref{fig:satelite/time_series_2021_P112/30_scl4_2021-05-09.png})
			% (see. \ref{fig:satelite/time_series_2021_P112/33_scl9_2021-05-24.png})
			% (see. \ref{fig:satelite/time_series_2021_P112/35_scl4_2021-06-03.png})
			% (see. \ref{fig:satelite/time_series_2021_P112/40_scl10_2021-06-28.png})
			% (see. \ref{fig:satelite/time_series_2021_P112/45_scl2_2021-07-23.png})
	}
}

\section{General Methods}{\label{sec:general_methods}
	Here we will only introduce Methods which will accure in several places. For interpolation methods we refer to sections \ref{sec:itpl_parametric} and \ref{sec:itpl_nonparametric}, for a robustification strategy to section \ref{sec:loess_robustify}. In section \ref{sec:itpl_param_est} we describe a method to objectively determine the quality of an interpolation, and in section \ref{sec:corr_correction} we present the NDVI correction together with an adapted interpolation strategy.


	\subsection{Root Mean Square Error (RMSE)}
		In this section we describe different criteria to evaluate models. Hence, given a vector $y\in \R^n$ and its estimator $\hat y$ (estimated using the model), we define the RMSE as:
		\begin{equation}
			\label{eq:rmse}
			 \sqrt{\frac{1}{n}\sum_{i=1}^n (y_i - \hat y_i)^2}
		\end{equation}
		% keine definition für R2 und relative RMSE da diese nur im appendix verwendet werden. Die definition nehmen wir dort vor.
		
		\subsection{Out-Of-Bag (\textit{OOB}) and Leave-One-Out-Cross-Validation (\textit{LOOCV})}{ \label{sec:OOB_LOOCV}
		The rationale for OOB and LOOCV is that we intend to evaluate a model $M$ with unseen data. That is, if $D$ describes the entire dataset and we train a model on a subset of $D$, we can use the remaining data to evaluate the model. 
		
		To formally introduce this, let:
		$$
			D=\{(X_{[j,:]},y_j)|\; X\in\R^{n\times p}, y\in \R^n, j=1,\dots,n\}
		$$
		be a dataset, $i\in \{1,\dots,n\}$ and $M^{(-i)}$ a model fitted on a subset of $D\setminus\{(X_{[i,:]},y_i)\}$. Then we call $\hat y_i:= M^{(-i)}(X_{[i,:]})$ an \textit{OOB} estimator of $y_i$. If we do this for all $i\in\{1,\dots,n\}$, we obtain $\hat y := \left(\hat y_1,\dots,\hat y_n\right)$ the OOB estimator for $y\in \R^n$.

		In the bootstrap (e.g., random forest) framework, we define $\hat y_i$ to be the average of all computed and admissible $M^{(-i)}$. 
		
		In the case that $M^{(-i)}$ was fitted on the set $D\setminus\{(X_i,y_i)\}$ (i.e., not a true subset), we call the corresponding $\hat y_i$ also the LOOCV estimator.	

		If we optimize some parameter via OOB (or LOOCV) this means that we search for the parameter that minimizes some loss function which takes the OOB (or LOOCV) residuals. Usually we approximate this parameter by searching on a grid. 
	}

}
