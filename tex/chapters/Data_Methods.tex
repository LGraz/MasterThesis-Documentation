\chapter{Problem Description}

\section{Available Data}
	{
		XXX field region Witzwil, Data from gregor perich (ref xxx)
		fields over 5 years 
		cereals (not other cultures)
	}

	\subsection{Sentinel 2 Satellite Image Data}{
		\subsubsection*{General Information}{
			The European Space Agency (ESA) \footnote{REF: https://sentinel.esa.int/web/sentinel/missions/sentinel-2} freely distributes the high quality images of the two Sentinel satellites 2 (S2). Together, both satellites have a revisit time of 5 days at the equator and 2-3 at mid-latitudes. However, at our study region we only receive an image every 5 days.
			In order to decrease the effect of atmospheric conditions like reflections and scattering, we will not work with the raw data but with the results of the Level-2A processing\footnote{REF https://sentinels.copernicus.eu/web/sentinel/technical-guides/sentinel-2-msi/level-2a/algorithm}\footnote{XXXREF gregor perich ``Data prior to March 2018 was only145
			available in the top-of-atmosphere L1C format and was downloaded as such [...] L1C data was processed to L2A product level using the `Sen2Cor' processor provided by ESA''}. 
		}

		\subsubsection*{Data Description}{
			The Level-2A processed images we use contain 12 spectral bands with local resolutions up to 10 meters (see \ref{table:S2-bands}).   
			\begin{table}[h]
    \centering
    \small
    \caption{List of spectral bands of the S2-satellites. Each band has its center at the wavelength $\lambda$ in $nm$ with the spectral width $\Delta\lambda$ in $nm$ with a spatial resolution $SR$ in $m$ \citep{jaramazESASentinel2Mission2013}.}
    \begin{tabular}{p{0.03\linewidth} p{0.04\linewidth} p{0.03\linewidth} p{0.03\linewidth} p{0.73\linewidth}}
    \toprule
        \hspace*{-5pt} Band & $\;\lambda$ & $\Delta\lambda$ & $SR$ & Purpose \\ \hline
        1 & 443 & 20 & 60 & Atmospheric correction (aerosol scattering) \\ %\hline
        2 & 490 & 65 & 10 & Sensitive to vegetation senescing, carotenoid, browning and soil background; atmospheric correction (aerosol scattering) \\ %\hline
        3 & 560 & 35 & 10 & Green peak, sensitive to total chlorophyll in vegetation \\ %\hline
        4 & 665 & 30 & 10 & Maximum chlorophyll absorption \\ %\hline
        5 & 705 & 15 & 20 & Position of red edge; consolidation of atmospheric corrections / fluorescence baseline. \\ %\hline
        6 & 740 & 15 & 20 & Position of red edge, atmospheric correction, retrieval of aerosol load. \\ %\hline
        7 & 783 & 20 & 20 & Leaf Area Index (LAI), edge of the Near-Infrared (NIR) plateau. \\ %\hline
        8 & 842 & 115 & 10 & LAI \\ %\hline
        8a & 865 & 20 & 20 & NIR plateau, sensitive to total chlorophyll, biomass, LAI and protein; water vapor absorption reference; retrieval of aerosol load and type. \\ %\hline
        9 & 945 & 20 & 60 & Water vapor absorption, atmospheric correction. \\ %\hline
        10 & 1375 & 30 & 60 & Detection of thin cirrus for atmospheric correction. \\ %\hline
        11 & 1610 & 90 & 20 & Sensitive to lignin, starch and forest above ground biomass. Snow/ice/cloud separation. \\ %\hline
        12 & 2190 & 180 & 20 & Assessment of Mediterranean vegetation conditions. Distinction of clay soils for the monitoring of soil erosion. Distinction between live biomass, dead biomass and soil, e.g. for burn scars mapping. \\
        \bottomrule
    \end{tabular}
    \label{table:S2-bands}
\end{table}

			Bands which have a lower resolution (20 and 60 meters) will be scaled up to 10 meters using cubic interpolation (REF gregor perich).
							% \begin{figure}[h]
							% 	\label{fig:satelite/sentinel-2-bands}
							% 	\center
							% 	\includegraphics[width=0.4\textwidth]{satelite/sentinel-2-bands.jpg}
							% 	\caption{XXX Sentinel 2 bands}
							% \end{figure}
			Additional to the spectral bands the ESA also supplies a \textbf{S}cene \textbf{C}lassification \textbf{L}ayer (\textit{SCL}) where for each location the observed subject is assigned to an \textit{SCL-class} (c.f. table~\ref{tab:satelite/scl_classes}). In chapter \ref{sec:itpl}  we will use this classification to filter out unreliable data points considering only SCL-classes 4 and 5.  
			
			
			\begin{table}[h]
				\caption{Overview: Scene Classification Layers (SCL)}
				\label{tab:satelite/scl_classes}
				\center
				\includegraphics[width=0.8\textwidth]{satelite/scl_classes.png}
			\end{table}
		}

		\subsubsection*{Data Illustration}{
			xxx plot beschreiben
			% 2x3 plot
				%satelite/time_series_2021_P112/15_scl5_2021-02-23.png
	%satelite/time_series_2021_P112/30_scl4_2021-05-09.png
	%satelite/time_series_2021_P112/33_scl9_2021-05-24.png
	%satelite/time_series_2021_P112/35_scl4_2021-06-03.png
	%satelite/time_series_2021_P112/40_scl10_2021-06-28.png
	%satelite/time_series_2021_P112/45_scl2_2021-07-23.png

\begin{figure*}
	\centering
	\begin{subfigure}[b]{0.31\textwidth}
		\centering
		\includegraphics[width=\textwidth]{satelite/time_series_2021_P112/15_scl5_2021-02-23.png}
		\caption[2021-02-23 scl5]%
		{{\small 2021-02-23 scl5}}    
		\label{fig:satelite/time_series_2021_P112/15_scl5_2021-02-23.png}
	\end{subfigure}
	\hfill
	\begin{subfigure}[b]{0.31\textwidth}  
		\centering 
		\includegraphics[width=\textwidth]{satelite/time_series_2021_P112/30_scl4_2021-05-09.png}
		\caption[2021-05-09 scl4]%
		{{\small 2021-05-09 scl4}}    
		\label{fig:satelite/time_series_2021_P112/30_scl4_2021-05-09.png}
	\end{subfigure}
	\hfill
	\begin{subfigure}[b]{0.31\textwidth}  
		\centering 
		\includegraphics[width=\textwidth]{satelite/time_series_2021_P112/33_scl9_2021-05-24.png}
		\caption[2021-05-24 scl9]%
		{{\small 2021-05-24 scl9}}    
		\label{fig:satelite/time_series_2021_P112/33_scl9_2021-05-24.png}
	\end{subfigure}

	\vskip\baselineskip
	\begin{subfigure}[b]{0.31\textwidth}   
		\centering 
		\includegraphics[width=\textwidth]{satelite/time_series_2021_P112/35_scl4_2021-06-03.png}
		\caption[2021-06-03 scl4]%
		{{\small 2021-06-03 scl4}}    
		\label{fig:satelite/time_series_2021_P112/35_scl4_2021-06-03.png}
	\end{subfigure}
	\hfill
	\begin{subfigure}[b]{0.31\textwidth}   
		\centering 
		\includegraphics[width=\textwidth]{satelite/time_series_2021_P112/40_scl10_2021-06-28.png}
		\caption[2021-06-28 scl10]%
		{\small 2021-06-28 scl10}    
		\label{fig:satelite/time_series_2021_P112/40_scl10_2021-06-28.png}
	\end{subfigure}
	\hfill
	\begin{subfigure}[b]{0.31\textwidth}  
		\centering 
		\includegraphics[width=\textwidth]{satelite/time_series_2021_P112/45_scl2_2021-07-23.png}
		\caption[2021-07-23 scl2]%
		{{\small 2021-07-23 scl2}}    
		\label{fig:satelite/time_series_2021_P112/45_scl2_2021-07-23.png}
	\end{subfigure}

	\vskip\baselineskip
	\begin{subfigure}[b]{0.7\textwidth}   
		\centering 
		\includegraphics[width=\textwidth]{interpol/ndvi_ts_scl.pdf}
		\caption[Corresponding NDVI time series]%
		{{\small Corresponding NDVI time series}}    
		\label{fig:interpol/ndvi_ts_scl45_grey.pdf}
	\end{subfigure}

	\caption{Satellite images of a field at selected times with a static background for orientation. Moreover, the NDVI time series of the red-highlighted pixel is shown in (g) colored by the SCL labels.} 
	\label{fig:witzwil_selected_satellite_images}
\end{figure*}



			% Description of plot
			In fig.~\ref{fig:witzwil_selected_satellite_images} 

			DE:  
			Die Abb.~\ref{fig:witzwil_selected_satellite_images} zeigt eine Auswahl von 6 Satelitenbildern von einer Parzelle, welche unsere Herrausvorderungen aufzeigen. Im Februar (Bild(a)) sehen wir wie erwartet keine Vegetation, sondern nackte Erde. Anfang Mai beobachten wir ein wolkenfreies dunkelgrünes feld. In (c) wird ersichtlich, dass wir bei starker Bewölkung keine Hoffnung haben nützliche information zu erhalten. Bild (d) zeigt auf, dass die SCL-Klassifizierung nicht zuverlässig ist. In (e) sehen wir ein blasses Grün. Vermutlich sehen wir durch zirrus wolken hindurch.   
			
			%% subfigures references:
			% (see. \ref{fig:satelite/time_series_2021_P112/15_scl5_2021-02-23.png})
			% (see. \ref{fig:satelite/time_series_2021_P112/30_scl4_2021-05-09.png})
			% (see. \ref{fig:satelite/time_series_2021_P112/33_scl9_2021-05-24.png})
			% (see. \ref{fig:satelite/time_series_2021_P112/35_scl4_2021-06-03.png})
			% (see. \ref{fig:satelite/time_series_2021_P112/40_scl10_2021-06-28.png})
			% (see. \ref{fig:satelite/time_series_2021_P112/45_scl2_2021-07-23.png})
		}
	}

	\subsection{Yieldmapping Data}{
		XXX description of how harvester gets data, knn-interpolation and rasterization (using linear interpolation), reference to gregors paper
		note: discrepancy between sum of estimated raster and manually weighted yield (per field per year)
		\begin{figure}
			\centering
			\begin{subfigure}{.5\textwidth}
			  \centering
			  \includegraphics[height=.75\linewidth]{satelite/witzwil_2021_P112_yield_harvester_cropped.png}
			  \caption{A subfigure XXX}
			%   \label{fig:sub1}
			\end{subfigure}%
			\begin{subfigure}{.5\textwidth}
				\centering
				\includegraphics[height=.75\linewidth]{satelite/witzwil_2021_P112_yield_cropped.png}
			  \caption{A subfigure xxx}
			%   \label{fig:sub2}
			\end{subfigure}
			\caption{xxx}
			\label{fig:test}
		\end{figure}
	}



	Gather Data:
	define Pixel

	scale of x axis?
	define DAS ,define GDD --> compare (plot)  (note that gdd can be non-unique)
	get GDD

	get NDVI

	DE
	Mit den  Bändern $B4$ und $B8$ berechnen wir den bekannten \textbf{N}ormalized \textbf{D}ifference \textbf{V}egetation \textbf{I}ndex (\textit{NDVI}) anhand der Formel:
	\begin{equation}
		NDVI = \frac{B8 - B4}{B8 + B4}
		\label{eq:ndvi}
	\end{equation}
	Bemerke, dass wir die berechneten Werte nur den \textit{observed NDVI} nennen, da wir aufgrund von wolken und schatten auf ungenauigkeiten gefasst seien müssen. 
