\chapter{Problem Description}

\section{Available Data}
	{
		Our study region is a farm of over 800ha, which is located in western Switzerland. From REF-gregor we acquire satellite image data (section \ref{sec:s2_img_data}), yield maps of several cereals from 2017 to 2021 (section \ref{sec:yieldmapping_data}), and meteorological data (section \ref{sec:gather_data_to_pixel}).
	}

	\subsection{Sentinel 2 Satellite Image Data}{
		\label{sec:s2_img_data}
		\subsubsection*{General Information}{
			The European Space Agency (ESA) \footnote{REF: https://sentinel.esa.int/web/sentinel/missions/sentinel-2} freely distributes the high-quality images of the two Sentinel satellites 2 (S2). Together, both satellites have a revisit time of 5 days at the equator and 2-3 at mid-latitudes. However, in our study region, we only receive an image every 5 days.
			In order to decrease the effect of atmospheric conditions like reflections and scattering, we will not work with the raw data but with the results of the Level-2A processing\footnote{REF https://sentinels.copernicus.eu/web/sentinel/technical-guides/sentinel-2-msi/level-2a/algorithm}\footnote{XXXREF gregor perich ``Data prior to March 2018 was only145
			available in the top-of-atmosphere L1C format and was downloaded as such [...] L1C data was processed to L2A product level using the `Sen2Cor' processor provided by ESA''}. 
		}

		\subsubsection*{Data Description}{
			The Level-2A processed images we use contain 12 spectral bands with local resolutions up to 10 meters (see \ref{table:S2-bands}).   
			\begin{table}[h]
    \centering
    \small
    \caption{List of spectral bands of the S2-satellites. Each band has its center at the wavelength $\lambda$ in $nm$ with the spectral width $\Delta\lambda$ in $nm$ with a spatial resolution $SR$ in $m$ \citep{jaramazESASentinel2Mission2013}.}
    \begin{tabular}{p{0.03\linewidth} p{0.04\linewidth} p{0.03\linewidth} p{0.03\linewidth} p{0.73\linewidth}}
    \toprule
        \hspace*{-5pt} Band & $\;\lambda$ & $\Delta\lambda$ & $SR$ & Purpose \\ \hline
        1 & 443 & 20 & 60 & Atmospheric correction (aerosol scattering) \\ %\hline
        2 & 490 & 65 & 10 & Sensitive to vegetation senescing, carotenoid, browning and soil background; atmospheric correction (aerosol scattering) \\ %\hline
        3 & 560 & 35 & 10 & Green peak, sensitive to total chlorophyll in vegetation \\ %\hline
        4 & 665 & 30 & 10 & Maximum chlorophyll absorption \\ %\hline
        5 & 705 & 15 & 20 & Position of red edge; consolidation of atmospheric corrections / fluorescence baseline. \\ %\hline
        6 & 740 & 15 & 20 & Position of red edge, atmospheric correction, retrieval of aerosol load. \\ %\hline
        7 & 783 & 20 & 20 & Leaf Area Index (LAI), edge of the Near-Infrared (NIR) plateau. \\ %\hline
        8 & 842 & 115 & 10 & LAI \\ %\hline
        8a & 865 & 20 & 20 & NIR plateau, sensitive to total chlorophyll, biomass, LAI and protein; water vapor absorption reference; retrieval of aerosol load and type. \\ %\hline
        9 & 945 & 20 & 60 & Water vapor absorption, atmospheric correction. \\ %\hline
        10 & 1375 & 30 & 60 & Detection of thin cirrus for atmospheric correction. \\ %\hline
        11 & 1610 & 90 & 20 & Sensitive to lignin, starch and forest above ground biomass. Snow/ice/cloud separation. \\ %\hline
        12 & 2190 & 180 & 20 & Assessment of Mediterranean vegetation conditions. Distinction of clay soils for the monitoring of soil erosion. Distinction between live biomass, dead biomass and soil, e.g. for burn scars mapping. \\
        \bottomrule
    \end{tabular}
    \label{table:S2-bands}
\end{table}

			Bands which have a lower resolution (20 and 60 meters) will be scaled up to 10 meters using cubic interpolation (REF gregor perich).
							% \begin{figure}[h]
							% 	\label{fig:satelite/sentinel-2-bands}
							% 	\center
							% 	\includegraphics[width=0.4\textwidth]{satelite/sentinel-2-bands.jpg}
							% 	\caption{XXX Sentinel 2 bands}
							% \end{figure}
			Additional to the spectral bands, the ESA also supplies a Scene Classification Layer (\textit{SCL}) where for each location the observed subject is assigned to an \textit{SCL-class} (cf. table~\ref{tab:satelite/scl_classes}). In chapter \ref{sec:itpl}  we will use this classification to filter out unreliable data points, considering only SCL-classes 4 and 5.  
			
			
			\begin{table}[h]
				\caption{Overview: Scene Classification Layers (SCL)}
				\label{tab:satelite/scl_classes}
				\center
				\includegraphics[width=0.8\textwidth]{satelite/scl_classes.png}
			\end{table}
		}

		\subsubsection*{Data Illustration}{
				%satelite/time_series_2021_P112/15_scl5_2021-02-23.png
	%satelite/time_series_2021_P112/30_scl4_2021-05-09.png
	%satelite/time_series_2021_P112/33_scl9_2021-05-24.png
	%satelite/time_series_2021_P112/35_scl4_2021-06-03.png
	%satelite/time_series_2021_P112/40_scl10_2021-06-28.png
	%satelite/time_series_2021_P112/45_scl2_2021-07-23.png

\begin{figure*}
	\centering
	\begin{subfigure}[b]{0.31\textwidth}
		\centering
		\includegraphics[width=\textwidth]{satelite/time_series_2021_P112/15_scl5_2021-02-23.png}
		\caption[2021-02-23 scl5]%
		{{\small 2021-02-23 scl5}}    
		\label{fig:satelite/time_series_2021_P112/15_scl5_2021-02-23.png}
	\end{subfigure}
	\hfill
	\begin{subfigure}[b]{0.31\textwidth}  
		\centering 
		\includegraphics[width=\textwidth]{satelite/time_series_2021_P112/30_scl4_2021-05-09.png}
		\caption[2021-05-09 scl4]%
		{{\small 2021-05-09 scl4}}    
		\label{fig:satelite/time_series_2021_P112/30_scl4_2021-05-09.png}
	\end{subfigure}
	\hfill
	\begin{subfigure}[b]{0.31\textwidth}  
		\centering 
		\includegraphics[width=\textwidth]{satelite/time_series_2021_P112/33_scl9_2021-05-24.png}
		\caption[2021-05-24 scl9]%
		{{\small 2021-05-24 scl9}}    
		\label{fig:satelite/time_series_2021_P112/33_scl9_2021-05-24.png}
	\end{subfigure}

	\vskip\baselineskip
	\begin{subfigure}[b]{0.31\textwidth}   
		\centering 
		\includegraphics[width=\textwidth]{satelite/time_series_2021_P112/35_scl4_2021-06-03.png}
		\caption[2021-06-03 scl4]%
		{{\small 2021-06-03 scl4}}    
		\label{fig:satelite/time_series_2021_P112/35_scl4_2021-06-03.png}
	\end{subfigure}
	\hfill
	\begin{subfigure}[b]{0.31\textwidth}   
		\centering 
		\includegraphics[width=\textwidth]{satelite/time_series_2021_P112/40_scl10_2021-06-28.png}
		\caption[2021-06-28 scl10]%
		{\small 2021-06-28 scl10}    
		\label{fig:satelite/time_series_2021_P112/40_scl10_2021-06-28.png}
	\end{subfigure}
	\hfill
	\begin{subfigure}[b]{0.31\textwidth}  
		\centering 
		\includegraphics[width=\textwidth]{satelite/time_series_2021_P112/45_scl2_2021-07-23.png}
		\caption[2021-07-23 scl2]%
		{{\small 2021-07-23 scl2}}    
		\label{fig:satelite/time_series_2021_P112/45_scl2_2021-07-23.png}
	\end{subfigure}

	\vskip\baselineskip
	\begin{subfigure}[b]{0.7\textwidth}   
		\centering 
		\includegraphics[width=\textwidth]{interpol/ndvi_ts_scl.pdf}
		\caption[Corresponding NDVI time series]%
		{{\small Corresponding NDVI time series}}    
		\label{fig:interpol/ndvi_ts_scl45_grey.pdf}
	\end{subfigure}

	\caption{Satellite images of a field at selected times with a static background for orientation. Moreover, the NDVI time series of the red-highlighted pixel is shown in (g) colored by the SCL labels.} 
	\label{fig:witzwil_selected_satellite_images}
\end{figure*}



			% Description of plot
			The figure ~\ref{fig:witzwil_selected_satellite_images} shows a selection of 6 satellite images of a field, which display our challenges. In February (image(a)), as expected, we see no vegetation but bare soil. At the beginning of May, we observe a cloudless dark green field. In (c) it is obvious that we have no chance to get useful information when there is a heavy cloud cover. Figure (d) shows that the SCL classification is not reliable, since we evidently observe clouds. In (e) we see a pale green. This likely shimmers through cirrus clouds. 
			
			%% subfigures references:
			% (see. \ref{fig:satelite/time_series_2021_P112/15_scl5_2021-02-23.png})
			% (see. \ref{fig:satelite/time_series_2021_P112/30_scl4_2021-05-09.png})
			% (see. \ref{fig:satelite/time_series_2021_P112/33_scl9_2021-05-24.png})
			% (see. \ref{fig:satelite/time_series_2021_P112/35_scl4_2021-06-03.png})
			% (see. \ref{fig:satelite/time_series_2021_P112/40_scl10_2021-06-28.png})
			% (see. \ref{fig:satelite/time_series_2021_P112/45_scl2_2021-07-23.png})
		}
	}

	\subsection{Harvest Yield Data}{
		\label{sec:yieldmapping_data}
		The crop yield data were collected using a combine harvester. Equipped with GPS, the harvester drives over the fields and continuously estimates the crop density in $t/ha$ (see fig. \ref{fig:satelite/witzwil_2021_P112_yield_harvester_cropped}). 
		We take the data set derived from this in REF-Gregor-Perich, where error-prone measurement points (such as during an egen curve) were removed and then the yield map was rasterized using linear interpolation (cf. fig. \ref{fig:satelite/witzwil_2021_P112_yield_cropped.png}).    

		
		Comparing the manually weighted yield and the sum of estimated raster (per field per year) we note a discrepancy of about $10\%$ (cf. REF-gregor). Since the relative estimation error is rather constant and we do not aim to estimate the absolute yield we will not consider this deviation.
		\begin{figure}
			\centering
			\begin{subfigure}{.5\textwidth}
			  \centering
			  \includegraphics[height=.75\linewidth]{satelite/witzwil_2021_P112_yield_harvester_cropped.png}
			  \caption{obtained by a combine harvester (cleaned)}
			  \label{fig:satelite/witzwil_2021_P112_yield_harvester_cropped}
			\end{subfigure}%
			\begin{subfigure}{.5\textwidth}
				\centering
				\includegraphics[height=.75\linewidth]{satelite/witzwil_2021_P112_yield_cropped.png}
			  \caption{rasterized to Sentinel 2 resolution.}
			  \label{fig:satelite/witzwil_2021_P112_yield_cropped.png}
			\end{subfigure}
			\caption{Crop yield density map of a field. Ranges from 0.1 t/ha (black) to 5.35 t/ha (white) }
			\label{fig:satelite_witzwil_yield}
		\end{figure}
	}


	\subsection{Gather Data}{
		\label{sec:gather_data_to_pixel}
		Before we join all the data, we define a few concepts.

		{% NDVI
			Using bands $B4$ and $B8$, we calculate the well-known Normalized Difference Vegetation Index (\textit{NDVI}) using the formula: (???REF nötig?)
			\begin{equation}
				NDVI = \frac{B8 - B4}{B8 + B4}
				\label{eq:ndvi}
			\end{equation}
			Note that we call the calculated values merely the \textit{observed NDVI}, as we must be aware of imprecisions due to clouds and shadows. 
		}

		{% GDD & DAS
			To define a timescale, we consider Days After Sowing (\textit{DAS}) and a transformed timescale, Growing Degree Days (\textit{GDD}) (\cite{mcmasterGrowingDegreedaysOne1997}). The latter are defined as the cumulative sum (since sowing) of temperature above a given base temperature $T_{base}$ \footnote{XXX For cereals we use $T_{base}=0$ }. Thus, the GGD for $n$ days after sowing will be equal to:
			\begin{equation}
				\label{eq:gdd}
				GDD_n := \sum_{i=0}^n \max(T_i - T_{base}, 0).
			\end{equation}
		} 

		Now we create a data set, which will contain all the necessary information. Given that we have the spectral data at a $10m \times 10m$ resolution, we introduce the concept of a Pixel. A \textit{Pixel} $P$ is associated with a $10m \times 10m$ square defined by the S2 satellites and contains all relevant information for a season and this location. More precisely, $P$ is a collection of general information (like yield and coordinates) and all associated $P_t$ of a given season. Where $P_t$ represents a tuple of the spectral data for time $t$, the NDVI calculated from it, and the associated GDD. 
		We will call the resulting data set \textit{PIXELS} as it is the collection of all Pixels (over all seasons). 
		
		Finally, we split PIXELS randomly into a train ($80\%$) and test  ($20\%$) set. 

	}

\section{General Methods}{
	We will only introduce general methods within this section, whereas more specific methods will be introduced in their context. We discuss interpolation methods in sections \ref{sec:itpl_parametric} and \ref{sec:itpl_nonparametric} , a robustification strategy in section \ref{sec:loess_robustify}, a method how we can objectively determine the quality of an interpolation in section \ref{sec:itpl_param_est}, and in section \ref{sec:corr_correction} we present the NDVI correction with an adapted interpolation strategy.

	\subsection{XXX Out-Of-Bag (\textit{OOB}) and Leave-One-Out-Cross-Validation (\textit{LOOCV})}{
		\label{sec:OOB_LOOCV}
	}
}
