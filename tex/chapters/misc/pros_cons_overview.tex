\begin{table}[!ht]
	\centering
	\caption[skip=10pt]{Summary of the studied interpolation methods containing important assumptions, advantages and disadvantages and whether the method supports weighted observations (w) and if the resulting interpolation is bounded w.r.t. a fixed interval (b).}
	\small
	\begin{tabular}{p{1.6cm}p{3.3cm}p{3.3cm}p{3.4cm}p{0.4cm}p{0.4cm}p{3cm}p{3cm}p{3cm}p{3cm}p{2.7cm}p{3cm}|}
		\toprule
		% \hline
		~                                                                                                                                                            &
		\textbf{Assumptions}                                                                                                                                         &
		\textbf{Advantages}                                                                                                                                                &
		\textbf{Disadvantages}                                                                                                                                                &
		\textbf{w}                                                                                                                                      &
		\textbf{b}                                                                                                                                        \\ \hline

		Double-Logistic                                                                                                                                              &
		\begin{cptitemize} \item[--]  Function first increases then decreases \item[--]  NDVI has a minimal value                            \end{cptitemize}        &
		\begin{cptitemize} \item[--]  Good for evergreen plants (if snow masks NDVI) \item[--]  Upper envelope                                \end{cptitemize}        &
		\begin{cptitemize} \item[--]  Parameter estimation can be very difficult \item[--]  Strange behavior for long data-gaps             \end{cptitemize}        &
		Yes                                                                                                                                                          &
		(Yes)                                                                                                                                                         \\ \hline%comment out?

		Fourier Approximation                                                                                                                                              &
		\begin{cptitemize} \item[--]  NDVI can be approximated by a 2cd order Fourier series.                            \end{cptitemize}        &
		\begin{cptitemize} \item[--]  Incorporates periodical growth-cycles                                 \end{cptitemize}        &
		\begin{cptitemize} \item[--]  Parameter estimation can be very difficult \item[--]  Curve easily exceeds bounds of the NDVI             \end{cptitemize}        &
		Yes                                                                                                                                                          &
		No                                                                                                                                                         \\ \hline%comment out?

		(Gaussian) Kernel Smoothing                                                                                                                                  &
		\begin{cptitemize} \item[--]  Close points are related to each other via a kernel function \end{cptitemize}                                                                                                                                                            &
		\begin{cptitemize} \item[--]  Simple  \item[--]  Computationally very fast                                                             \end{cptitemize}        &
		\begin{cptitemize} \item[--]  Biased, especially at `peaks' and `valleys'   \item[--]  Bandwidth: fails if there are big data-gaps                                                     \end{cptitemize}               &
		Yes                                                                                                                                                          &
		Yes                                                                                                                                                            \\ \hline%comment out?

		Universal Kriging                                                                                                                                            &
		\begin{cptitemize} \item[--]  Function is a realization of a stationary Gaussian process                                      \end{cptitemize}               &
		\begin{cptitemize} \item[--]  Informative parameters \item[--]  Flexible                                                             \end{cptitemize}        &
		\begin{cptitemize} \item[--]  Regression to the mean \item[--]  Assumptions clearly not met                                          \end{cptitemize}        &
		Yes                                                                                                                                                          &
		(Yes)                \\ \hline%comment out?                                                                                                                                         \\ %\hline%comment out?

		SG Filter                                                                                                                                        &
		\begin{cptitemize}
			\item[--]  High frequencies are noise (Low-Pass-Filter) \item[--]  Equidistant points \item[--]  Local polynomials\end{cptitemize}                                              &
		\begin{cptitemize} \item[--]  Computationally very fast                                                                   \end{cptitemize}                   &
		\begin{cptitemize} \item[--]  Cannot deal natively with missing data (need some interpolation)                              \end{cptitemize}                 &
		No                                                                                                                                                           &
		(Yes)                                                                                                                                                         \\ \hline%comment out?

		SG +~NDVI                                                                                                                                                    &
		\begin{cptitemize} \item[--]  Upper envelope \item[--]  Vegetation cannot grow faster than some slope                                \end{cptitemize}        &
		\begin{cptitemize} \item[--]  Biological knowledge                                                                            \end{cptitemize}               &
		\begin{cptitemize} \item[--]  Bad ``upper envelope'' since weights are not used for the estimation itself                    \end{cptitemize}               &
		(No)                                                                                                                                                         &
		(Yes)                                                                                                                                                         \\ \hline%comment out?

		LOESS                                                                                                                                                        &
		\begin{cptitemize} \item[--]  Local  polynomial with points closer to the estimated point are more important                  \end{cptitemize}               &
		\begin{cptitemize} \item[--]  Flexible \item[--]  Generalization of SG \item[--]  Weighting function makes intuitive sense                  \end{cptitemize} &
		\begin{cptitemize} \item[--]  Computationally expensive                                                                       \end{cptitemize}               &
		Yes                                                                                                                                                          &
		(Yes)                                                                                                                                                         \\ \hline%comment out?

		B-Splines (Smoothed)                                                                                                                                         &
		\begin{cptitemize} \item[--]  Function can be approximated by a linear combination of B-splines basis functions               \end{cptitemize}               &
		\begin{cptitemize} \item[--]  General assumption \item[--]  Flexible shape                                                            \end{cptitemize}        &
		\begin{cptitemize} \item[--]  Unbounded \item[--]  No intuitive meaning for smoothing                                                \end{cptitemize}        &
		Yes                                                                                                                                                            &
		No                                            \\ \hline%comment out?

		Smoothing Splines                                                                                                                                            &
		\begin{cptitemize} \item[--]  2cd derivative of function is integrable                                                        \end{cptitemize}               &
		\begin{cptitemize} \item[--]  Intuitive meaning of penalty \item[--]  General assumptions \item[--]  Flexible shape                         \end{cptitemize} &
		\begin{cptitemize} \item[--]  Choice of smoothing parameter                                                                                       \end{cptitemize}               &
		Yes                                                                                                                                                          &
        No \\                                                                         
% Penalized Regression Splines &
		% \begin{itemize}
		%     \item[--]  High
		%     \item[--]  Bs
		% \end{itemize} &
		% ~ &
		% ~ &
		% ~ &
		% ~ \\ \hline%comment out?

		% Whittaker                                                                                                                                                    &
		% % \parskip=0pt
		% % \begin{minipage}[t]{\linewidth}
		% % 	\begin{itemize}[nosep,after=\strut]
		% % 		\item[--] First
		% % 		\item[--] Second
		% % 	\end{itemize}
		% % \end{minipage}                                                               &
		% ~                                                                                                                                                            &
		% ~                                                                                                                                                            &
		% ~                                                                                                                                                            &
		% ~                                                                                                                                                              \\ \hline%comment out?

		% Fourier                                                                                 &
		% ~                                                                                       &
		% ~                                                                                       &
		% ~                                                                                       &
		% ~                                                                                       &
		% ~                                                                                             \\ \hline%comment out?

		% methodname &
		% ~ &
		% ~ &
		% ~ &
		% ~ &
		% ~ \\ \hline%comment out?
		\bottomrule
	\end{tabular}
	\label{table:pros_cons_overview}
\end{table}
