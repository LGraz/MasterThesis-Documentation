\chapter{Results}

\section{Goodness of Fit for Selected Interpolation Methods}{
	\label{sec:results_itpl}
	Table \ref{tab:cv-statistics_itpl-methods} benchmarks the interpolation methods (on $P^{SCL45}$) with respect to various score functions. The score functions take the absolute values of the LOOCV residuals and summarize them in a number (the smaller the better). For each of the 5 selected interpolation methods, we consider the basic version and the robustified (see section \ref{sec:loess_robustify}) version.

	\begin{table}[h]
		\begin{center}
			\caption{Comparing the goodness of fit for different interpolation methods (on $P^{SCL45}$) measured with the score functios (which take the LOOCV residuals as input) listed in the left column. $q_X$ denotes here the $X\%$ quantile.}
			\scriptsize
			\begin{tabular}{lrrrrrrrrrr}
 & ss & loess & dl & bspl & fourier & ss rob & loess rob & dl rob & bspl rob & fourier rob \\
rmse & \background-color#e1e1e1 \color#000000 0.066 & \background-color#eeeeee \color#000000 0.062 & \background-color#f1f1f1 \color#000000 0.061 & \background-color#bfbfbf \color#000000 0.077 & \background-color#eeeeee \color#000000 0.062 & \background-color#cbcbcb \color#000000 0.073 & \background-color#dbdbdb \color#000000 0.068 & \background-color#e7e7e7 \color#000000 0.064 & \background-color#afafaf \color#000000 0.082 & \background-color#000000 \color#f1f1f1 0.137 \\
qtile50 & \background-color#646464 \color#f1f1f1 0.035 & \background-color#828282 \color#f1f1f1 0.032 & \background-color#bfbfbf \color#000000 0.026 & \background-color#1d1d1d \color#f1f1f1 0.042 & \background-color#a0a0a0 \color#f1f1f1 0.029 & \background-color#8d8d8d \color#f1f1f1 0.031 & \background-color#a0a0a0 \color#f1f1f1 0.029 & \background-color#f1f1f1 \color#000000 0.021 & \background-color#646464 \color#f1f1f1 0.035 & \background-color#000000 \color#f1f1f1 0.045 \\
qtile75 & \background-color#8e8e8e \color#f1f1f1 0.061 & \background-color#a2a2a2 \color#f1f1f1 0.057 & \background-color#d7d7d7 \color#000000 0.047 & \background-color#494949 \color#f1f1f1 0.074 & \background-color#c7c7c7 \color#000000 0.050 & \background-color#9d9d9d \color#f1f1f1 0.058 & \background-color#b7b7b7 \color#000000 0.053 & \background-color#f1f1f1 \color#000000 0.042 & \background-color#6d6d6d \color#f1f1f1 0.067 & \background-color#000000 \color#f1f1f1 0.088 \\
qtile85 & \background-color#b8b8b8 \color#000000 0.078 & \background-color#c4c4c4 \color#000000 0.074 & \background-color#e5e5e5 \color#000000 0.063 & \background-color#7c7c7c \color#f1f1f1 0.098 & \background-color#d9d9d9 \color#000000 0.067 & \background-color#b8b8b8 \color#000000 0.078 & \background-color#cacaca \color#000000 0.072 & \background-color#f1f1f1 \color#000000 0.059 & \background-color#8e8e8e \color#f1f1f1 0.092 & \background-color#000000 \color#f1f1f1 0.139 \\
qtile90 & \background-color#c8c8c8 \color#000000 0.090 & \background-color#d2d2d2 \color#000000 0.085 & \background-color#dedede \color#000000 0.080 & \background-color#9b9b9b \color#f1f1f1 0.111 & \background-color#d7d7d7 \color#000000 0.083 & \background-color#b4b4b4 \color#000000 0.099 & \background-color#cfcfcf \color#000000 0.087 & \background-color#f1f1f1 \color#000000 0.071 & \background-color#969696 \color#f1f1f1 0.113 & \background-color#000000 \color#f1f1f1 0.183 \\
qtile95 & \background-color#e6e6e6 \color#000000 0.120 & \background-color#f1f1f1 \color#000000 0.110 & \background-color#efefef \color#000000 0.112 & \background-color#d8d8d8 \color#000000 0.133 & \background-color#e7e7e7 \color#000000 0.119 & \background-color#dcdcdc \color#000000 0.130 & \background-color#eaeaea \color#000000 0.117 & \background-color#e8e8e8 \color#000000 0.118 & \background-color#bcbcbc \color#000000 0.160 & \background-color#000000 \color#f1f1f1 0.335 \\
\end{tabular}

			\normalsize
			\label{tab:cv-statistics_itpl-methods}
		\end{center}
	\end{table}
	
	DL is the best among both robustified and non-robustified with respect to most of the score functions used (all except q95) and is especially superior to the other parametric approach, which is Fourier interpolation. Especially the robust Fourier interpolation performs poorly. The LOESS dominates (i.e. is superior on every score function) all other non-prametric methods, but is closely followed by the SS. The BSPL, on the other hand, is the worst non-parametric method tested here. 
}


\section{Robustification and NDVI-Correction} \label{sec:results_ndvi_corr}
\begin{table}
	\begin{center}
		\caption{XXX RMSE of yield prediction. For the relative RMSE and the coefficient of determination (R\textsuperscript{2}) see table \ref{tab:methods_vs_yieldprediction_relative} and \ref{tab:methods_vs_yieldprediction_r2}}.
		\small
		\begin{tabular}{lrrrrrrr}
\toprule
 & RF & OLS\textsuperscript{SCL} & OLS\textsuperscript{all} & MARS & GAM & LASSO & no corrections \\
\midrule
SS & {\cellcolor[HTML]{1C1C1C}} \color[HTML]{F1F1F1} 1.144 & {\cellcolor[HTML]{F1F1F1}} \color[HTML]{000000} 1.033 & {\cellcolor[HTML]{CFCFCF}} \color[HTML]{000000} 1.051 & {\cellcolor[HTML]{E0E0E0}} \color[HTML]{000000} 1.042 & {\cellcolor[HTML]{D8D8D8}} \color[HTML]{000000} 1.046 & {\cellcolor[HTML]{E0E0E0}} \color[HTML]{000000} 1.042 & {\cellcolor[HTML]{7B7B7B}} \color[HTML]{F1F1F1} 1.095 \\
SS\textsuperscript{rob} & {\cellcolor[HTML]{1C1C1C}} \color[HTML]{F1F1F1} 1.144 & {\cellcolor[HTML]{C9C9C9}} \color[HTML]{000000} 1.054 & {\cellcolor[HTML]{8F8F8F}} \color[HTML]{F1F1F1} 1.084 & {\cellcolor[HTML]{7D7D7D}} \color[HTML]{F1F1F1} 1.094 & {\cellcolor[HTML]{A6A6A6}} \color[HTML]{F1F1F1} 1.072 & {\cellcolor[HTML]{A8A8A8}} \color[HTML]{F1F1F1} 1.071 & {\cellcolor[HTML]{828282}} \color[HTML]{F1F1F1} 1.091 \\
DL & {\cellcolor[HTML]{111111}} \color[HTML]{F1F1F1} 1.150 & {\cellcolor[HTML]{545454}} \color[HTML]{F1F1F1} 1.115 & {\cellcolor[HTML]{525252}} \color[HTML]{F1F1F1} 1.116 & {\cellcolor[HTML]{525252}} \color[HTML]{F1F1F1} 1.116 & {\cellcolor[HTML]{767676}} \color[HTML]{F1F1F1} 1.097 & {\cellcolor[HTML]{747474}} \color[HTML]{F1F1F1} 1.098 & {\cellcolor[HTML]{000000}} \color[HTML]{F1F1F1} 1.159 \\
DL\textsuperscript{rob} & {\cellcolor[HTML]{000000}} \color[HTML]{F1F1F1} 1.159 & {\cellcolor[HTML]{3A3A3A}} \color[HTML]{F1F1F1} 1.128 & {\cellcolor[HTML]{505050}} \color[HTML]{F1F1F1} 1.117 & {\cellcolor[HTML]{B6B6B6}} \color[HTML]{000000} 1.064 & {\cellcolor[HTML]{7E7E7E}} \color[HTML]{F1F1F1} 1.093 & {\cellcolor[HTML]{676767}} \color[HTML]{F1F1F1} 1.105 & {\cellcolor[HTML]{060606}} \color[HTML]{F1F1F1} 1.156 \\
\bottomrule
\end{tabular}

		\label{tab:methods_vs_yieldprediction}
		\normalsize
	\end{center}
\end{table}





\begin{equation}\label{eq:corr_lm}
	\begin{aligned}		
		\widehat{\operatorname{NDVI}}_{\text{corr}}  = &
		0.711 \operatorname{NDVI}_\text{observed}  
		+ \mathbbm 1_{{SCL = 2}} 0.215 
		+ \mathbbm 1_{{SCL = 3}} 0.237 
		+ \mathbbm 1_{{SCL = 4}} 0.210 \\ &
		+ \mathbbm 1_{{SCL = 5}} 0.116 
		+ \mathbbm 1_{{SCL = 6}} 0.162 
		+ \mathbbm 1_{{SCL = 7}} 0.327 
		+ \mathbbm 1_{{SCL = 8}} 0.474 \\ &
		+ \mathbbm 1_{{SCL = 9}} 0.575 
		+ \mathbbm 1_{{SCL = 10}} 0.306 
		+ \mathbbm 1_{{SCL = 11}} 0.512 
	\end{aligned}
\end{equation}
- strong upwards correction for SCL classes 8, 9 and 11 (correspond to `medium probability clouds', `high probability clouds' and `thin cirrus clouds').

\begin{equation}\label{eq:corr_lm_res}
	\begin{aligned}		
		\widehat{\operatorname{abs}}\left({\operatorname{NDVI}_\text{``true''} - \operatorname{NDVI}}_{\text{corr}}\right)  = &
		-0.133 \operatorname{NDVI}_\text{observed}  
		+ \mathbbm 1_{{SCL = 2}} 0.186 
		+ \mathbbm 1_{{SCL = 3}} 0.185 \\ &
		+ \mathbbm 1_{{SCL = 4}} 0.146 
		+ \mathbbm 1_{{SCL = 5}} 0.089 
		+ \mathbbm 1_{{SCL = 6}} 0.167 \\ &
		+ \mathbbm 1_{{SCL = 7}} 0.203 
		+ \mathbbm 1_{{SCL = 8}} 0.181  
		+ \mathbbm 1_{{SCL = 9}} 0.173 \\ &
		+ \mathbbm 1_{{SCL = 10}} 0.180 
		+ \mathbbm 1_{{SCL = 11}} 0.172 
	\end{aligned}
\end{equation}
- the higher the observed NDVI the lower the estimated absolute residual.
- estimated absolute resiudals are the smalles for SCL classes 4 and 5.  
