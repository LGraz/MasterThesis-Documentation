\chapter{Results}\label{sec:results}

\section{Goodness of Fit for Selected {{IM}}s}{
	\label{sec:results_itpl}
	The benchmarks of the in section~\ref{sec:itpl_preselection} selected {{IM}}s (considering only SCL45 observations) with respect to various score functions are displayed in table~\ref{tab:cv-statistics_itpl-methods}. The score functions summarize the absolute values of the LOOCV residuals (the smaller, the better). For each of the 5 selected {{IM}}s, we consider the basic and the robustified (see section~\ref{sec:loess_robustify}) version.

	\begin{table}[h]
		\begin{center}
			\caption[Goodness of fit for {{IM}}s  measured with score functions.]{Comparing the goodness of fit for selected {{IM}}s --- measured with score functions (see section~\ref{sec:scorefun}) that take the LOOCV residuals as input. Colored row-wise.}
			\scriptsize
			\begin{tabular}{lrrrrrrrrrr}
 & ss & loess & dl & bspl & fourier & ss rob & loess rob & dl rob & bspl rob & fourier rob \\
rmse & \background-color#e1e1e1 \color#000000 0.066 & \background-color#eeeeee \color#000000 0.062 & \background-color#f1f1f1 \color#000000 0.061 & \background-color#bfbfbf \color#000000 0.077 & \background-color#eeeeee \color#000000 0.062 & \background-color#cbcbcb \color#000000 0.073 & \background-color#dbdbdb \color#000000 0.068 & \background-color#e7e7e7 \color#000000 0.064 & \background-color#afafaf \color#000000 0.082 & \background-color#000000 \color#f1f1f1 0.137 \\
qtile50 & \background-color#646464 \color#f1f1f1 0.035 & \background-color#828282 \color#f1f1f1 0.032 & \background-color#bfbfbf \color#000000 0.026 & \background-color#1d1d1d \color#f1f1f1 0.042 & \background-color#a0a0a0 \color#f1f1f1 0.029 & \background-color#8d8d8d \color#f1f1f1 0.031 & \background-color#a0a0a0 \color#f1f1f1 0.029 & \background-color#f1f1f1 \color#000000 0.021 & \background-color#646464 \color#f1f1f1 0.035 & \background-color#000000 \color#f1f1f1 0.045 \\
qtile75 & \background-color#8e8e8e \color#f1f1f1 0.061 & \background-color#a2a2a2 \color#f1f1f1 0.057 & \background-color#d7d7d7 \color#000000 0.047 & \background-color#494949 \color#f1f1f1 0.074 & \background-color#c7c7c7 \color#000000 0.050 & \background-color#9d9d9d \color#f1f1f1 0.058 & \background-color#b7b7b7 \color#000000 0.053 & \background-color#f1f1f1 \color#000000 0.042 & \background-color#6d6d6d \color#f1f1f1 0.067 & \background-color#000000 \color#f1f1f1 0.088 \\
qtile85 & \background-color#b8b8b8 \color#000000 0.078 & \background-color#c4c4c4 \color#000000 0.074 & \background-color#e5e5e5 \color#000000 0.063 & \background-color#7c7c7c \color#f1f1f1 0.098 & \background-color#d9d9d9 \color#000000 0.067 & \background-color#b8b8b8 \color#000000 0.078 & \background-color#cacaca \color#000000 0.072 & \background-color#f1f1f1 \color#000000 0.059 & \background-color#8e8e8e \color#f1f1f1 0.092 & \background-color#000000 \color#f1f1f1 0.139 \\
qtile90 & \background-color#c8c8c8 \color#000000 0.090 & \background-color#d2d2d2 \color#000000 0.085 & \background-color#dedede \color#000000 0.080 & \background-color#9b9b9b \color#f1f1f1 0.111 & \background-color#d7d7d7 \color#000000 0.083 & \background-color#b4b4b4 \color#000000 0.099 & \background-color#cfcfcf \color#000000 0.087 & \background-color#f1f1f1 \color#000000 0.071 & \background-color#969696 \color#f1f1f1 0.113 & \background-color#000000 \color#f1f1f1 0.183 \\
qtile95 & \background-color#e6e6e6 \color#000000 0.120 & \background-color#f1f1f1 \color#000000 0.110 & \background-color#efefef \color#000000 0.112 & \background-color#d8d8d8 \color#000000 0.133 & \background-color#e7e7e7 \color#000000 0.119 & \background-color#dcdcdc \color#000000 0.130 & \background-color#eaeaea \color#000000 0.117 & \background-color#e8e8e8 \color#000000 0.118 & \background-color#bcbcbc \color#000000 0.160 & \background-color#000000 \color#f1f1f1 0.335 \\
\end{tabular}

			\normalsize
			\label{tab:cv-statistics_itpl-methods}
		\end{center}
	\end{table}
	
	DL performs the best among both robustified and non-robustified with respect to most of the score functions used (all except QAR\textsuperscript{95}) and is in particular superior to the other parametric approach, which is FS. Especially the robust FS performs poorly. The LOESS is superior than all other non-parametric methods with respect to every score function. However, it is closely followed by the SS. The BS exhibits the worst performance out of all non-parametric method tested here. 
}


\section{Yield Prediction Error for Tested ISs} \label{sec:results_ndvi_corr}
	\begin{table}
		\begin{center}
			\caption[Relative Yield Estimation Error for ISs]{Relative YPE for various ISs. For the non-relative YPE and the coefficient of determination (R\textsuperscript{2}) cf. table~\ref{tab:methods_vs_yieldprediction} and~\ref{tab:methods_vs_yieldprediction_r2}.}
			\small
			\begin{tabular}{lrrrrrrr}
\toprule
 & RF & OLS\textsuperscript{SCL} & OLS\textsuperscript{all} & MARS & GAM & LASSO & no corrections \\
\midrule
SS & {\cellcolor[HTML]{282828}} \color[HTML]{F1F1F1} 0.155 & {\cellcolor[HTML]{F1F1F1}} \color[HTML]{000000} 0.140 & {\cellcolor[HTML]{C9C9C9}} \color[HTML]{000000} 0.143 & {\cellcolor[HTML]{D6D6D6}} \color[HTML]{000000} 0.142 & {\cellcolor[HTML]{D6D6D6}} \color[HTML]{000000} 0.142 & {\cellcolor[HTML]{D6D6D6}} \color[HTML]{000000} 0.142 & {\cellcolor[HTML]{797979}} \color[HTML]{F1F1F1} 0.149 \\
SS\textsuperscript{rob} & {\cellcolor[HTML]{282828}} \color[HTML]{F1F1F1} 0.155 & {\cellcolor[HTML]{C9C9C9}} \color[HTML]{000000} 0.143 & {\cellcolor[HTML]{939393}} \color[HTML]{F1F1F1} 0.147 & {\cellcolor[HTML]{797979}} \color[HTML]{F1F1F1} 0.149 & {\cellcolor[HTML]{A0A0A0}} \color[HTML]{F1F1F1} 0.146 & {\cellcolor[HTML]{AEAEAE}} \color[HTML]{000000} 0.145 & {\cellcolor[HTML]{868686}} \color[HTML]{F1F1F1} 0.148 \\
DL & {\cellcolor[HTML]{1A1A1A}} \color[HTML]{F1F1F1} 0.156 & {\cellcolor[HTML]{5D5D5D}} \color[HTML]{F1F1F1} 0.151 & {\cellcolor[HTML]{505050}} \color[HTML]{F1F1F1} 0.152 & {\cellcolor[HTML]{505050}} \color[HTML]{F1F1F1} 0.152 & {\cellcolor[HTML]{797979}} \color[HTML]{F1F1F1} 0.149 & {\cellcolor[HTML]{797979}} \color[HTML]{F1F1F1} 0.149 & {\cellcolor[HTML]{000000}} \color[HTML]{F1F1F1} 0.158 \\
DL\textsuperscript{rob} & {\cellcolor[HTML]{0D0D0D}} \color[HTML]{F1F1F1} 0.157 & {\cellcolor[HTML]{434343}} \color[HTML]{F1F1F1} 0.153 & {\cellcolor[HTML]{505050}} \color[HTML]{F1F1F1} 0.152 & {\cellcolor[HTML]{AEAEAE}} \color[HTML]{000000} 0.145 & {\cellcolor[HTML]{868686}} \color[HTML]{F1F1F1} 0.148 & {\cellcolor[HTML]{6B6B6B}} \color[HTML]{F1F1F1} 0.150 & {\cellcolor[HTML]{0D0D0D}} \color[HTML]{F1F1F1} 0.157 \\
\bottomrule
\end{tabular}

			\label{tab:methods_vs_yieldprediction_relative}
			\normalsize
		\end{center}
	\end{table}
	The YPE for the in section~\ref{sec:corr_itpl_stat} chosen {{IS}}s is given in table~\ref{tab:methods_vs_yieldprediction_relative}. We note that robustification does not improve the quality of the fit (measured via the YPE) in most cases. 
	In terms of YPE, SS tend to be better than DL (with and without robustification), especially if no correction is made. The {{IS}} that leads to the lowest YPE is the OLS\textsuperscript{SCL} with SS. Note that it is the simplest model which performs best.Given that the OLS\textsuperscript{SCL} models have very good interpretability, we also present the regression equations below. The corrected NDVI is calculated using 
	\begin{equation}\label{eq:corr_lm}
		\begin{aligned}		
			{\widehat{\operatorname{NDVI}}}_{\text{true}}  = &
			0.711 \operatorname{NDVI}_\text{observed}  
			+ 0.215 \,\mathbbm 1_{{SCL = 2}} 
			+ 0.237 \,\mathbbm 1_{{SCL = 3}} 
			+ 0.210 \,\mathbbm 1_{{SCL = 4}} \\ &
			+ 0.116 \,\mathbbm 1_{{SCL = 5}} 
			+ 0.162 \,\mathbbm 1_{{SCL = 6}} 
			+ 0.327 \,\mathbbm 1_{{SCL = 7}} 
			+ 0.474 \,\mathbbm 1_{{SCL = 8}} \\ &
			+ 0.575 \,\mathbbm 1_{{SCL = 9}} 
			+ 0.306 \,\mathbbm 1_{{SCL = 10}} 
			+ 0.512 \,\mathbbm 1_{{SCL = 11}} 
		\end{aligned}
	\end{equation} 
	where $\mathbbm 1_{{SCL = 2}}$ is equal to one if the current observation corresponds to SCL class 2 and zero otherwise\footnote{$\mathbbm 1$ is also called an indicator function or characteristic function in mathematics.}. Whereas, we obtain the estimated absolute residuals (cf. equation~\refeq{eq:absndvires}) by: 
	\begin{equation}\label{eq:corr_lm_res}
		\begin{aligned}		
			\hat v  = &
			-0.133 \operatorname{NDVI}_\text{observed}  
			+ 0.186 \, \mathbbm 1_{{SCL = 2}} 
			+ 0.185 \, \mathbbm 1_{{SCL = 3}} 
			+ 0.146 \, \mathbbm 1_{{SCL = 4}} \\ &
			+ 0.089 \, \mathbbm 1_{{SCL = 5}} 
			+ 0.167 \, \mathbbm 1_{{SCL = 6}} 
			+ 0.203 \, \mathbbm 1_{{SCL = 7}} 
			+ 0.181 \, \mathbbm 1_{{SCL = 8}} \\ & 
			+ 0.173 \, \mathbbm 1_{{SCL = 9}} 
			+ 0.180 \, \mathbbm 1_{{SCL = 10}} 
			+ 0.172 \, \mathbbm 1_{{SCL = 11}} 
		\end{aligned}
	\end{equation} 
	Thus, if we observe a pixel with SCL class 4 (`vegetation') but a NDVI of only 0.4, the corrected NDVI would be $0.711 \cdot 0.4 + 0.21 = 0.494$ with an estimated absolute residual of $-0.133 \cdot 0.4 + 0.146 = 0.093$.
	In equation~\refeq{eq:corr_lm}, we notice the strongest upwards correction for SCL classes 8, 9 and 11 (`medium probability clouds', `high probability clouds' and `thin cirrus clouds'). The estimated absolute residuals, however, are the smallest for SCL classes 4 and 5 (`vegetation' and `bare soil'). Furthermore, the higher the observed NDVI the lower are the estimated absolute residuals.
	For the \texttt{R}-output of the \texttt{summary} function of the two models, we refer to the appendix~\ref{app:ols-scl-summary}. 


	







