\chapter{Introduction}


Satelite image time series are used in ...
    The European Space Agency makes the images from the Sentinel 2 satelites freely avialable
        Extracting indicies time series (like NDVI) and used to model ... (not only of interest to researchers but also public agents and insurance companies)
                        - Plant Models REF
                        - Season Start (start of spring) (community name: land-surface-plant-phenology)
                        - Yield prediction
                        - crop classification
            erronous observations --> convervative (SCL) filtration 
                --> Data gaps
                    currently done: interpolation and smoothing techniques
%%%%%%%%%%%%%%%%%%%%%%%%%%%%%%%%%%%%%%%%%%%%%%%%%%%%%%%%%%%%%%%%%%%%%%%%%%%%%%%%%%%
                    we give an overview + review over popular interpolation methods + discuss how data gaps influence the given methods + discuss approach of robustifing against outliers
                    Select suitable ones in our NDVI setting --> benchmark
                Try to eliminate data gaps by not using strong SCL-filtration but weighting. Weighting comes from an uncertainty estimation done by a statistical model (develope a proxy for the true NDVI) <-- we tried various <-- evalute different IS's with objective defined quality measure (which relies on the assumption that a NDVI TS which better models the plant growth is more suitable for predicting yield)
            


Research Questions
\begin{Nenumerate}
    \item 1 review of interpolation methods
    \item 2 errornous observations --- how to deal with them
    \item 3 data gaps --- influence itpl mehtods 
    \item 4 data gaps --- how to deal with them
    % \item 5 develope own strategy of interpolation
    \item 6 how to compare two NDVI interpolations?
\end{Nenumerate}
1 in \ref{sec:itpl}
2 robustification \ref{sec:loess_robustify}
3 discussed in \ref{sec:discussion_itpl_data_gaps}
4 utilize observations filterd before and estimating how reliable each of them are \ref{sec:corr}
6 \ref{sec:ndvi_corr_eval}

% Roadmap



%smoothing
``Similarly, smoothing the time series of satellite data is helpful to address inconsistency in observation frequency and timing due to clouds and other sensor artefacts \cite{skakunWinterWheatYield2019}''


\todo[inline]{Why do we do interpolation in NDVI (and other indices) time series? What are possible shortcomings thereof?}

\begin{itemize}
    \item Doublelogistic (winter-ndvi)
    \item parametric / non-parametric approaches
    \item spatio-temporal approaches
\end{itemize}









%%% Local Variables: 
%%% mode: latex
%%% TeX-master: "MasterThesisSfS"
%%% End: 
