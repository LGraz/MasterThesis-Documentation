\chapter{Introduction}

Remote sensing aims to measure target variables efficiently from a distance. 
% stakeholders + applications  
Large scale monitoring of forest and agricultural vegetation dynamics is of great interest to authorities, insurance companies and research. Examples include crop classification for subsidizing farmers \citep{henitsSentinel2EnablesNationwide2022} and the creation of crop models for estimating crop yields or nitrogen concentrations \citep{couraultSTICSCropModel2021,perichCropNitrogenRetrieval2021}. 
% Season Start (start of spring) (community name: land surface plant phenology)
For this, freely distributed multi-spectral satellite imagery from the  Sentinel-2 (S2) satellites are examined \citep{esaSentinel22022}.
% NDVI
In order to transform the high dimensional satellite images into easily interpretable metrics, spectral indices such as the Normalized Difference Vegetation Index (NDVI) are used \citep{rouseMonitoringVernalAdvancement1974}. The NDVI serves as a proxy for photosynthetic activity \citep{gamonRelationshipsNDVICanopy1995a}, and thus the corresponding {NDVI Time Series ({TS})} reflects the vegetation development. 
% S2 issues (clouds ...)
The quality of a satellite image, however, depends on atmospheric conditions and thus in case of a dense cloud cover, the information content derived from the NDVI is impaired. Therefore, \cite{esaEuropeanSpaceAgency2022} also provides a Scene Classification Layer (SCL), which provides additional metadata about what is observed (e.g., shadows, clouds, vegetation, etc.) . So when extracting the NDVI {TS} from the Sentinel 2 satellite imagery {TS}, we can filter out the contaminated observations using the SCL classification. However, due to this filtration it may occur that we have no observations for several weeks, especially in winter. It is also possible that some observations are wrongly classified by the SCL (e.g., as vegetation) and thus result in an erroneous NDVI causing an outlier in the TS. Consequently, the main challenge is to interpolate an NDVI {TS}, which can contain both large data gaps and outliers. 

% state-of-the-art
Currently, there are several approaches to address these issues. One is to look at the observed evolution of the canopy coverage and assume its bell shape for the NDVI {TS} given the strong correlation between NDVI and photosynthetic activity. Approaches to model this include a \nth{2} order Fourier approximation \citep{stockliEuropeanPlantPhenology2004} or a Double Logistic function \citep{beckImprovedMonitoringVegetation2006}.
On the other hand, assumptions are made about more abstract properties of the curve, such as smoothness. We divide these into local and global approaches. Nadaraya-Watson \citep{strbacEstimationEvapotrasnpirationUrban2017}, Savitzky-Golay Filter \citep{chenSimpleMethodReconstructing2004a} and Locally Reweighed Regression \citep{omoriAssessmentPaddyFields2021} use a sliding window to interpolate the {TS} stepwise. Global methods like B-Splines \citep{gurungPredictingEnhancedVegetation2009} and Smoothing Splines \citep{caiPerformanceSmoothingMethods2017} reduce the squares of all residuals simultaneously, and Universal Kriging fits a Gaussian process to the data \citep{chandolaScalableTimeSeries2010}.
% SS defined in \cite{cravenSmoothingNoisyData1978}

\bigskip
The research questions pursued in this thesis are:
\begin{Nenumerate}
    \item Which {{IM}}s are used in the context of NDVI, and what are their advantages and disadvantages?
    \item How may contaminated data be dealt with?
    \item How do data gaps affect interpolation?
    \item How to deal with data gaps?
    \item How can we recognize a good interpolation of the NDVI?
\end{Nenumerate}
\bigskip



% our contribution + roadmap:
In this thesis, we will discuss the strengths and weaknesses of Interpolation Methods ({{IM}}s) and evaluate them with respect to NDVI interpolation. For this purpose, we use the Sentinel 2 satellite image {TS} and crop yield maps of different fields of different cereal species on a farm in Witzwil, Switzerland over the years 2017-2021. After presenting the available data, illustrating challenges and defining different concepts in chapter~\ref{sec:data_methods} (\nameref{sec:data_methods}), we turn to the two main blocks of this thesis. One covers the study of IMs and the other presents a general procedure of correcting (NDVI) TS with uncertainty estimation by utilizing additional information.
On the first block, in chapter~\ref{sec:itpl} (\nameref{sec:itpl}) we examine parametric and non-parametric {{IM}}s and discuss their strengths and weaknesses (question i.). We generalize and test an iterative technique that makes IMs more robust to outliers by weighting them less (question ii.). To evaluate IMs, we present an approach that uses out-of-bag residuals (question v.). In section~\ref{sec:discussion_itpl_data_gaps} (\nameref{sec:discussion_itpl_data_gaps}), we discuss how different {{IM}}s respond to data gaps (question iii.), and in section~\ref{sec:itpl_preselection} (\nameref{sec:itpl_preselection}) we preselect {{IM}}s. We evaluate this preselection in the results section~\ref{sec:results_itpl} (\nameref{sec:results_itpl}) and select two candidates from different {{IM}}s in section~\ref{sec:itpl_candiate_selection} (\nameref{sec:itpl_candiate_selection}).
For the second block, we correct possibly contaminated data with statistical models in chapter~\ref{sec:corr} (\nameref{sec:corr}) (question ii.) and utilize previously ignored observations, which we hope will further reduce data gaps (question iv.). Thus, we no longer filter the observations a priori via the SCL, but instead correct the observed NDVI and weight the observations via estimated uncertainties. By combining different statistical models and IMs, we get 28 Interpolation Strategies ({{ISs}}). We compare those with a vegetation-oriented quality measure (question v.) and describe the results in section~\ref{sec:results_ndvi_corr} (\nameref{sec:results_ndvi_corr}). Based on these results, in section~\ref{sec:discussion_corr} (\nameref{sec:discussion_corr}) we argue what the best {{IS}} is. In addition, we justify why our NDVI correction can be understood as unsupervised learning and why we relied only on satellite imagery and not on meteorological data for the NDVI correction.
Our conclusions of this thesis, recommendations, as well as an outlook on future work is given in chapter~\ref{sec:Conclusion} (\nameref{sec:Conclusion}). 


% und Definieren verschiedene Konzepte. Darunter eine transformation der Zeitachse, welche datenlücken im Winter zusammenschrumpfen lässt (Frage iv.). 




%smoothing:   ``Similarly, smoothing the {TS} of satellite data is helpful to address inconsistency in observation frequency and timing due to clouds and other sensor artefacts \cite{skakunWinterWheatYield2019}''


%%% Local Variables: 
%%% mode: latex
%%% TeX-master: "MasterThesisSfS"
%%% End: 
