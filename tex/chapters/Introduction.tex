\chapter{Introduction}

Remote sensing aims to measure target variables efficiently from a distance. In this context, satellite imagery Time Series ({TS}) such as the imagery {TS} of the multi-spectral Sentinel 2 satellites freely distributed by the European Space Agency are used.
% stakeholders + applications  
Large scale monitoring of forest and agricultural vegetation dynamics is of great interest to authorities, insurance companies and environmental and climate researchers. Examples include crop classification for subsidizing farmers and the creation of crop models for estimating crop yields or nitrogen concentrations. 
    % Season Start (start of spring) (community name: land surface plant phenology)
% NDVI
In order to transform the high dimensional satellite images into easily interpretable metrics, spectral indices such as the Normalized Difference Vegetation Index (NDVI) are used. The NDVI serves as a proxy for vegetation density, and the corresponding {TS} reflects the vegetation development. 
% S2 issues (clouds ...)
The quality of a satellite image however depends on atmospheric conditions and thus in case of a dense cloud cover the information content derived from the NDVI is impaired. Therefore, the European Space Agency also provides a Scene Classification Layer (SCL), which provides additional metadata about what is observed (e.g., shadows, clouds, vegetation, etc.). So when extracting the NDVI {TS} from the Sentinel 2 satellite imagery {TS}, we can filter out the corrupted observations using the SCL classification. However, due to this filtration it may occur that we have no observations for several weeks, especially in winter, or that some observations are wrongly classified by the SCL (e.g., as vegetation) and thus result in an erroneous NDVI. Consequently, the main challenge is to interpolate an NDVI {TS}, which can contain both large data gaps and outliers. 

% state-of-the-art
There are several approaches to adequately address this issue. One is to look at the observed evolution of vegetation density and assume its bell shape for the NDVI {TS} given the strong correlation between NDVI and vegetation density. Approaches to model this include a 2nd order Fourier approximation \citep{stockliEuropeanPlantPhenology2004} or a Double Logistic function \citep{beckImprovedMonitoringVegetation2006}.
On the other hand, assumptions are made about more abstract properties of the curve, such as smoothness or the like. We divide these into local and global approaches. Nadaraya-Watson \citep{strbacEstimationEvapotrasnpirationUrban2017}, Savitzky-Golay Filter \citep{chenSimpleMethodReconstructing2004a} and Locally Reweighted Regression \citep{omoriAssessmentPaddyFields2021} use a sliding window to interpolate the {TS} stepwise. Global methods like B-Splines \citep{gurungPredictingEnhancedVegetation2009} and Smoothing Splines \citep{caiPerformanceSmoothingMethods2017} reduce the squares of all residuals simultaneously, and Universal Kriging fits a Gaussian process to the data \citep{chandolaScalableTimeSeries2010}.


% SS defined in \cite{cravenSmoothingNoisyData1978}


% my contribuition:
In this thesis, we will discuss strengths and weaknesses of these interpolation methods and evaluate them with respect to NDVI interpolation. For this purpose, we use the Sentinel 2 satellite image {TS} and crop yield maps of different fields of different wheat species on a farm in Witzwil, Switzerland over the years 2017-2021.
To improve interpolation methods, we generalize and test an iterative technique that makes interpolations more robust to outliers by weighting them less. Additionally, we determine how data gaps affect the different interpolation methods. Furthermore, using NDVI as an example, we present a general interpolation procedure that utilizes additional information to correct the target variable with an uncertainty estimate and then interpolates. Thus, we no longer have to filter the observations a priori via the SCL, but instead correct the observed NDVI and weight the observations via the estimated uncertainties. Combining interpolation methods with the NDVI correcting models produces 28 interpolation strategies. We benchmark these against an objective quality measure, which assumes that the better an NDVI TS models crop growth, the more appropriate it is for estimating crop yield.

Die Hauptaufgaben, welchen wir in dieser Thesis nachgehen wollen lauten also:
\begin{Nenumerate}
    \item 1 rezensiere und bewerte verwendete Interpolationsmethoden
    \item 2 Wie kann mit fehlerhaften daten umgegangen werden?
    \item 3 Wie beinflussen datenlücken die Interpolation?
    \item 4 Wie soll man mit datenlücken umgehen?
    \item 5 Wie kann man Intepolationsstrategien vergleichen?
\end{Nenumerate}

Roadmap ...
% 1 in \ref{sec:itpl}
% 2 robustification \ref{sec:loess_robustify}
% 3 discussed in \ref{sec:discussion_itpl_data_gaps}
% 4 utilize observations filterd before and estimating how reliable each of them are \ref{sec:corr}
% 5 \ref{sec:ndvi_corr_eval}


%smoothing:   ``Similarly, smoothing the {TS} of satellite data is helpful to address inconsistency in observation frequency and timing due to clouds and other sensor artefacts \cite{skakunWinterWheatYield2019}''


%%%%%%%%%%%%%%%%%%%%%%%%%%%%%%%%%%%%%%%%%%%%%%%%%%%%%%%%%%%%%
%%%    G E R M A N 
%%%%%%%%%%%%%%%%%%%%%%%%%%%%%%%%%%%%%%%%%%%%%%%%%%%%%%%%%%%%%
% Remote sensing zielt darauf ab, ziel-Grössen effizient aus der Entfernung messen zu können. Hier finden Satellitenbilder Zeitreihen Verwendung, wie etwa die von der europäischen Raum-Agentur (ESA) kostenlos veröffentlichten Bilder Zeitreihen der Multi-spektralen Sentinel 2 (S2) Satelliten.
% % stakeholders + applications  
% Die Vegetationsentwicklung von Wäldern und landwirtschaftlich relevanten Flächen im grossen Stiele zu überwachen, ist unter anderem für public angents, Versicherungen, Umwelt- und Klimaforscher von grossem Interesse. Mögliche Ziele sind hierbei eine crop Klassifizierung für das Subventionieren von Bauern oder das Erstellen von Pflanzenmodellen, um Ernteertrag oder Stickstoffkonzentration zu schätzen. 
%     % Season Start (start of spring) (community name: land surface plant phenology)
% % NDVI
% Um die hochdimensionalen Satellitenbilder in leicht interpretierbare Grössen zu transformieren, werden spektrale Indizes, wie der Normalized Difference Vegetation Index (NDVI) benutzt. Dieser ist ein Proxy für die Vegetationsdichte und die korrespondierende Zeitreihe spiegelt somit die Vegetationsentwicklung wider. 
% % S2 issues (clouds ...)
% Der Informationsgehalt von einem Satellitenbild ist jedoch abhängig vom Zustand der Atmosphäre und so trägt der davon abgeleitete NDVI bei einer dichten Wolkendecke keine Informationen über die Vegetation am Boden. Daher liefert die ESA zusätzlich eine Scene Classification Layer (SCL), welche Aufschluss gibt, was beobachtet wird (z.B Schatten, Wolken, Vegetation, etc.). So können wir bei der Extraktion der NDVI Zeitreihe aus der S2 Satellitenbilder Zeitreihe, anhand der SCL Klassifizierung, die uninformativen Beobachtungen herausfiltern. Durch diese Filtration kann es jedoch vorkommen, dass wir besonders im Winter über mehrere Wochen keine Observationen haben, oder dass manche Beobachtungen zu Unrecht durch die SCL als informativ bewertet wird (z.B. als Vegetation) und somit in einem fehlerhaften NDVI resultiert. 
% Die heaupt herausforderungn sind also eine NDVI zeitreihe zu interpolieren, welche sowohl große datenlücken als auch ausreißer beinhalten kann. 

% % state-of-the-art
% Es giebt verschiedene Herangehensweisen diesem Notstand adequat zu begegnen. Zum hat man sich an der beobachteten Entwicklung der Vegetationsdichte orientiet und auf grund der starken korrelation zwischen NDVI und vegetationsdichte eine Glockenform der NDVI Zeitreihe angenommen. Ansätze diese zu modellieren beinhalten eine Fourier Approximation 2ter Ordnung REF oder einer Double Logistic funktion REF. Zum anderem trifft man annahmen über abstraktere eigenschaften der kurve. Diese unterteilein wir in Lokale und Globale ansätze. NW, SG und LOESS verwenden ein sliding Fenster um stückweise die Zeitreihe zu interpolieren. Globale methoden wie BS und SS verwenden reduzieren die quadrate aller residuen simultan und GK fitted ein Gaussaischen Process zu den daten.
    % Diese beiden Probleme versucht man gegenwärtig mit Interpolation und Smoothing zu lösen. 
    % Parametrische Lösungsansätze beinhalten die approximation durch eine Fourierreihe \citep{stockliEuropeanPlantPhenology2004} oder eine Double Logistic funktion, welche starke annahmen über die Form des NDVI trifft \citep{beckImprovedMonitoringVegetation2006}.
    % Nicht-parametrische methoden sind hingegen flexibler. Darunter zählt der Savitzky-Golay Filter \citep{chenSimpleMethodReconstructing2004a} und seine verallgemeinerung die Locally weighted Regression \citep{omoriAssessmentPaddyFields2021} 
    % der Nadaraya-Watson schätzer \citep{strbacEstimationEvapotrasnpirationUrban2017} 
    % sowie das Universal Kriging (auch bekannt als Gaussian Process Regression) \citep{chandolaScalableTimeSeries2010} 
    % Das spline fitting wird mit den B-splines \citep{gurungPredictingEnhancedVegetation2009} und den Smoothing Splines \citep{caiPerformanceSmoothingMethods2017}

% my contribuition:
% In dieser Thesis werden wir stärken und schwächen von diesen gängigen Interpolationsmethoden diskutieren und hinsichtlich der NDVI Interpolation bewerten. Dafür benutzen wir die S2 Satellitenbilder Zeitreihe und Ernteertragskarten von verschiedenen Feldern verschiedenen Weizenarten auf einer Farm in Witzwil in der Schweiz über die Jahre 2017-2021.
% Um die Interpolationmethoden zu verbessern, verallgemeinern und testen wir eine iterativen Technik, die Interpolationen robuster gegen Ausreisser machen soll, indem sie weniger Gewicht bekommen. Zudem ermitteln wir, wie Datenlücken die verschiedenen Interpolationmethoden beeinflussen. Ausserdem stellen wir am Beispiel des NDVI eine generelle Interpolations-prozedur vor, welche anhand von zusätzlichen Informationen die Zielvariable mit einer Unsicherheitsschätzung korrigiert und anschliessend interpoliert. Somit müssen wir die Observationen nicht mehr a priori via der SCL filtern \textcolor{gray}, sondern korrigieren den beobachteten NDVI und gewichten die Beobachtungen via der geschätzten Unsicherheiten. Durch die Kombination von Interpolationsmethoden mit den NDVI korrigierenden Modellen ergebn sich somit 28 Intepolationsstrategien. Diese benchmarken wir mit einem objektiven Qualitätsmass, welches annimmt, dass je besser eine NDVI TS das Pflanzenwachstum modelliert, desto geeigneter ist sie, um den Ernteertrag zu schätzen. 


%%% Local Variables: 
%%% mode: latex
%%% TeX-master: "MasterThesisSfS"
%%% End: 
