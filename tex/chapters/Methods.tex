\chapter{Interpolation Methods}
12345678912345678912345678123456789123456789123456789123456789123456789123456789
\begin{my_pros_cons_table}{
        \item item 1
        \item item 2
    }{
        \item 1
        \item 2
    }
\end{my_pros_cons_table}

123456789123456789123456789123456789123456789123456789123456789123456789123456789

\section{Setting}

We are given data in the form of $\left(x_{i}, Y_{i}\right)$ for $i=1, \ldots, n)$. Assume that it can be represented by
$$
    Y_{i}=m\left(x_{i}\right)+\varepsilon_{i},
$$
where $\varepsilon_i$ is some noise and $m: \mathbb{R} \rightarrow \mathbb{R}$ being some (non-parametric regression) function. If we assume that $\varepsilon_{1}, \ldots, \varepsilon_{n}$ i.i.d. with $\mathbb{E}\left[\varepsilon_{i}\right]=0$ then $$m(x)=\mathbb{E}[Y \mid x]$$
Different assumptions on $m$ will lead to the following models:

\section{Overview}

\section{Methods - Description}

\subsection{Kernel Regression}
\label{sec:Kernel}
As described previously, we would like to estimate
\begin{equation}
    \label{eq:nadaraya}
    \mathbb{E}[Y \mid X=x]
    = \int_{\mathbb{R}} y f_{Y \mid X}(y \mid x) d y
    =\frac{\int_{\mathbb{R}} y f_{X, Y}(x, y) d y}{f_{X}(x)},
\end{equation}
where $f_{Y \mid X}, f_{X, Y}, f_{X}$ denote the conditional, joint and marginal densities.
This can be done with a kernel $K$:
$$
    \hat{f}_{X}(x)=\frac{\sum_{i=1}^{n} K\left(\frac{x-x_{i}}{h}\right)}{n h}, \hat{f}_{X, Y}(x, y)=\frac{\sum_{i=1}^{n} K\left(\frac{x-x_{i}}{h}\right) K\left(\frac{y-Y_{i}}{h}\right)}{n h^{2}}
$$
By plugging the above into equation \ref{eq:nadaraya} we arrive at the \textit{Nadaraya-Watson} kernel estimator:
$$\hat{m}(x)=\frac{\sum_{i=1}^{n} K\left(\left(x-x_{i}\right) / h\right) Y_{i}}{\sum_{i=1}^{n} K\left(\left(x-x_{i}\right) / h\right)}$$


\begin{my_pros_cons_table}{
        \item can be assigned degrees of freedom (trace of the hat-matrix)
        \item estimation of the noise variance $\hat \sigma_\varepsilon^2$ (XXX c.f. CompStat 3.2.2)
    }{
        \item choice of kernel
        \item if the $x \mapsto K(x)$ is not continuous, $\hat m $ isn't either
        \item choice of bandwidth, especially if $x_i$ are not equidistant.
    }
\end{my_pros_cons_table}


**Examples:**
Normal, Box
For local bandwidth selection see Brockmann et al. (1993) XXX


\subsection{loess}
\label{sec:loess}


\subsection{Savitzky-Golay Filter}
\label{sec:Savitzky–Golay}
The \textit{Savitzky-Golay Filter}, introduced in \cite{savitzkySmoothingDifferentiationData1964} is a technique in signal processing and can be used to filter out high frequencies (low-pass filter) as argued in \cite{schaferWhatSavitzkyGolayFilter2011}. Furthermore, it also can be used for smoothing by filtering high frequency noise while keeping the low frequency signal.
First we choose a window size $m$. Then, for each point $j \in \{m, m+1, \dots, n-m\}$ we fit a polynomial of degree $k$ by:
$$\hat y_j=\min_{p\in P_k}\sum_{i=-m}^{m}(p (x_{j+i})-y_{i+j})^{2},$$
were $P_k$ denotes the Polynomials of degree $k$ over $\mathbb R$.

For equidistant points this can efficiently be calculated by
$$
    \hat y_{j}=\sum_{i=-m}^{m} c_{i} y_{j+i},
$$
where the $c_i$ are only dependent on the $m$ and $k$ and are tabulated in the original paper.
\begin{my_pros_cons_table}{
        \item popular technique in signal processing
        \item efficient calculation for equidistant points
    }{
        \item no natural way of how to estimate points which are not in the data. XXX

    }
\end{my_pros_cons_table}

\subsubsection*{Interpolation and Smoothing}
In a rather famous paper \cite{chenSimpleMethodReconstructing2004a} a ``robust'' method based on the Savitzky-Golay has been used.
The method is based on the assumption that due to atmospheric effects the observed NDVI tends to be underestimated and that it cannot increase too quickly\footnote{The latter is argued by the biological impossibility of such fast vegetation changes}.

\textbf{Algorithm:}
\begin{enumerate}
    \item Remove points which are labeled as cloudy
    \item Remove points which would indicate an increase greater than 0.4 within 20 days
    \item Linearly interpolate to obtain an equidistant time series $X^0$
    \item Apply the Savitzky-Golay Filter to obtain a new time series $X^1$
    \item Update $X^1$ by applying again a Savitzky-Golay Filter. Repeat this until $w^T |X^1-X^0|$ stops decreasing, where w is a weight vector with $w_i = \min\left(1, 1 - \frac{X^1_i-X^0_i}{\max_i\|X^1_i-X^0_i\|}\right)$. This reduces the impact of outliers\footnote{Here we call a point $i$ an outlier if $X^0_i<X^1_i$.} and by repeating this step we approach the ``upper NDVI envelope''
\end{enumerate}

\begin{my_pros_cons_table}{
        \item Upper envelope matches intuition for the NDVI. Therefore, it is robust against outliers with small values.
        \item
    }{
        \item Not generalizable to other spectral indices.
        \item Linear interpolation to account for missing data might be not appropriate.
        \item No smooth interpolation between two measurements.
    }
\end{my_pros_cons_table}


\subsubsection*{Extension: Spatial-Temporal-Savitzky-Golay Filter}
One notable adaptation of the Savitzky-Golay is the presented by \cite{caoSimpleMethodImprove2018b}. The key difference is the additional assumption of the cloud cover being discontinuous and that we can improve by looking at adjacent pixels\footnote{Here, we say that a pixel is adjacent if it is the same pixel but from a different year (keeping the same day of the year) or (if not enough of such temporal-adjacent pixel are found) it is spatially adjacent}. Because we are working with rather high resolution satellite data, and we need the variance in the predictors we will waive this extension.

\subsection{Double Logistic}
\label{sec:double_logistic}
The Double Logistic smoothing as introduced in \cite{beckImprovedMonitoringVegetation2006} heavily relies on shape assumptions of the fitted curve (i.e. the NDVI time series).

Assumptions:
\begin{itemize}
    \item There is a minimum NDVI level $Y_{\min}$ in the winter (e.g. due to evergreen plants), which might be masked by snow. This can be estimated beforehand, taking into several years into account.
    \item The growth cycle can be divided into an increase and a decrease period where the time series follows a logistic function. The maximum increase (or decrease) is observed at $t_0$ (or $t_1$) with a slope of $d_0$ (or $d_1$).
\end{itemize}

The equation of the double-logistic fit is given by:
\begin{equation*}
    Y(t) = Y_{\min} + \left(Y_{\max}-Y_{\min}\right)\left(\frac{1}{1+e^{-d_0(t-t_0)}}+\frac{1}{1+e^{-d_1(t-t_1)}}-1\right)
\end{equation*}
Where the five free parameters: $Y_{\max}$, $d_0$, $d_1$, $t_0$, $t_1$ are initially estimated by least squares. Similar as for the Savitzky-Golay Filter (c.f. section~\ref{sec:Savitzky–Golay}) we reestimate (only once) the parameters by giving less weight to the overestimated observations and more weight to the underestimated observations\footnote{For the details on the weights we refer to \cite{beckImprovedMonitoringVegetation2006}}.

\subsection{Fourier Approximation}
\label{sec:fourier_approx}
$$
    \operatorname{NDVI}(t)=\sum_{j=0}^{2} a_{j} \times \cos \left(j \times \Phi_{t}\right)+b_{j} \times \sin \left(j \times \Phi_{t}\right)
$$
where $\Phi=2 \pi \times(t-1) / n$.

\cite{beckImprovedMonitoringVegetation2006} shows in their lag-plots a heavy autocorrelation of resiudals

\begin{my_figure}[h]{width=1\textwidth}{interpol/fourier_dl_comparison}
    \caption{Here we observe the nice fitting possibilitys of the two parametric methods but notice also some misbehaviour}
\end{my_figure}

% \subsection{Polynomial interpolation}
% \label{sec:Polynomial}

% \subsection{Polynomial approximation}
% \label{sec:Polynomial}

\subsection{Cubic Smoothing Splines}
\label{sec:Cubic}
We interpolate with a function in $C^2$ (space of three time continuous differentiable functions) which is defined piecewise by cubic polynomials.
**Pros**
Regression splines (B-splines)
\cite{woodSmoothingParameterModel2016}
use a basis of the spline space (e.g. B-splines or j-th cardinal basis) and fit the splines of degree k to approximate the data.

\subsection{B-splines}
\label{sec:B}
from \cite{lycheSplineMethods2005}
$$
    S(x)=\sum_{j=0}^{n-1} c_{j} B_{j, k ; t}(x)
$$
$$
    \begin{array}{r}
        B_{i, 0}(x)=1, \text { if } t_{i} \leq x<t_{i+1}, \text { otherwise } 0 \\
        B_{i, k}(x)=\frac{x-t_{i}}{t_{i+k}-t_{i}} B_{i, k-1}(x)+\frac{t_{i+k+1}-x}{t_{i+k+1}-t_{i+1}} B_{i+1, k-1}(x)
    \end{array}
$$

**Smoothing:**
We can relax the constrain that we have to perfectly interpolate. Thus we use the minimum number of knots\footnote{SciPy uses FITPACK and DFITPACK, the documentation suggests that smoothness is achieved by reducing the number knots used} such that:
$\sum_{i=1}^n(w (y_i - \hat y_i))^2 \leq s$
\begin{my_pros_cons_table}{
        \item can be assigned degrees of freedom
        \item extendable to "smooth" version
        \item performs also well if points are not equidistant
    }{
        \item smoothing process does not translate well to a interpretation (unlike smoothing splines)
        \item choice of smoothing parameter $s$
    }
\end{my_pros_cons_table}


\subsection{Natural Smoothing Splines}
\label{sec:Natural}
Let $\mathcal F$ be the Sobolev space (the space of functions of which the second derivative is integrable). Then the unique\footnote{Strictly speaking it is only unique for $\lambda > 0$} minimizer
$$\hat m :=\argmin_{f \in \mathcal F} \sum_{i=1}^{n}\left(Y_{i}-{f}\left(x_{i}\right)\right)^{2}+\lambda \int {f}^{\prime \prime}(x)^{2} d x$$
is a natural\footnote{It is called natural since it is affine outside the data range ($\forall x\notin [x_1, x_n]:\hat m''(x) = 0$)} cubic spline.

\begin{my_pros_cons_table}{
        \item can be assigned degrees of freedom (trace of the hat-matrix)
        \item efficient estimation (closed form solution)
        \item intuitive penalty (we don't want the function to be too ``wobbly'' --- change slopes)
        \item performs also well if points are not equidistant
        \item fixes the Runge's phenomenon (fluctuation of high degree polynomial interpolation)
    }{
        \item choose $\lambda$
    }
\end{my_pros_cons_table}


\subsection{Penalized Regression Splines}
\label{sec:Penalized}
Intuition: similar as Natural Smoothing Splines, but we choose knots


\subsection{Whittaker Smoother}
\label{sec:whittaker}
% from [HERE](https://eigenvector.com/wp-content/uploads/2020/01/WhittakerSmoother.pdf):  
%     The Whittaker Smoother: Eiler's paper[1] introduces the following objective function
%     $$
%     O(\mathbf{z})=(\mathbf{y}-\mathbf{z})^{\mathrm{T}} \mathbf{W}_{0}(\mathbf{y}-\mathbf{z})+\lambda_{\mathrm{s}} \mathbf{z}^{\mathrm{T}} \mathbf{D}_{\mathrm{s}}^{\mathrm{T}} \mathbf{D}_{\mathrm{s}} \mathbf{z}
%     $$
%     where $\mathbf{y}$ is a $N \times 1$ vector of measured data, $\mathbf{z}$ is smooth curve to be fit to the data, $\mathbf{W}_{0}$ is a diagonal matrix of weights (typically $0 \leq w_{0, n} \leq 1$ for $n=1, \ldots, N, \mathbf{D}_{\mathrm{s}}$ is a second derivative operator (e.g., $\mathbf{D}_{\mathrm{s}} \mathbf{z}$ is the second derivative of $\mathbf{z}$ ) and $\lambda_{\mathrm{s}}$ is a scalar penalty on the smoothing term. When data are missing, the corresponding weight, $w_{0, n}$, can be set to zero. Once that $\mathbf{W}_{0}$ and $\lambda_{\mathrm{s}}$ are given (set by default or provided by the user) the corresponding estimate of $\mathbf{z}$ is given by
%     $$
%     \hat{\mathbf{z}}=\left(\mathbf{W}_{0}+\lambda_{\mathrm{s}} \mathbf{D}_{\mathrm{s}}^{\mathrm{T}} \mathbf{D}_{\mathrm{s}}\right)^{-1} \mathbf{W}_{0} \mathbf{y}
%     $$
%     For example, an optical emission (OES) spectrum is plotted Figure 1 along with two smoothed versions shown for $\mathbf{W}_{0}=\mathbf{I}$ and $\lambda_{\mathrm{s}}=0.1$ (low smoothing) and $\lambda_{\mathrm{s}}=10$ (stronger smoothing).   
% **Original paper states use of the first derivative**  
% --> second derivative is very similar to smoothing splines

\subsection{Kriging}
\label{sec:Kriging}

Kriging was developed in geostatistics to deal with autocorrelation of the response variable at nearby points. By applying the notion that two spectral indices which are (timewise) close should also take similar values we justify the application of Kriging. In the end we would like to fit a smooth Gaussian process to the data. For this subsection we will follow \cite{diggleGaussianModelsGeostatistical2007}.

\subsubsection*{Definitions and Assumptions}

A \textit{Gaussian Process} $\{S(t) : t\in \mathbb R\} $ is a stochastic process if $(S(t_1),\dots,S(t_k))$ has a multivariate Gaussian distribution for every collection of times ${t_1, \dots , t_k}$.
$S$ can be fully characterized by the mean $\mu(t):=E[S(t)]$ and its covariance function $\gamma\left(t, t^{\prime}\right)=\operatorname{Cov}\left(S(t), S\left(t^{\prime}\right)\right)$

Assumption:
We will assume the Gaussian process to be stationary. That is for $\mu(t)$ to be constant in $t$ and $\gamma(t,t')$ to depend only on $h=t-t'$. Thus, we will write in the following only $\gamma(h)$.\footnote{Note that the process is also \textit{isotropic} (i.e. $\gamma(h)=\gamma(\|h\|$) since we are in a one-dimensional setting and the covariance is symmetric.}


We also define the variogram of a Gaussian process as
$$V(h):=V\left(t, t+h\right):=\frac{1}{2} \operatorname{Var}\left(S(t)-S(t+h)\right)\\ %align XXX
    =(\gamma(0))^2(1-\operatorname{corr}(S(t),S(t+h)))
$$
And decide to use a gaussian Variogram defined by
$$V(h) = p \cdot\left(1-e^{-\frac{h^{2}}{\left(\frac{4}{7} r\right)^{2}}}\right)+n,$$
where $h$ is the distance, $n$ is the nugget, $r$ is the range and $p$ is the partial sill visuilized in figure~\ref{fig:interpol/kriging-gauss-variogram}.\footnote{Strictrly speaking we use a scaled version of the variogram. Thus only the ratio of $p/n$ matters.}
\begin{my_figure}[h]{width=0.7\textwidth}{interpol/kriging_gauss_variogram}
    \caption{Gaussian Variogram with nugget=1, partial sill=3, range=55}
    \label{fig:interpol/kriging_gauss_variogram}
\end{my_figure}

\begin{my_figure}{width=1\textwidth}{interpol/kriging_parameter}
    \caption{On the left we see how the interpolation change if we increase the nugget and the range parameter. On the right we compare two kriging interpolations where one takes parameters by numerically maximizing the (which results in a very small nugget) and the other takes the median of many such numerical optimizations.}
\end{my_figure}


\subsection{Other Methods to study:}
From inroduction of \cite{chenSimpleMethodReconstructing2004a}:\\
(1) threshold-
based methods, such as the best index slope extraction
algorithm (BISE) (Viovy et al., 1992); (2) Fourier-based
fitting methods (Cihlar, 1996; Roerink et al., 2000; Sellers
et al., 1994); and (3) asymmetric function fitting methods
such as the asymmetric Gaussian function fitting approach
(Jonsson Eklundh, 2002) and the weighted least-squares
linear regression approach (Swets et al., 1999).
