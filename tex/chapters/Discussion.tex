\chapter{Discussion}
    \todo[inline]{Here in the discussion, you should take up the points you mentioned in the introduction}

    SCL is prone to errors as can be seen in figure~\ref{fig:witzwil_selected_satellite_images}. A machine learning approach like the one developed in \cite{raiyaniSentinel2ImageScene2021} could be used instead.

\section{Interpolation Methods}{
    \todo[inline]{You already capture the "main" structure of your thesis with the interpolation and the NDVi correction sections. Can you combine them both in a "synthesis" subsection at the end of the discussion?}
}

XXX discuss results from table \label{tab:methods_vs_yieldprediction}

\section{NDVI Correction}{
    \subsection{Bootstrap}
        The question arises if we can build the correction model on the same year as we want to apply it on. Usually, a similar approach might carry the danger of overfitting. However, we have not used any ground truth at any point (until the evaluation). Instead, we estimated the ``true'' NDVI with the assumption \ref{true_ndvi_assumption} via OOB. Thus, we have bootstrapped our way out of the problem. Consequently, we reason that we can apply our method to a new (comparable) dataset and solve the correction again via this bootstrap.
    \subsection{Using Additional Covariates}{
        \todo{where does this section belong to? Chapter `NDVI Correction' or `Further Work'?}
        In section \ref{sec:corr_data_table} we have only used the spectral data (and the observational NDVI calculated from them) as covariates. Since we have the weather data available (c.f. REF-SEC), it would be a small effort to incorporate it, together with statistics collected from it (i.e. GDD or `rainfall in the last 30 days'). 
    
        We decided against using this data, because on the one hand we have the problem that we have practically too few observations (we observe only 5 years) and we expect the weather in our study region to be rather homogeneous which is suggested by the fact that the weather data published by Meteoswiss are for a grid with a resolution of 1 km. On the other hand, we want the underlying model not to learn improper relationships. For example, the model might automatically predict a high NDVI for a day in summer (detected by high GDD / many sunshine hours / high temperature) just because it is ``used'' to observing a lot of vegetation in summer. 
        Including temporally (e.g., $P_{t-1}$ and $P_{t+1}$) and geographically adjacent pixels would likely improve performance. However, for simplicity, we omit it here\footnote{This is done for simplicity of understanding and using the model, since one would need to adapt to some convention of how to supply the data of adjacent pixels without redundancy (i.e. supplying $P_t$ multiple times).}.
    }

    - weight/uncertainty function 
    (problem of weight function -> some outer points get really low weights (just because others in the middle have very little residuals and thus very high weight))

    \subsection{High RMSE in Yield Prediction}{
        How much can we expect to get? We have multiple sources of uncertainty in the data:
        \begin{Nenumerate}
            \item Uncertainty in Yield data collected by the combine harvester
            \item Uncertainty in Yield data through rasterization
            \item Uncertainty in satellite images through ``measurement errors'' introduced via clouds and other atmospheric effects 
            \item Uncertainty introduced by interpolating (especially when long data-gaps are present)
        \end{Nenumerate}
    }
}
