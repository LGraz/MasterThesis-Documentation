\chapter{NDVI Correction / Improve NDVI Data}

Let's remind ourselves that the data from the Sentinel-2 is equipped with a scene classification layer (scl) and we therefore have some information of what is observed at each pixel for each sampled time (c.f. table~\ref{table:satelite/scl_classes}). In this chapter we would like to improve the observed NDVI values by using more information than just the two bands used to calculate the NDVI (B4 and B8).

\section{Considering other SCL Classes}
In figure ~\ref{fig:ndvi_corr/residuals_scl_classes} we see for example that some blue points\footnote{The blue points correspond to the scl-class 10: Thin cirrus clouds} follow the interpolated line closely and that they might be useful in improving an interpolation fit.

\begin{my_figure}[ht]{width=0.6\textwidth}{ndvi_corr/residuals_scl_classes}
    \caption{A smoothing splines fit considering green and yellow points (scl-classes 4 and 5)}
    \label{fig:ndvi_corr/residuals_scl_classes}
\end{my_figure}

To get an impression whether there is some useful information contained in the remaining scl-classes (all except 4 and 5) we would like to compare the observed NDVI with the true NDVI. But since we do not have any ground truth data we will make the following assumtion:


\begin{definition}{XXXAssumption (true NDVI)}
    The true NDVI value at time $t$ can be successfully estimated by out-of-bag interpolation using high quality observations. That is the interpolated value (using XXX) considering the points $P^{scl45}\setminus P_t$. In the following we will call this estimate the ``true''-NDVI
\end{definition}

shall pair every observed NDVI value with its out-of-bag-estimate. Then for each category we collect all pairs and create a scatter plot in fig~\ref{fig:ndvi_corr/scl_residuals_scatter}XXXXXXXXXXXXXX



\begin{enumerate}
    \item For each pixel and for each observation (every scl-class):\\
          estimate the NDVI value (via out-of-the-box interpolation\footnote{That is, we use all observations (with scl-class 4 or 5) but the current one.})
    \item
\end{enumerate}







\begin{my_figure}[h]{width=1\textwidth}{ndvi_corr/scl_residuals_scatter}
    \caption{XXX caption XXX}
    \label{fig:ndvi_corr/scl_residuals_scatter}
\end{my_figure}
