\chapter{NDVI Correction / Improve NDVI Data}

Let's remind ourselves that the data from the Sentinel-2 is equipped with a scene classification layer (scl) and we therefore have an idea what is observed at each pixel for each sampled time (c.f. table~\ref{table:satelite/scl_classes}). In this chapter we would like to improve the observed NDVI values by using more information than just the two bands used to calculate the NDVI (B4 and B8).

\section{Considering other SCL Classes}
In figure ~\ref{fig:ndvi_corr/residuals_scl_classes} we see for example that some blue points\footnote{The blue points correspond to the scl-class 10: Thin cirrus clouds} follow the interpolated line closely and that they might be useful in improving an interpolation fit.
\begin{my_figure}[ht]{width=0.6\textwidth}{ndvi_corr/residuals_scl_classes}
    \caption{A smoothing splines fit considering scl45}
    \label{fig:ndvi_corr/residuals_scl_classes}
\end{my_figure}

In order to decide XXXXXXXXXXXXXXXXXXXXXXXXx



\begin{enumerate}
    \item For each pixel and for each observation (every scl-class):\\
          estimate the NDVI value (via out-of-the-box interpolation\footnote{That is, we use all observations (with scl-class 4 or 5) but the current one.})
    \item
\end{enumerate}







\begin{my_figure}[h]{width=1\textwidth}{ndvi_corr/scl_residuals_scatter}
    \caption{XXX caption XXX}
    \label{fig:ndvi_corr/scl_residuals_scatter}
\end{my_figure}
