\chapter{NDVI Correction}

{
    Let's remind ourselves that the data from the Sentinel-2 is equipped with a scene classification layer (\textit{SCL}) and we therefore have some information of what is observed at each pixel for each sampled time (cf. table~\ref{tab:satelite/scl_classes}). So far we have only considered cloud-free points (i.e. SCL-classes 4 and 5). In this chapter we would like to improve the NDVI interpolation by inspecting also other SCL-classes and by using more information than just the two bands used to calculate the NDVI (B4 and B8).
}

\section{Considering other SCL Classes}{
    In figure~\ref{fig:ndvi_corr/residuals_scl_classes} we notice that some blue points\footnote{The blue points correspond to the SCL-class 10: Thin cirrus clouds} follow the interpolated line closely and that they might be useful in improving an interpolation fit.

    \begin{my_figure}[ht]{width=0.6\textwidth}{ndvi_corr/residuals_scl_classes}
        \caption{A smoothing splines fit considering green and yellow points (SCL45)}
        \label{fig:ndvi_corr/residuals_scl_classes}
    \end{my_figure}

    To get an impression whether there is some useful information contained in the remaining SCL-classes (all except 4 and 5) we would like to compare the observed NDVI with the true NDVI. But since we do not have any ground truth data, we will make the following assumption:

    \begin{assumption}{1}%(true NDVI)
        \label{true_ndvi_assumption}
        The true NDVI value at time $t$ can be successfully estimated by out-of-bag interpolation using high quality observations. That is the interpolated value (using an interpolation method from chapter \ref{sec:itpl}) considering the points $P^{SCL45}\setminus P_t$. In the following, we will call this estimate the ``true''-NDVI.
    \end{assumption}

    We would like to get an idea if there is any hope to recover infomation from SCL-classes other than 4 and 5. For that, we will check for the other SCL-classes if there is a relation between the ``true''-NDVI\footnote{\label{footnote:truendvi} i.e. the out-of-bag (OOB) estimate using smoothing splines} and the observed NDVI. Thus, we pair each ``true''-NDVI with its observed one, collect all pairs and create a scatter plot for each SCL-class in fig~\ref{fig:ndvi_corr/scl_residuals_scatter}.
    As expected the ``true'' and the observed NDVI seem to be highly correlated for SCL45. But we can also detect some patterns of correlation in the SCL-classes 2, 3, 7, 8 and 10.  

    \begin{my_figure}[h]{width=1\textwidth}{ndvi_corr/scl_residuals_scatter}
        \caption{For each SCL class, we compare the true NDVI with the observed NDVI. (The true NDVI was estimated with OOB smoothing splines and we used all observations of 10\% of the total training pixels.)}
        \label{fig:ndvi_corr/scl_residuals_scatter}
    \end{my_figure}

    It might be tempting to include some of the above SCL classes (for interpolation). But on the one hand the choice would not be objective and on the other hand the correlation seems to be weaker than for SCL45. Therefore, in the following section we shall try to correct the observed NDVI and estimate the uncertainty of each correction.  
}



\section{Correction}{
    \label{sec:corr_correction}
    % \subsection{XXX idea -and- stepwise plots}
    {
        We recall the satellite images in figure~\ref{fig:satelite/time_series_2021_P112/35_scl4_2021-06-03.png}, where we had cloudy images despite scl4 labeled and see fragments in figure~\ref{fig:satelite/time_series_2021_P112/40_scl10_2021-06-28.png} even though we are supposed to see clouds (scl 10 - Cirrus clouds). The SCL classification is based only on a mixed model trained using the s2 bands. 
        
        We will improve our NDVI interpolation by not relying on the existing SCL classification, but by training our own model to estimate/correct NDVI using all S2 bands (see sections~\ref{sec:corr_data_table} and~\ref{sec:corr_methods}). After we have corrected the observed NDVI, we will find out how uncertain our corrections are and translate these uncertainties into weights (in section~\ref{sec:corr_uncertainty}). These we will use for the subsequent interpolation. This step-by-step procedure is illustrated by the REF graph in the appendix.

        Finally, in section~\ref{sec:ndvi_corr_eval} we will evaluate this correction procedure, considering different interpolation methods and correction models.
    }

    \subsection{Response and Covariates}{
        \label{sec:corr_data_table}

        For training a NDVI correction model, we need ground-truth (response) and informative covariates. We organize those in a table, where each row corresponds to a $P_t$ (i.e., a pixel at a time $t$). 
        For the response we will again use the assumption \ref{true_ndvi_assumption}. There is no canonical answer to the question which covariates we should use. It is a tradeoff between simplicity/generalizability and performance (with the danger of overfitting). 
        Our desire with the NDVI correction is to develop a product that is simple for others to understand and use. Therefore, in the subsequent we will only take the spectral data of the satellite and the observed NDVI derived from it as covariates\footnote{We do not mention the intercept explicitly, but it will also be included.}.  
    }    

    \subsection{Correction Methods}{
        \label{sec:corr_methods}
        In the following, we will introduce different modelling approaches, which we will use to model the relation between the response $y = y_{\text{true OOB NDVI}}\in \R^n$ and the covariates encoded in the design matrix\footnote{This is the Matrix which contains of all covariates.} $X\in\R^{n \times p}$. Furthermore, we will use the matlab `\texttt{:}' notation to indicate rows and collumns of a matrix (e.g. $X{:,3}$ is the $3$rd collumn of $X$). 

        XXX Note that in order to reduce computation time only $10\%$ of the training data has been used to fit the subsequent models.

        
%     ML-methoden: muss ich sie beschreiben? Es scheint mir wissen zu sein, was ich vorraussetzen kann. 
    % --> ja, kurz und bündig erklären
\subsection{Ordinary Least Squaers (\textit{OLS})}{
    The OLS is a linear model which aims to minimize the sum of the squared residuals. Let $y\in \R^n$ be the vector of responses and $X\in \R^{n\times p}$ be the design matrix, where each row corresponds to one pixel and each column consist of one covariate\footnote{Strictly speaking, since SCL-classes are dummy variables }. We assume a linear relationship between $y$ and $X$ and allow for Gaussian noise. That is:
    \begin{equation}
        \label{eq:ols}
        y = X\beta  + \epsilon \quad \text{ where }\epsilon \overset{i.i.d.}{\sim}\mathcal{N}(0,\sigma^2)
    \end{equation}
    Assuming that $X$ is regular, we can estimate the regression coefficients $\beta$ by
    \begin{equation}
        \hat \beta = (X^TX)^{-1}X^Ty = \argmin_{\beta in \R^p}\|y - X\beta\|_2^2 
    \end{equation}

    We will train two models, one using only the SCL-classes as covariates and the other one using all covariates (which are discussed in section \ref{sec:corr_data_table}).

    \begin{my_pros_cons_table}{
        \item Simple method with good interpretability of coefficients.
        \item Computationally cheap.
    }{
        \item Catches only linear relationships.
        \item No integrated variable selection.\footnote{There is the possibility of stepwise model selection by dropping or adding some covariates but for higher p this gets too expensive since the time complexity grows in $\mathcal{O}(np^4)$ (computing $X^TX$ requires $\mathcal{O}(np^2)$ which assuming $n>p$ dominates $\mathcal{O}(p^3) $ needed for the Cholesky Decomposition of $X^TX$) % p^2 time OLS. 
        }
    }
    \end{my_pros_cons_table}
}
\subsection{Least Absolute Shrinkage and Selection Operator (LASSO)}{
    The LASSO can be similarly expressed than the OLS but adds a penalty to the minimization problem:
    \begin{equation}
        \hat \beta_\lambda = \argmin_{\beta \in \R^p}\|y - X\beta\|_2^2 + \lambda \|\beta\|_1 = \argmin_{\beta \in \R^p \text{ and } \|\beta\|_1<\lambda}\|y - X\beta\|_2^2.\footnote{The last two terms are equivalent by lagrangian optimization}
        \label{eq:lasso}
    \end{equation}
    Even though we do not have a closed form solution for equation \eqref{eq:lasso} we can solve it easily via optimization, since the function $\beta \in \{\beta\in\R^p|\|\beta\|_1<\lambda\}\|\ \mapsto \|y - X\beta\|_2^2$  is continuous and convex.

    \cite{tibshiraniRegressionShrinkageSelection2011} shows that the LASSO solution tends to be sparse (for moderate $\lambda$). That is $\beta_i = 0$ for most $i = 1,\dots,p$

    In order to know which $\lambda$ to choose, we try a huge range of possible values. For each $\beta_\lambda$, we calculate the cross-validated $RMSE_\lambda$
    \footnote{The cross validated Root Mean Square Error is the mean of the RMSE's obtained for each fold (using the model trained on the remaining folds). 
    We use the following definition of the $RMSE$: $\sqrt{\sum_{i=1}^n(y-\hat y)^2/n}$
    } (and its standard deviation $\sigma_\lambda$ using the $k$ folds) and define the $\lambda$ with the smallest corresponding  $RMSE_\lambda$ as $\lambda_{min}$. From here we choose the largest $\lambda$ for which the $RMSE_\lambda$ is smaller than $RMSE_{\lambda_{min}}+\sigma_\lambda$. This yields a simpler model while keeping the $RMSE$ reasonable model.

    We will apply the LASSO using the selected covariates in section \ref{sec:corr_data_table} and their second degree of interactions.\footnote{This is if our covariates are $\{a,b\}$, then we will now use $\{a,b,ab,a^2,b^2\}.$}
    
    \begin{my_pros_cons_table}{
        \item Usually yields a sparse solution. This tends to give better generalizability (prediction performance on unseen data).
        \item Successfully deals with correlation in covariates. 
        \item Interpretable results.
    }{
        \item Estimate is biased.
        \item Computationally expensive.
    }
    \end{my_pros_cons_table}
}
\subsection{Random Forest (\textit{RF})}{
    To define a random Forest introduced by \cite{breimanRandomForests2001}  we will first define what a Tree is. A \textit{(decision) Tree} is a graph $(V,E)$ without circles, a distinct root node, every node has at most two children and every leaf has a value assigned to it. At each node there is a boolean condition testing if one variable is greater than some value and a pointer to one child depending on the boolean value. To evaluate a tree we start at the root node, test the boolean expression and go to the node indicated by the resulting pointer. This we repeat until we end up at a leaf-node, where we return the value assigned to it. 
    
    To build such a Tree, we will recursively partition the covariate space using greedy splits\footnote{For computational reasons, we will only use splits along one covariate. So we `cut' our covariate space into rectangles.} decreasing the RMSE\footnote{To calculate the RMSE, we need a prediction. Let $P$ be the current partition, then the predicted value for some $x\in A \in P$ is the mean of the responses of all the points in $A$ (included in the training data).} each time. If the set we want to split contains less than a certain amount of training points, we stop.
    
    To build a \textit{Random Forest} we will bootstrap-aggregate\footnote{That is we will sample (with replacement) several times n observations from our original data and fit a Tree to each such sample.} many such Trees\footnote{Building the Tree, this time we will not test every covariate at each node (for the RMSE minimization) but a node-specific subsample of the covariates. Thus, also the ``second best split'' can be selected.}. The prediction of the Random Forest for a new point $x$ is then the mean of the predictions from all the Trees. 
    \begin{my_pros_cons_table}{
        \item Captures non-linear relationships.
        \item Captures all interactions and performs automatic variable selection.
        \item Can deal with missing data.
    }{
        \item The resulting (prediction) function is not continuous but locally constant.
        \item Computationally expensive.
        \item No interpretability.
    }
    \end{my_pros_cons_table}
}
\subsection{Multivariate Adaptive Regression Splines (\textit{MARS})}{
    A MARS model as introduced in \cite{friedmanMultivariateAdaptiveRegression1991} can be described by 
    \begin{equation}
        \label{eq:mars}
        g(x) = \sum_{m=0}^M \beta_m h_m(x),
    \end{equation}
    where the $h_m$ are simple functions (explained later) and the $\beta_m$ are estimated via Least Squares. 
    
    In the building procedure of a MARS model, we first select many of those simple functions and later drop some of them to avoid overfitting. For the construction of those simple functions, define $\mathcal{B}$ be the set of pairs of `hockystick functions'
    \begin{equation}
        \label{eq:mars_basis_fun}
        \mathcal{B}:=\left\{
            \left(b_1,b_2\right) 
            | \;
            \left(b_1(x),b_2(x)\right) = \left(\left(x_{j}-d\right)_+,\left(d-x_{j}\right)_+\right),
            %  x\in\R^p ,
            d =X_{1, j},  \ldots, X_{n, j},\;
            j=1, \ldots, p
        \right\}
    \end{equation}
    and the set $\mathcal{M}=\left\{1\right\}$ of all functions currently in the model. Now, consider $\mathcal{C}$ the set of candidate functions-pairs 
    \begin{equation}
        \label{eq:mars_candidate}
        \mathcal{C}:=\left\{
            \left(h(\cdot)b_1(\cdot),  h(\cdot)b_2(\cdot) \right)
            \;| \;\; h\in\mathcal{M}, \; 
            (b_1,b_2) \in \mathcal{B}
        \right\}
    \end{equation}
    and select the pair (which when added to $\mathcal{M}$ and the coefficients refitted) reduces the RMSE the most. Add the selected pair to $\mathcal{M}$ and repeat until the RMSE reduction becomes insignificant.

    Finally, to avoid overfitting, we prune the set $\mathcal{M}$ by optimizing a LOOCV score.\footnote{This means that we perform an iterative procedure to reduce the number of functions in $\mathcal{M}$. For every function $h$ in $\mathcal{M}$, we compute the model using $\mathcal{M}\\\{h\}$. We discard the function which -- when excluding from $\mathcal{M}$ -- leads to the best LOOCV score.}  

    To reduce computational complexity, we follow the recommendation from REF\cite{stephenEarthMultivariateAdaptive2021} and restrict $h$ in equation~\eqref{eq:mars_candidate} to be of degree one (so it is also in a pair of $\mathcal{B}$). Consequently, $\mathcal{C}$ contains functions with a degree of at most 2. 

    \begin{my_pros_cons_table}{
        \item Catches non-linear relationships.
        \item Interpretability via functions in $\mathcal{M}$ and their coefficients.
        \item Allows for interactions with variable selection.
    }{
        \item Computationally expensive (can be reduced by restricting the degree of interactions).
    }
    \end{my_pros_cons_table}
}
\subsection{General Additive Model (\textit{GAM})}{
    GAMs as described in \cite{hastieGeneralizedAdditiveModels1987} are a special case of Projection Pursuit Regression, where only the $p$ directions parallel to the coordinate axes are considered. The result is different to a linear model since the coordinate functions are not restricted to be linear but are assumed to be non-parametric functions. The model can be written as:
    \begin{equation}
        \label{eq:gam}
        g_{add}(x) = \mu + \sum_{i=1}^pg_j(x_j).\footnote{wher $g_j$ is a real-valued function. For identifiability we also demand $\mathbb{E}[g_j(X_{:,j})] = 0$ for $j=1,\dots,p$.}
    \end{equation}  

    To estimate the non-parametric functions, we can use smoothing splines (ref sec.~\ref{sec:Natural_SS}). For this let $\mathcal{S}_j$ be the function which takes some $z\in\R^n$ and returns the smoothing splines fitted to $(X_{:,j}, z)$ where the smoothing parameter is optimized by GCV.
    Since we cannot fit all $g_j$ simultaneously, we will use a strategy named Backfitting. We basically cycle through the indices $1,\dots p$ and refit $\hat g_j$ each time. The following illustrates the procedure: 
    \begin{eqnarray*}
        1) \quad \hat g_1 &=& \mathcal S_1(y - \mu)    \\
        2) \quad \hat g_j &=& \mathcal S_j(y - \mu -\hat g_1(X_{[:,1]})-\dots -\hat g_{j-1}(X_{[:,{j-1}]})) \quad \text{for }j=2,\dots,p       \\
        3) \quad \hat g_1 &=& \mathcal S_1(y - \mu -\hat g_2(X_{[:,2]})-\dots -\hat g_p(X_{[:,p]}))       \\
        4) \quad \hat g_j &=& \mathcal S_j(y - \mu - \sum_{k\neq j}\hat g_k(X_{[:,k]})) \quad \text{for }j=2,\dots,p       \\
         & \vdots        
    \end{eqnarray*}
    We repeat step $3)$ and $4)$ until the change falls below some tolerance.

    \begin{my_pros_cons_table}{
        \item Captures non-linearity.
        \item Good interpretability.
    }{
        \item No automatic variable selection.
        \item Computationally expensive.
    }
    \end{my_pros_cons_table}
}

    }
    
    \subsection{Uncertainty Estimation}{
        \label{sec:corr_uncertainty}
        Once we correct the NDVI using the previous section, we are left with the problem that not every correction is equally reliable.\footnote{One correction is illustrated in the figure \ref{fig:step_plot/2017-206_corr.pdf}. In this figure, the outer points (labeled as clouds) have a large scatter.}. Hence, we are interested in a measure of how uncertain an estimate is. 

        We do this by replacing the response with the absolute residuals $v := \left|y -\hat y\right|$ and modeling their relationship with the covariates defined by $X$.  In this way, we obtain a model for the absolute residuals $v$ and the estimator $\hat v$.  
    }

    \subsection{Interpolation}{
        \label{sec:corr_link}
        Consider now a pixel $P$, $\hat y^{(P)}$ its corrected NDVI and $\hat v^{(P)}$ the estimated uncertainties of $\hat y^{(P)}$. In order to interpolate $\hat y^{(P)}$ we will give less weight unreliable observations. Thus, we define the weightfunction: 
        \begin{equation}
            \label{eq:corr_link}
            w^{(P)}_\tau:=\frac{1}{R} \frac{1}{\hat v^{(P)}_\tau}, 
            \quad \text{ for } \tau=1,\dots, n_P
        \end{equation}  
        where $\tau$ is an index over the satellite images and $R:=\frac{\sum_i^{n_P}\hat v^{(P)}_i}{n_P}$ a normalization constant. The normalization is needed, since for some interpolation methods inflating the sum of weights would decrease the effect of the smoothing. 
    }
}

\section{Resulting Interpolation Stragegies}{
    \label{sec:corr_itpl_stat}
    We have developed the following procedure to obtain a new interpolation (keyword-wise):
    \begin{Nenumerate}
        \item OOB Interpolation (+ robustify?)
        \item Correction 
        \item Uncertainty estimation
        \item Interpolation (+ robustify?)
    \end{Nenumerate}
    At each step we have a choice, more precisely:
    \begin{Nitemize}
        \item Interpolation: Smoothing Splines / Double Logistic
        \item Robustify: Yes / No
        \item Correction \& uncertainty estimation: RF / OLS -- considering only SCL-classes / OLS -- considering all selected covariates / MARS / GAM / LASSO / no correction.
    \end{Nitemize}
    As it is not feasible to try every possible combination, we make the following restrictions of which combinations we will consider:
    \begin{Nitemize}
        \item We use the same interpolation method each time.
        \item Either we robustify both times or we do not robustify at all.
        \item We use the same underlying method for correction and uncertainty estimation.
    \end{Nitemize}

    In this fashion, we obtain 28 distinct interpolation strategies, which we will benchmark in the next section.
}

\section{Evaluation Method}{
    \label{sec:ndvi_corr_eval}
    In this section, we introduce the relative yield-estimation-accuracy (\textit{RYEA}) and utilize it to evaluate the interpolation strategies from section~\ref{sec:corr_itpl_stat}. 

    \begin{definition}(RYEA) 
        Let $y\in \R^n$ be the yield, $M$ be a model for estimating $y$, and $\hat y = M(X)$ where $X$ describes the data\footnote{We will use the matrixes derived in section \ref{sec:corr_yield_est}}. 
        We define the RYEA as the relative RMSE in yield estimation. Formally expressed:
        \begin{equation}
            RYEA = \frac{\sqrt{\sum_{i=1}^n(y_i - \hat y_i)}}{\bar y}
        \end{equation}
        \label{def:ryea}
    \end{definition}

    \subsection{Idea}{
        The fundamental assumption is that the closer the interpolated NDVI time series is to the true one, the better it can be used to determine crop yield. Implicitly, we believe that an NDVI time series which better models yield will incorporate more true information about the underlying vegetation. 
        Therefore, we want to determine a comparable RYEA for each interpolation strategy and choose it as a benchmark criterion. 
        This is an objective measure, since we have not considered crop yield in any of our previous steps. Moreover, this criterion is justified by the fact that yield estimation has been a motivation for the interpolation.
    }

    \subsection{Yield Estimation}{
        \label{sec:corr_yield_est}
        For all the pixels, we will interpolate the NDVI time series with every interpolation strategy. From the interpolated NDVI time series, we would like to estimate the yield. However, given the high dimensionality and different lengths of the interpolation (not every time series has the same start and end point), we must first map each NDVI time series into a low dimensional vector space. For this we will use the following statistics:
        \begin{Nitemize}
            \item Maximum slope
            \item Minimum slope
            \item Integral\footnote{\label{note:integral-min} We will only consider the integral of the function $max(0, NDVI - 0.3)$, where $0.3$ is assumend to be a minimal NDVI value. REF} over all
            \item Peak (i.e. maximal NDVI)
            \item Peak GDD (i.e. value at which the peak is attained)
            \item Integral\footnoteref{note:integral-min} up to the peak
            \item Integral\footnoteref{note:integral-min} after peak
            \item Integral\footnoteref{note:integral-min} from 0-685 GDD
            \item Integral\footnoteref{note:integral-min} from 685-1075 GDD    
        \end{Nitemize}
        For the choice we were inspired by REF-kamir. However, we deliberately omit any statistic that involves the minimum (e.g. the NDVI-range), since we regard the minimum as very error-prone (clouds) and uninformative measure. 
        
        As a result, we obtain for each interpolation strategy a matrix in which each row corresponds to a pixel and contains both the yield and the characterizing statistics.
        Using this matrix, we train a random forest\footnote{The choice of the modelling approach does not matter to much, as long as it is general enough (i.e. able to approximate any function) and we use the same one for each interpolation strategy.} for yield estimation, and compute the integrated OOB estimates\footnote{By the integrated OOB estimates, we denote the predictions for each pixel where only trees are used, where the pixel has not been used (as $n_{tree}$, the number of Trees, grows the fraction of trees which do not contain a certain pixel converges to $\frac{1}{e}$).} $\hat y$. Finally, for each interpoaltion strategy, we calculate the RYEA. The results are shown in table \ref{tab:methods_vs_yieldprediction}.
    }

    
    
    
}



%satelite/time_series_2021_P112/35_scl4_2021-06-03.png 
%satelite/time_series_2021_P112/40_scl10_2021-06-28.png 
%satelite/time_series_2021_P112/30_scl4_2021-05-09.png 
%satelite/time_series_2021_P112/15_scl5_2021-02-23.png 
%satelite/time_series_2021_P112/45_scl2_2021-07-23.png 
%satelite/time_series_2021_P112/33_scl9_2021-05-24.png
