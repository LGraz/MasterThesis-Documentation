\chapter{NDVI Correction / Improve NDVI Data}

Let's remind ourselves that the data from the Sentinel-2 is equipped with a scene classification layer (SCL) and we therefore have some information of what is observed at each pixel for each sampled time (c.f. table~\ref{tab:satelite/scl_classes}). In this chapter we would like to improve the observed NDVI values by using more information than just the two bands used to calculate the NDVI (B4 and B8).

\section{Considering other SCL Classes}
In figure ~\ref{fig:ndvi_corr/residuals_scl_classes} we see for example that some blue points\footnote{The blue points correspond to the SCL-class 10: Thin cirrus clouds} follow the interpolated line closely and that they might be useful in improving an interpolation fit.

\begin{my_figure}[ht]{width=0.6\textwidth}{ndvi_corr/residuals_scl_classes}
    \caption{A smoothing splines fit considering green and yellow points (SCL45)}
    \label{fig:ndvi_corr/residuals_scl_classes}
\end{my_figure}

To get an impression whether there is some useful information contained in the remaining SCL-classes (all except 4 and 5) we would like to compare the observed NDVI with the true NDVI. But since we do not have any ground truth data, we will make the following assumption:


\begin{definition}{XXXAssumption (true NDVI)}
    The true NDVI value at time $t$ can be successfully estimated by out-of-bag interpolation using high quality observations. That is the interpolated value (using XXX) considering the points $P^{SCL45}\setminus P_t$. In the following, we will call this estimate the ``true''-NDVI
\end{definition}

shall pair every observed NDVI value with its out-of-bag-estimate. Then for each category we collect all pairs and create a scatter plot in fig~\ref{fig:ndvi_corr/scl_residuals_scatter}XXXXXXXXXXXXXX



\begin{enumerate}
    \item For each pixel and for each observation (every SCL-class):\\
          estimate the NDVI value (via out-of-the-box interpolation\footnote{That is, we use all observations (in SCL45) but the current one.})
    \item
\end{enumerate}







\begin{my_figure}[h]{width=1\textwidth}{ndvi_corr/scl_residuals_scatter}
    \caption{XXX caption XXX}
    \label{fig:ndvi_corr/scl_residuals_scatter}
\end{my_figure}


\section{XXX Correction}
roadmap ...

\subsection{xxx idea -and- stepwise plots}

\subsection{xxx data-table-construction}
xxx discussion about choosen covariates:
list of things we considered but rejected + reasoning

\subsection{xxx ml-methods}



\section{xxx Evalutation Method}
yield estimation is a main goal. 
Claim that yield-estimation-accuracy is a objective measure :
    - we have not looked at the yield so far 
    - if the one NDVI-time-series predicts the yield better than a different one, we conclude that the first time-series carries more true information about the plants
Now: "yield ~ NDVI-TS / derived-covariates" 

now we will discuss how to perform the yield estimation
\subsection{yield estimation}
problem: high dimensionality and unequal duration/length -> use features

name approaches for yield estimation (we will use a simple but flexible one)

random forest >> for evaluation out-of-bag estimates
\subsubsection{Covariates used}
reference to kamir et al, why we did choosed some and not others


\section{Results}
\begin{table}
	\begin{center}
		\caption{XXX RMSE of yield prediction}
		\small
		\begin{tabular}{lrrrrrrr}
\toprule
 & RF & OLS-SCL & OLS-all & MARS & GAM & LASSO & no-correction \\
\midrule
ss & {\cellcolor[HTML]{000000}} \color[HTML]{F1F1F1} 1.144 & {\cellcolor[HTML]{F1F1F1}} \color[HTML]{000000} 1.033 & {\cellcolor[HTML]{CACACA}} \color[HTML]{000000} 1.051 & {\cellcolor[HTML]{DDDDDD}} \color[HTML]{000000} 1.042 & {\cellcolor[HTML]{D4D4D4}} \color[HTML]{000000} 1.046 & {\cellcolor[HTML]{DDDDDD}} \color[HTML]{000000} 1.042 & {\cellcolor[HTML]{6A6A6A}} \color[HTML]{F1F1F1} 1.095 \\
dl & {\cellcolor[HTML]{222222}} \color[HTML]{F1F1F1} 1.150 & {\cellcolor[HTML]{ADADAD}} \color[HTML]{000000} 1.115 & {\cellcolor[HTML]{A7A7A7}} \color[HTML]{F1F1F1} 1.116 & {\cellcolor[HTML]{A7A7A7}} \color[HTML]{F1F1F1} 1.116 & {\cellcolor[HTML]{F1F1F1}} \color[HTML]{000000} 1.097 & {\cellcolor[HTML]{EDEDED}} \color[HTML]{000000} 1.098 & {\cellcolor[HTML]{000000}} \color[HTML]{F1F1F1} 1.159 \\
ss-rob & {\cellcolor[HTML]{000000}} \color[HTML]{F1F1F1} 1.144 & {\cellcolor[HTML]{F1F1F1}} \color[HTML]{000000} 1.054 & {\cellcolor[HTML]{A2A2A2}} \color[HTML]{F1F1F1} 1.084 & {\cellcolor[HTML]{878787}} \color[HTML]{F1F1F1} 1.094 & {\cellcolor[HTML]{C3C3C3}} \color[HTML]{000000} 1.072 & {\cellcolor[HTML]{C5C5C5}} \color[HTML]{000000} 1.071 & {\cellcolor[HTML]{8F8F8F}} \color[HTML]{F1F1F1} 1.091 \\
dl-rob & {\cellcolor[HTML]{000000}} \color[HTML]{F1F1F1} 1.159 & {\cellcolor[HTML]{4E4E4E}} \color[HTML]{F1F1F1} 1.128 & {\cellcolor[HTML]{696969}} \color[HTML]{F1F1F1} 1.117 & {\cellcolor[HTML]{F1F1F1}} \color[HTML]{000000} 1.064 & {\cellcolor[HTML]{A8A8A8}} \color[HTML]{F1F1F1} 1.093 & {\cellcolor[HTML]{888888}} \color[HTML]{F1F1F1} 1.105 & {\cellcolor[HTML]{060606}} \color[HTML]{F1F1F1} 1.156 \\
\bottomrule
\end{tabular}

		\normalsize
	\end{center}
\end{table}

\
