\chapter*{\vspace{-3.2cm} Notations}
\label{c:Notation}
\vspace{-0.6cm}
Since this thesis, despite its applied nature, is located at the Mathematics Department, we adhere to the convention of speaking in the first-person plural `we'.\\
Furthermore, only equations that are referenced elsewhere are equipped with a number.

\section*{Variables}\vspace{-0.2cm}
\renewcommand{\arraystretch}{1.3} % for non-dense tables
\begin{longtable}{p{0.12\linewidth} p{0.87\linewidth}}
$c$		& a (vector of) constant(s)\\
$\lambda \in \R$		& a scalar\\
$n\in \mathbb{N}$		& sample size\\
$i,j$		& indices in $\{1,\dots,n\}$\\
$n\in \R^n$		& time, usually in GDD\\
$w \in \R^n$		& a vector of weights for each location $x$\\
$y\in \R^n$		& response in 1-dim interpolation setting\\
$\hat y\in \R^n$		& estimate of $y$\\
$\bar y\in \R$		& sample mean of $y$\\
$r \in \R^n$		& residuals given by $y - \hat y$\\
$X\in\R^{n\times p}$ & the design matrix. Each row corresponds to one observation and each column to one covariate.\\
$X_{[:,j]}$ 	& the j\textsuperscript{th} column of $X$\\
$X_{[i,:]}$ 	& the i\textsuperscript{th} row of $X$
% $x\in \R^n$		& covariable in 1-dim interpolation setting
\end{longtable}

%%%%%%%%%%%%%%%%%%%%%%%%%%%
\section*{Abbreviations and Objects}\vspace{-0.2cm}
\begin{longtable}{p{0.12\linewidth} p{0.87\linewidth}}
NDVI	
		& Normalized Difference Vegetation Index \citep{rouseMonitoringVernalAdvancement1974}.\\

TS	
		& Time Series. \\
IM	
		& Interpolation Method. That is a simple\footnote{I.e., no combination of various methods.} method that interpolates data $(t_i,y_i)_{i = 1,\dots ,n}$ and yields a function $f(t)=y$, approximating the data. \\
IS	
		& Interpolation Strategy. This is the category of functions that map $(t_i,y_i)_{i=1,\dots,n}$ to a function $f(t)=y$, approximating the data. So a IS describes a strategy of how to arrive at an interpolation starting from the data $(t_i,y_i)_{i=1,\dots,n}$. For this, initial data may be corrected (cf. chapter~\ref{sec:corr}), (possible different) IMs (iteratively) used, weightings applied (cf. robustification in section~\ref{sec:loess_robustify}). Note, that strictly speaking, every IM also is an IS. But usually, we expect an IS to involve a more `complex' procedure. \\
S2	
		& Sentinel 2 satellites. Two multi-spectral image satellites deployed by the European Space Agency. \\
SCL	
		& Scene Classification Layer provided by the European Space Agency that gives an estimation of the land cover class of each pixel. It indicates what one can expect at a pixel at a sampled time. For an overview, see table~\ref{tab:satelite/scl_classes}\\
Pixel	
		& A pixel originates of an image pixel and describes a square of 10 x 10 meters in the field that coincides with the resolution (and location) of the Sentinel-2 pixels. Such pixels are illustrated in figure~\ref{fig:satelite/witzwil_2021_P112_yield_cropped.png}. Additional information like yield is also attached.\\

$P_t$	
		& the observed data (weather and spectral bands) at time $t$ and the location of one pixel. \\

$P$	
		& a pixel. We see it as a collection of all the observations at the specified location within one season. More formally, $P := \left\{P_t | t\text{ is a valid sample time within a defined season}\right\}$\\


$P^{SCL45}$	
		& is similar to $P$ but we only consider observations that belong to the classes 4 and 5. This is used done to get a subset of observations which are less contaminated by clouds and shadows.\\

DAS	
		& Days After Sowing\\

GDD	
		& Growing Degree Days -- cumulative sum of ``$\max(0, \text{temperature}-\text{threshold})$''\\

YPE 	
		& (Relative) Yield Prediction Error. See Definition~\ref{def:YPE}\\

OOB 	
		& Out Of the Box. Describes the procedure of  estimating the value for a point by a model that has not seen this point before (see section~\ref{sec:OOB_LOOCV}).\\

LOOCV 	
		& Leave One Out Cross Validation. Describes the procedure of estimating the value for a point by a model that has seen all the points except the current one (see section~\ref{sec:OOB_LOOCV}).
\end{longtable} 

\section*{Statistical Models}\vspace{-0.2cm}
\begin{longtable}{p{0.12\linewidth} p{0.87\linewidth}}
	DL
		& Double Logistic (see section~\ref{sec:double_logistic})\\
	FS
		& Fourier Series (see section~\ref{sec:fourier_approx})\\
	NW
		& Nadaraya-Watson (see section~\ref{sec:Kernel})\\
	UK
		& Universal Kriging (see section~\ref{sec:Kriging})\\
	SG
		& Savitzky-Golay Filter (see section~\ref{sec:Savitzky-Golay})\\
	LOESS
		& Locally Weighted Regression (see section~\ref{sec:loess})\\
	BS
		& B-splines (see section~\ref{sec:B})\\
	SS
		& Smoothing Splines (see section~\ref{sec:Natural_SS})\\
	OLS
		& Ordinary Least Squares (see section~\ref{sec:corr_model_OLS})\\
	OLS\textsuperscript{SCL}
		& OLS using only the observed NDVI and SCL classes (as factor variables)\\
	OLS\textsuperscript{all}
		& OLS using the covariates OLS\textsuperscript{SCL} uses and the spectral bands\\
	LASSO
		& Least Absolute Shrinkage and Selection Operator (see section~\ref{sec:corr_model_LASSO})\\
	GAM
		& General Additive Model (see section~\ref{sec:corr_model_GAM})\\
	RF
		& Random Forest (see section~\ref{sec:corr_model_RF})\\
	MARS
		& Multivariate Adaptive Regression Splines (see section~\ref{sec:corr_model_MARS})\\
\end{longtable} \renewcommand{\arraystretch}{1}

% \bigskip











%%% Local Variables: 
%%% mode: latex
%%% TeX-master: "MasterThesisSfS"
%%% End: 
