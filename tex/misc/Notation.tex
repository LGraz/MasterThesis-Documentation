\chapter*{Notation}
\label{c:Notation}

$c$: a (vector of) constant(s)

$\lambda \in \R$: a scalar

$n\in \N$: sample size

$i,j$ are indices in $\{1,\dots,n\}$

$x\in \R^n$: covariable in 1-dim interpolation setting

$w \in \R^n$: a vector of weights for each location $x$

$y\in \R^n$: response in 1-dim interpolation setting

$\hat y\in \R^n$: estimate of $y$

$\bar y\in \R$: mean of $y$

$r \in \R^n$: residuals given by $y - \hat y$


%%%%%%%%%%%%%%%%%%%%%%%%%%%

Pixel: A pixel describes a specific location in a field. It has the size of 10 x 10 meters and coincides with the resolution (and location) of the sentinel-2 pixels. Such pixels are illustrated in figure~\ref{fig:satelite/yield-raster}. Additional information like yield is also attached.

$P_t$: this describes the observed data (weather and spectral bands) at time $t$ and the location of one pixel. 

$P$: a pixel. We see it as a collection of all the observations at the specified location within one season. More formally, $P := \left\{P_t | t\text{ is a valid sample time within a defined season}\right\}$

SCL: scene classification layer. This indicates what one can expect at a pixel at a sampled time. For an overview cf. table~\ref{tab:satelite/scl_classes}

$P^{SCL45}$: similar to $P$ but we only consider observations which belong to the classes 4 and 5. This is used done to get a subset of observations which are less contaminated by clouds and shadows.

NDVI: normalized vegetation difference index

DAS: days after sowing

GDD: growing degree days -- cumulative sum of ($\text{temperature}-\text{threshold})^+$

XXX ML models and their shortnames

RYEA : relative yield-estimation-accuracy. Definition \ref{def:ryea}
%%% Local Variables: 
%%% mode: latex
%%% TeX-master: "MasterThesisSfS"
%%% End: 
