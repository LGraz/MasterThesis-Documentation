\chapter*{Notations}
\label{c:Notation}

\section*{Variables}

\renewcommand{\arraystretch}{1.3} % for non-dense tables
\begin{tabular}{l l}
$c$		& a (vector of) constant(s)\\
$\lambda \in \R$		& a scalar\\
$n\in \mathbb{N}$		& sample size\\
$i,j$		& are indices in $\{1,\dots,n\}$\\
$x\in \R^n$		& covariable in 1-dim interpolation setting\\
$w \in \R^n$		& a vector of weights for each location $x$\\
$y\in \R^n$		& response in 1-dim interpolation setting\\
$\hat y\in \R^n$		& estimate of $y$\\
$\bar y\in \R$		& sample mean of $y$\\
$r \in \R^n$		& residuals given by $y - \hat y$
\end{tabular}

%%%%%%%%%%%%%%%%%%%%%%%%%%%
\section*{Abbreviations and Objects}
\begin{longtable}{p{0.08\linewidth} p{0.87\linewidth}}
Pixel	
		& A pixel originates of an image pixel and describes a square of 10 x 10 meters in the field which coincides with the resolution (and location) of the Sentinel-2 pixels. Such pixels are illustrated in figure~\ref{fig:satelite/witzwil_2021_P112_yield_cropped.png}. Additional information like yield is also attached.\\

$P_t$	
		& describes the observed data (weather and spectral bands) at time $t$ and the location of one pixel. \\

$P$	
		& is a pixel. We see it as a collection of all the observations at the specified location within one season. More formally, $P := \left\{P_t | t\text{ is a valid sample time within a defined season}\right\}$\\

SCL	
		& Scene Classification Layer provided by the European Space Agency (ESA) that gives an estimation of the land cover class of each pixel. It indicates what one can expect at a pixel at a sampled time. For an overview, c.f. table~\ref{tab:satelite/scl_classes}\\

$P^{SCL45}$	
		& is similar to $P$ but we only consider observations which belong to the classes 4 and 5. This is used done to get a subset of observations which are less contaminated by clouds and shadows.\\

NDVI	
		& Normalized Difference Vegetation Index \citep{rouseMonitoringVernalAdvancement1974}\\

DAS	
		& Days After Sowing\\

GDD	
		& Growing Degree Days -- cumulative sum of ``$\max(0, \text{temperature}-\text{threshold})$''\\

RYEA 	
		& Relative Yield-Estimation-Accuracy. Definition \ref{def:ryea}\\

OOB 	
		& Out Of the Box. Describes the procedure of  estimating the value for a point but not consider the point itself (c.f. section \ref{sec:OOB_LOOCV})\\
\end{longtable} \renewcommand{\arraystretch}{1}


XXX ML models and their shortnames

European Space Agency (ESA)

\subsection*{MATLAB Matrix Notation}{ \label{sec:MATLAB}
		We will use the MATLAB `\texttt{:}' notation to indicate rows and columns of a matrix. That is if $X\in\R^{n\times p}$ is a matrix, then $X_{[:,3]}$ is the $3$rd column of $X$ and $X_{[2,:]}$ is the second row of X. 
}


%%% Local Variables: 
%%% mode: latex
%%% TeX-master: "MasterThesisSfS"
%%% End: 
