\chapter{Reproducibility}\label{app:reproducibility}

\section{Reproduce Results}
For reproducibility of the whole computations, we refer to our codebase at:\\ \url{https://github.com/LGraz/MasterThesis-Code}\\ In order to reproduce our computations and results, set up the directory as described in the README. The the `YieldMapping' Data used, is published alongside \cite{perichPixelbasedCropYield2022}. Execute the computations via the script \texttt{./shell\_scripts/reproduce.sh} and do not execute the python and R files by hand (unless you follow the order in \texttt{./shell\_scripts/reproduce.sh}). 

\section{R--Package}
We also provide an \texttt{R} package for a general time series correction and interpolation if additional data is available at: \\
\url{https://github.com/LGraz/CorrectTimeSeries} \\
In our case we consider the NDVI time series and the additional data consists of the unused spectral bands.

We recommend installing it via the \texttt{devtools} package by:\\
\texttt{devtools::install\_github("LGraz/CorrectTimeSeries")}

In the following, we shall give a stand-alone example of how the \texttt{R} package can be used:

\lstinputlisting[title= Example of how to use the \texttt{CorrectTimeSeries} package]{tex/misc/CorrectTimeSeries.R}

