\chapter{Further Material}

\section{Data and Methods}{
	\subsection{GDD}\label{app:gdd_examples}
		\cite{baileyUsingGrowingDegree2018} tabulates the corresponding GDD for each stage of wheat.
		\begin{table}[H]
			\centering
			\small
			\begin{tabular}{p{0.2\linewidth} p{0.57\linewidth}  p{0.13\linewidth}} 
				\toprule
				Stage  & Description    & GDD \\
				\hline Emergence & Leaf tip just emerging from above-ground coleoptyle. & $125-160$ \\
				\hline Leaf development & Two leaves unfolded. & $169-208$ \\
				\hline Tillering & First tiller visible  & $369-421$ \\
				\hline Stem elongation & First node detectable. & $592-659$ \\
				\hline Anthesis & Flowering commences; first anthers of cereals are visible. & $807-901$ \\
				\hline Seed fill & Seed fill begins. Caryopsis of cereals watery ripe (first grains have reached half of their final size). & $1068-1174$ \\
				\hline Dough stage & Soft dough stage, grain contents soft but dry, fingernail impression does not hold. & $1434-1556$ \\
				\hline Maturity complete & Grain is fully mature and drydown begins. Ready for harvest when dry. & $1538-1665$ \\
				\bottomrule
			\end{tabular}
		\end{table}
}


\section{Interpolation}
\begin{my_figure}[H]{width=1\textwidth}{interpol/2x3_loess_robust}
	\caption{The LOESS smoother \RobItPlot}
	\label{fig:interpol/2x3_loess_robust}
\end{my_figure}

\begin{my_figure}[H]{width=1\textwidth}{interpol/2x3_B-Splines_robust}
	\caption{B-splines \RobItPlot}
	\label{fig:interpol/2x3_B-Splines_robust}
\end{my_figure}

\begin{my_figure}[H]{width=1\textwidth}{interpol/2x3_DL_robust}
	\caption{A Double Logistic curve \RobItPlot}
	\label{fig:interpol/2x3_DL_robust}
\end{my_figure}

% \subsection{Skewness of LOOCV residuals}
% \begin{my_figure}[h]{width=0.6\textwidth}{interpol/res_cv}
% 	\caption{XXX caption XXX}
% 	\label{fig:interpol/res_cv}
% \end{my_figure}


\section{NDVI correction}
\todo[inline]{page breaks}


% step_plot/2017-201_ndvi.pdf 
% step_plot/2017-202_itpl.pdf 
% step_plot/2017-203_itpl_rew.pdf 
% step_plot/2017-204_ndvi_scl.pdf 
% step_plot/2017-205_show_res.pdf 
% step_plot/2017-206_corr.pdf 
% step_plot/2017-207_uncert.pdf 
% step_plot/2017-208_corr_itpl_rew.pdf

\begin{figure*}
            \vspace{-15pt}
			\centering
			\begin{subfigure}[b]{0.42\textwidth}
				\centering
				\includegraphics[width=\textwidth]{step_plot/2017-201_ndvi.pdf}
				\vspace{-20pt}
                \caption[NDVI time series with scl45]%
				{{\footnotesize NDVI time series with scl45}}    
				\label{fig:step_plot/2017-201_ndvi.pdf}
			\end{subfigure}
			\hfill
			\begin{subfigure}[b]{0.42\textwidth}  
				\centering 
				\includegraphics[width=\textwidth]{step_plot/2017-202_itpl.pdf}
				\vspace{-20pt}
                \caption[Interpolation via Smoothing Splines]%
				{{\footnotesize Interpolation via Smoothing Splines}}    
				\label{fig:step_plot/2017-202_itpl.pdf}
			\end{subfigure}

			\vskip\baselineskip
			\begin{subfigure}[b]{0.42\textwidth}   
				\centering 
				\includegraphics[width=\textwidth]{step_plot/2017-203_itpl_rew.pdf}
				\vspace{-20pt}
                \caption[Robustly reweighted fit]%
				{{\footnotesize Robustly reweighted fit}}    
				\label{fig:step_plot/2017-203_itpl_rew.pdf}
			\end{subfigure}
			\hfill
			\begin{subfigure}[b]{0.42\textwidth}   
				\centering 
				\includegraphics[width=\textwidth]{step_plot/2017-204_ndvi_scl.pdf}
				\vspace{-20pt}
                \caption[Now also consider other SCL-classes]%
				{\footnotesize Now also consider other SCL-classes}    
				\label{fig:step_plot/2017-204_ndvi_scl.pdf}
			\end{subfigure}

            \vskip\baselineskip
			\begin{subfigure}[b]{0.42\textwidth}   
				\centering 
				\includegraphics[width=\textwidth]{step_plot/2017-205_show_res.pdf}
				\vspace{-20pt}
                \caption[OOB estim. for each point using SCL45]%
				{{\footnotesize OOB estim. for each point using SCL45}}    
				\label{fig:step_plot/2017-205_show_res.pdf}
			\end{subfigure}
			\hfill
			\begin{subfigure}[b]{0.42\textwidth}   
				\centering 
				\includegraphics[width=\textwidth]{step_plot/2017-206_corr.pdf}
				\vspace{-20pt}
                \caption[Correct NDVI]%
				{\footnotesize Correct NDVI}    
				\label{fig:step_plot/2017-206_corr.pdf}
			\end{subfigure}

            \vskip\baselineskip
			\begin{subfigure}[b]{0.42\textwidth}   
				\centering 
				\includegraphics[width=\textwidth]{step_plot/2017-207_uncert.pdf}
				\vspace{-20pt}
                \caption[Estimate absolute errors]%
				{{\footnotesize Estimate absolute errors}}    
				\label{fig:step_plot/2017-207_uncert.pdf}
			\end{subfigure}
			\hfill
			\begin{subfigure}[b]{0.42\textwidth}   
				\centering 
				\includegraphics[width=\textwidth]{step_plot/2017-208_corr_itpl_rew.pdf}
				\vspace{-20pt}
                \caption[Robust interpolation on weights derived from uncertainties.]%
				{\footnotesize Robust interpolation on weights derived from uncertainties.}    
				\label{fig:step_plot/2017-208_corr_itpl_rew.pdf}
			\end{subfigure}
            \caption{Stepwise illustration of robust NDVI-Correction}
		\end{figure*}


\begin{table}[H]
	\begin{center}
		\caption{Non-relative RMSE for yield prediction}.
		\small
		\begin{tabular}{lrrrrrrr}
\toprule
 & RF & OLS\textsuperscript{SCL} & OLS\textsuperscript{all} & MARS & GAM & LASSO & no corrections \\
\midrule
SS & {\cellcolor[HTML]{1C1C1C}} \color[HTML]{F1F1F1} 1.144 & {\cellcolor[HTML]{F1F1F1}} \color[HTML]{000000} 1.033 & {\cellcolor[HTML]{CFCFCF}} \color[HTML]{000000} 1.051 & {\cellcolor[HTML]{E0E0E0}} \color[HTML]{000000} 1.042 & {\cellcolor[HTML]{D8D8D8}} \color[HTML]{000000} 1.046 & {\cellcolor[HTML]{E0E0E0}} \color[HTML]{000000} 1.042 & {\cellcolor[HTML]{7B7B7B}} \color[HTML]{F1F1F1} 1.095 \\
SS\textsuperscript{rob} & {\cellcolor[HTML]{1C1C1C}} \color[HTML]{F1F1F1} 1.144 & {\cellcolor[HTML]{C9C9C9}} \color[HTML]{000000} 1.054 & {\cellcolor[HTML]{8F8F8F}} \color[HTML]{F1F1F1} 1.084 & {\cellcolor[HTML]{7D7D7D}} \color[HTML]{F1F1F1} 1.094 & {\cellcolor[HTML]{A6A6A6}} \color[HTML]{F1F1F1} 1.072 & {\cellcolor[HTML]{A8A8A8}} \color[HTML]{F1F1F1} 1.071 & {\cellcolor[HTML]{828282}} \color[HTML]{F1F1F1} 1.091 \\
DL & {\cellcolor[HTML]{111111}} \color[HTML]{F1F1F1} 1.150 & {\cellcolor[HTML]{545454}} \color[HTML]{F1F1F1} 1.115 & {\cellcolor[HTML]{525252}} \color[HTML]{F1F1F1} 1.116 & {\cellcolor[HTML]{525252}} \color[HTML]{F1F1F1} 1.116 & {\cellcolor[HTML]{767676}} \color[HTML]{F1F1F1} 1.097 & {\cellcolor[HTML]{747474}} \color[HTML]{F1F1F1} 1.098 & {\cellcolor[HTML]{000000}} \color[HTML]{F1F1F1} 1.159 \\
DL\textsuperscript{rob} & {\cellcolor[HTML]{000000}} \color[HTML]{F1F1F1} 1.159 & {\cellcolor[HTML]{3A3A3A}} \color[HTML]{F1F1F1} 1.128 & {\cellcolor[HTML]{505050}} \color[HTML]{F1F1F1} 1.117 & {\cellcolor[HTML]{B6B6B6}} \color[HTML]{000000} 1.064 & {\cellcolor[HTML]{7E7E7E}} \color[HTML]{F1F1F1} 1.093 & {\cellcolor[HTML]{676767}} \color[HTML]{F1F1F1} 1.105 & {\cellcolor[HTML]{060606}} \color[HTML]{F1F1F1} 1.156 \\
\bottomrule
\end{tabular}

		\label{tab:methods_vs_yieldprediction}
		\normalsize
	\end{center}
\end{table}


\begin{table}[H]
	\begin{center}
		\caption{Coefficient of determination (R\textsuperscript{2}) of yield prediction}
		\small
		\begin{tabular}{lrrrrrrr}
\toprule
 & RF & OLS\textsuperscript{SCL} & OLS\textsuperscript{all} & MARS & GAM & LASSO & no corrections \\
\midrule
SS & {\cellcolor[HTML]{D2D2D2}} \color[HTML]{000000} 0.431 & {\cellcolor[HTML]{000000}} \color[HTML]{F1F1F1} 0.486 & {\cellcolor[HTML]{222222}} \color[HTML]{F1F1F1} 0.477 & {\cellcolor[HTML]{131313}} \color[HTML]{F1F1F1} 0.481 & {\cellcolor[HTML]{1A1A1A}} \color[HTML]{F1F1F1} 0.479 & {\cellcolor[HTML]{131313}} \color[HTML]{F1F1F1} 0.481 & {\cellcolor[HTML]{767676}} \color[HTML]{F1F1F1} 0.455 \\
SS\textsuperscript{rob} & {\cellcolor[HTML]{D2D2D2}} \color[HTML]{000000} 0.431 & {\cellcolor[HTML]{2A2A2A}} \color[HTML]{F1F1F1} 0.475 & {\cellcolor[HTML]{5F5F5F}} \color[HTML]{F1F1F1} 0.461 & {\cellcolor[HTML]{727272}} \color[HTML]{F1F1F1} 0.456 & {\cellcolor[HTML]{494949}} \color[HTML]{F1F1F1} 0.467 & {\cellcolor[HTML]{494949}} \color[HTML]{F1F1F1} 0.467 & {\cellcolor[HTML]{6E6E6E}} \color[HTML]{F1F1F1} 0.457 \\
DL & {\cellcolor[HTML]{E1E1E1}} \color[HTML]{000000} 0.427 & {\cellcolor[HTML]{9D9D9D}} \color[HTML]{F1F1F1} 0.445 & {\cellcolor[HTML]{A0A0A0}} \color[HTML]{F1F1F1} 0.444 & {\cellcolor[HTML]{A0A0A0}} \color[HTML]{F1F1F1} 0.444 & {\cellcolor[HTML]{7B7B7B}} \color[HTML]{F1F1F1} 0.454 & {\cellcolor[HTML]{7E7E7E}} \color[HTML]{F1F1F1} 0.453 & {\cellcolor[HTML]{F1F1F1}} \color[HTML]{000000} 0.423 \\
DL\textsuperscript{rob} & {\cellcolor[HTML]{F1F1F1}} \color[HTML]{000000} 0.423 & {\cellcolor[HTML]{B3B3B3}} \color[HTML]{000000} 0.439 & {\cellcolor[HTML]{A0A0A0}} \color[HTML]{F1F1F1} 0.444 & {\cellcolor[HTML]{3D3D3D}} \color[HTML]{F1F1F1} 0.470 & {\cellcolor[HTML]{727272}} \color[HTML]{F1F1F1} 0.456 & {\cellcolor[HTML]{8A8A8A}} \color[HTML]{F1F1F1} 0.450 & {\cellcolor[HTML]{EDEDED}} \color[HTML]{000000} 0.424 \\
\bottomrule
\end{tabular}

		\label{tab:methods_vs_yieldprediction_r2}
		\normalsize
	\end{center}
\end{table}

\subsection{OLS-SCL Model Outputs}\label{app:ols-scl-summary}
\lstinputlisting[title= R Summary of the NDVI correction model (c.f. equation \refeq{eq:corr_lm})]{tex/chapters/misc/lm_scl.txt}
\lstinputlisting[title= R Summary of the NDVI correction model (c.f. equation \refeq{eq:corr_lm_res})]{tex/chapters/misc/lm_scl_res.txt}


\todo[inline]{replace space before ref by tilda}
\todo[inline]{check quantile definitions}
\todo[inline]{schwarz weiss färbung der IS tabelle korrigieren}
\todo[inline]{so wenig wie möglich abkürzungen in den fig und table captions}
\todo[inline]{refer to data aviability}
\todo[inline]{abkürzungen Fourier und in tabellen}
